\documentstyle[german,a4]{article}
\begin{document}
\centerline{\Large \bf Die Ticket-Initialisierungsdatei {\tt red.tickets}}
\centerline{\small V1.0 --- \today}
\section{Grammatik der Datei}
\begin{verbatim}
<ticket_file>     ::= <dim_specs>'.'
<dim_specs>       ::= <dim_spec> | <dim_spec><dim_specs>
<dim_spec>        ::= <dim_id><proc_specs>
<dim_id>          ::= '*'<number>NL
<proc_specs>      ::= <proc_spec> | <proc_spec><proc_specs>
<proc_spec>       ::= '@'<proc_list>'('<ticket_list>')'<start_proc_list>NL
<start_proc_list> ::= ';'<number> | ';'<number><start_proc_list>
<proc_list>       ::= <proc_range> | <proc_range>,<proc_list>
<proc_range>      ::= <number> | <number>'-'<number>
<ticket_list>     ::= <ticket_spec> | <ticket_spec>;<ticket_list>
<ticket_spec>     ::= <number> | <proc_list>:<number>
<number>          ::= <digit> | <digit><number>
\end{verbatim}
\section{Eine Beispieldatei}
\begin{tabular}{ll}
1 & {\tt *1}\\ 
2 & {\tt @0(1:4);1}\\
3 & {\tt @1(3);0}\\
4 & {\tt *2}\\
5 & {\tt @0-2(3);1;2;3}\\
6 & {\tt @3(0)}\\
7 & {\tt .}\\
\end{tabular}
\\[2ex]
{\bf Erl\"auterungen:}\\
\begin{tabular}{ll}
1 & Hier beginnt die Belegung f\"ur einen nCube der Dimension 1\\
2 & Prozessor 0 bekommt 4 Tickets f\"ur Prozessor 1\\ 
  & und verteilt zuerst an Prozessor 1\\
3 & Prozessor 1 bekommt f\"ur alle direkten Nachbarn 3 Tickets\\
  & und verteilt zuerst an Prozessor 0\\
4 & Hier beginnt die Belegung f\"ur einen nCube der Dimension 2\\
5 & Die Prozessoren 0,1 und 2 bekommen jeweils f\"ur ihre direkten\\
  & Nachbarn 3 Tickets, 0 verteilt zuerst an 1, 1 an 2 und 2 an 3\\
6 & Prozessor 3 bekommt f\"ur alle direkten Nachbarn 0 Tickets, kann\\
  & also \"uberhaupt nicht verteilen.\\
7 & Hier endet die Datei\\
\end{tabular}
\\[2ex]
{\bf Anmerkungen:}
\begin{itemize}
\item Ist f\"ur die Gr\"o\3e eines Subcubes keine Initialisierung vorgegeben,
wird die Standardbelegung benutzt.
\item Ist f\"ur einen Prozessor keine Initialisierung vorgegeben, wird die
Standardbelegung benutzt.
\end{itemize}
\end{document}

