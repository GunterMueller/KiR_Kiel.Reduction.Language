\section{More about the Editor and the \pired System}

In this chapter we will address a few more things you ought to know
about the \pired editor and the system by which it is backed up. They'll 
make life a little easier
for you when it comes to editing  and running in a step-by-step mode
large programs, and to filing your programs.

There is also a {\mys help file}\index{help file} available which you may consult
if you get lost.
 When pressing the function key F4,
you will get an overview over nine {\mys help keys}\index{help key} which will guide you further down into several groups of editor and system functions. In addition to that
you will also find a complete definition of the syntax of \kir and, in short
  form, a description of how to edit and execute \kir programs.   

\subsection{The Layout of \kir Programs on your Display}

The {\mys syntax-directed editor}\index{syntax-directed editing} of \pired works with a screen of
fixed size on which a given program expression is to be laid out.
Unlike line-oriented editors, it does not include a scrolling
mechanism. Thus, if the full layout of an expression
expands beyond the screen size, the editor must introduce 
abbreviations in various syntactical positions in order to make it fit.

The greatest common divisor program is a very
good-natured example in this sense. Since
both functions used in the program are tail-end recursive, it never 
expands beyond one function call and thus almost (!) always completely
fits onto the screen. However, depending on its actual size you may or may
not have noticed in the expanded function body of {\tt Gcd} that is
being returned after the first reduction step syntactical positions
here and there which are filled with question marks {\tt ?}. They
 are {\mys abbreviations}\index{abbreviating expressions} of the subexpressions which the editor
was unable to lay out in the remainig space.

You can see more of these question marks if the programs are getting
bigger in size or if they are specified in terms of 
recursive functions that are not tail-end and thus recursively expand on 
the screen when executing them. 

A good example in kind is the {\mys factorial function}\index{factorial function}. Initially, it
fits conveniently
into the {\sc fa} field, with lots of space left. 
%-------------------------------------------------------------------
\begin{verbatim}

def 
  F [ N ] = if ( N gt 1 )
            then ( N * F [ ( N - 1 ) ] )
            else 1
in F [ 20 ]

\end{verbatim}
%-------------------------------------------------------------------
When reducing this program step-by-step you will see that it builds up nested
multiplications from left to right, with the application of the 
function definition to the decrementing argument 
shifting away in the same direction.
%--------------------------------------------------------------------
\begin{verbatim}


( 20 * ( 19 * ( 18 * ( 17 * ap def 
                                 F [ N ] = if ( N gt 1 )
                                           then ( N * F [ ( N - 1 ) ] )
                                           else 1
                               in F
                            to [ 16 ] ) ) ) )

\end{verbatim}
%----------------------------------------------------------------------
 Rather than letting it spill over on the right-hand
side of the screen, the editor lets this application `disappear'
 under a question mark so that a full {\mys top-level expression}\index{top-level
expression} (i.e., the
outermost syntactical structure) remains visible.
%-----------------------------------------------------------------
\begin{verbatim}


( 20 * ( 19 * ( 18 * ( 17 * ( 16 * ( 15 * ( 14 * ( 13 * ? ) ) ) ) ) ) ) )

\end{verbatim}
%--------------------------------------------------------------------

 As you proceed with further reductions, the expression continues
 to expand under the question mark 
(provided the argument value is large enough), but 
what you can see of it on the screen remains as it is. 

In order to display what is hidden behind a question mark you simply need to
move the cursor either directly under it or to a constructor that is close to it, say
the innermost opening parenthesis by which it is preceded. If you now press the
function key F3, the {\sc fa} field fills with the new cursor expression (i.e., the cursor
re-appears in the left upper corner of the screen), and the
surrounding context disappears. This expression may contain more question marks
behind which you may look by proceeding in the same way. In doing this, you
recursively {\mys climb down}\index{climbing down the syntax tree} the syntax tree of your (intermediate) program to the sub-expression that you are interested in. As explained before, you may perform reductions in this part of the program
without worrying about undesirable side-effects in other parts: there simply aren't any.

The easiest way of {\mys climbing up}\index{climbing up the syntax tree} in the syntax tree and thus returning
to larger contexts is to move the cursor to the topmost constructor of the expression that
is currently in  the {\sc fa} field (if it is not already there) and to hit the {\tt up} key.  Every time you do this,
you move upwards to the next father node of the syntax tree as usual,  but you also
 get displayed on your screen the full
expression of which this node is the topmost constructor, possibly with increasingly  larger
 sub-expressions again  disappearing behind question marks as you move on in this direction.
If you wish to return in one step to the top of the entire program, simply type
on your keyboard {\tt CTR-n}. There is no other way of moving upwards, e.g.,
by reversing the steps you have taken downwards. 

Another problem for which a remedy is necessary arises in connection with large sets of 
fairly complex {\mys function definitions}\index{nested function definitions}, possibly nested over several levels. 
These definitions may easily take up the entire space that is available on the screen, 
in many cases requiring lots of abbreviations in the form of question marks. This situation is
getting worse when performing just a few reduction steps after which an expanded intermediate
program is to be displayed. It will likely contain, in even more abbreviated form,
 several basically identical copies of
{\tt def} constructs which have been (recursively) substituted for function identifiers in
 the original goal expression. All abbreviations that occur in such a
program representation are introduced by the
editor, mainly based on the screen space that is left in a particular situation, not
with respect to components that are of primary interest, say
 the changeable program parts where reductions are actually 
taking place. Thus, valuable screen space   is generally  wasted for things that you
may not be interested in, whereas the important parts disappear behind question marks.
 
In order to have a little more {\mys control}\index{controlling abbreviations} over what is being displayed and how it is laid
out, the editor provides under the command mode F8
a few commands that introduce abbreviations in dedicated places (syntactical positions)
of a program. 
They include
\begin{description}
\item[{\tt abb\_def}]  which leaves the topmost level of function definitions intact but abbreviates
all others;
\item[{\tt abb\_idef}] which abbreviates only function definitions that are local to
others (i.e. {\tt def} constructs in {\tt def} constructs);
\item[{\tt dgoal}] which displays the goal expression of an abbreviated {\tt def} construct;
\item[{\tt dfct}] which displays the headers of abbreviated function definitions;
\item[{\tt small}] which reduces the space demand of an expression by suppressing 
blanks in the layout whereever possible without creating ambiguities;
\item[{\tt vertical}] which changes the layout of syntactical constructs from
horizontal to vertical;
\end{description}

Once you are in the command mode (after having pressed F8), you may type into the input line any one of these
commands followed by {\tt on} or {\tt off} (with one blank in between), whichever way you want it.
Any combination of these {\mys layout control modes}\index{layout control mode}\index{layout mode} is legitimate and applies to all components
of the expression that is actually in the {\sc fa} field. A particular
 mode combination is sustained until explicitely changed, i.e., it may apply to several programs
on which you work in succession.

The best way of getting acquainted with these {\mys layout commands}\index{layout command} is by means of
a suitable program example which in a nutshell clearly exposes their effects.
A good one is the following nonsense program:
%---------------------------------------------------------------
\begin{figure}
\begin{verbatim}

ap def 
     F [ U, V ] = def 
                     G [ U, V ] = def 
                                     H [ U, V ] = G [ F, H ]
                                   in F [ G, H ]
                   in G [ F, G ]
   in F [ F, F ]
to [ C ]

\end{verbatim}
\end{figure}
%------------------------------------------------------------------
It consists of three nested function definitions which in some seemingly
 weird way
recursively call on each other, making
full use of the {\mys higher-order function}\index{higher-order function} feature of \kir:
all applications are solely
specified in terms of function identifiers both in function and
argument positions, i.e., as applications of functions to functions which
 return new functions as results. So, it is worth the while to study
 the program behavior
also from this point of view.

 When performing a few reductions step by step, you
will note that the program expands to some extent due to the substitution
of function identifiers by the respective complete function definitions, but 
otherwise it behaves reasonably well insofar as it cycles, in periods of
six reduction steps, through the same sequence of intermediate programs.
Thus, the program is in fact tail-end recursive. Since it obviously never
terminates, it is also meaningless.
 
However, let's return to the original purpose of introducing this program.
Since it produces in a rather intricate way nesting levels of
function defintions and function applications, it is also ideally suited
to study on any of the intermediate programs
 which parts of it are being suppressed or exposed by turning the
 {\tt abb\_def}, {\tt abb\_idef},
{\tt dgoal} and {\tt dfct} modes on and off, and how the layout changes when
activating or deactivating the {\tt vertical} mode\footnote{The {\tt small} mode has no effect on the layout of this program, you will have to wait until we
introduce more \kir constructs to find out what it does.}.

Having played with these layout modes for a while, you may have gotten a little confused as to which of them are actually on and which are off. To find out
what the current situation is, you may enter the command {\tt editparms}
which brings up on your screen a number of {\mys editor parameters}, of
which those that are of interest at the moment are at the bottom of the list.
Hitting the {\tt enter} (or {\tt return}) key gets you back to the 
standard \pired user interface. 

\subsection{Deleting Terms from a \kir Program}

By means of the above program we may also get a little more specific
 about the {\mys deletion}\index{deleting sub-terms} of  (sub-)
 terms from an existing program. We indicated in the
preceding section that this can be done by
making a term to be deleted the cursor expression (i.e., move
the cursor to its topmost constructor) and then hitting the {\tt delete}
key on your keyboard: the term will disappear and the surrounding 
program will generally change its syntactical structure accordingly. So, let's
try a few {\tt delete} operations on the program to see what happens.

The best way of systematically studying the effects of the {\tt delete}
function  is to
 set out with the components of the innermost {\tt def} construct,
say, by deleting the formal parameter {\tt U} from the header of the
function {\tt H} or by deleting the argument {\tt H} from the application
on the right-hand side of this definition. In either case, the term simply
disappears together with the comma by which it was separated from the
other term inside the brackets. Next you may wish to delete the entire
application {\tt G[F]} from the right-hand side of the definition. So,
you move the cursor under the function term {\tt G} and hit the {\tt delete}
key, whereupon the editor returns the empty term {\tt \#} in its place.
The same happens if you do this to the goal expression {\tt G[F,G]}. In both
cases, the selected term disappears, but otherwise the syntactical 
structure of the {\tt def} construct is sustained for there are still
fragments left that cannot simply go down the drain with what you 
intended to take away. 

Moving the cursor under a {\tt def} constructor and hitting the {\tt delete}
key replaces the entire {\tt def} construct by what originally was its
goal (or body) expression, i.e., it deletes all the function definitions
and thus turns into free variables all occurrences of function identifiers
left over in the goal. You may observe this effect over several steps
 by removing the {\tt def}
constructs in your program systematically from innermost to outermost.
If there is just one function defined under a {\tt def}, the same 
can be accomplished by moving the cursor under the function name
in the header of the definition. If it comprises more
function definitions, only the one thus selected
is removed from the entire construct.

As a general rule, the {\mys {\tt delete} function}\index{delete function}
\begin{itemize}
\item removes from an $n$-ary structure the selected component and
thus shrinks it to a structure of arity $(n-1)$, provided the component
is one of those whose numbers may vary, e.g., a list element,
a sub-term of an application,
 or a function definition in a {\tt def} construct;
\item replaces with an empty term {\tt \#}
 an essential component of a construct, e.g., the
goal expression in a {\tt def} construct, the right-hand side of a
function definition, or any component in a construct with a
fixed arity, e.g., in an {\tt if\_then\_else} clause. 
\end{itemize}

\subsection{Some Other Useful Editor Commands}

While strictly syntax-directed editing greatly facilitates writing
syntactically correct programs, there are also, and inevitably so, some inconveniences
associated with it, of which one is the movement of the cursor, and the other one
relates to searching for and replacing certain (sub-) expressions, possibly in several places.

The default cursor movement as explained on pages 10 and 11 in fact follows 
the $n$-ary tree structures built up by the constructor syntax that is used for
the internal representation of \kir-programs as $\lambda$-terms.
 Since this syntax does not in an obvious way
relate to the syntax of \kir~ as you see it on  your screen, 
the cursor movements effected by pressing
the respective keys sometimes appear to be rather unusual and unexpected relative to what you
are used to from line-oriented editing.  

To remedy this problem, you have available under the command mode (to be entered with
the function key F8) an editor command {\tt curmode[par]} which, when used 
with the parameter
\begin{description}
\item[{\tt par = off}] enables the syntax-oriented (tree traversing) cursor movement;
\item[{\tt par = on}] enables a line-oriented mode in which the cursor is moved as
indicated by the arrows on the keys.
\end{description}

Another three editor commands may be used to search for and replace (sub-) expressions
by others. They are
\begin{description}
\item[{\tt findexpr[string]}] which searches, from the current cursor position downwards,
for the first occurrence of a subexpression starting with {\tt string}, i.e., the subexpression
need not be fully specified. The command {\tt next} may be used to search for subsequent
 occurrences.  
\item[{\tt find}] works as {\tt findexpr}, but takes as search pattern the topmost expression
held in the first auxiliary buffer.
\item[{\tt replace[expr\_1;expr\_2; mode]}] which searches, again from the current cursor
position downwards, for occurrences of the subexpression {\tt expr\_1} and replaces them
with the expression {\tt expr\_2}. The parameter {\tt mode} may be set to
\begin{description}
\item[{\tt mode = all}] to have replaced  all occurrences of {\tt expr\_1} from the cursor 
position downwards; 
\item[{\tt mode = ask}] in which case the editor will ask you what to do next, with the
following options:
\begin{description}
\item[y] make a replacement;
\item[n] skip and move on to the next occurrence of the search expression;
\item[a] terminate the {\tt ask} mode and do all replacements from the current
position downwards;
\item[q] quit the {\tt replace} command.
\end{description}
\end{description}
\end{description}
If the parameters of the {\tt replace} command are not specified, the editor will ask
you to do so.
After completion of a {\tt replace} command the system tells you on the message line
how many replacements have actually been made.

\subsection{The Filing System}

In order to provide some essential {\mys file services}\index{file service}, \pired makes use of
the underlying UNIX file system. All files which you create while
working with the system are stored under the same directory
under which you've  called the {\mys executable file}\index{executable file} {\tt reduma} for
\pired itself. If \pired is installed under your own user
directory,  then you can find {\tt reduma} under the path name
$$
{\tt ../user/red/red.o}\;\;.
$$
Now, there are basically two ways of {\mys calling}\index{calling \pired} {\tt reduma}. You can do this
\begin{itemize}
\item either directly under your {\tt user} directory with
$$
{\tt <unix\_prompt> : red/red.o/reduma}\;\;,
$$
in which case all files are placed under {\tt user};
\item or by first climbing down the directory path with
$$
{\tt <unix\_prompt> : cd\; red/red.o}\;\; 
$$
and then calling it with
$$
{\tt <unix\_prompt> : reduma}\;\;,\;\;
$$
in which case you'll find all your files under the directory
path {\tt user/red/red.o}\footnote{Note that the compiled version
 \pired$^+$ requires a special file directory {\tt ed/} under the 
directory {\tt red/} (and at the same level as the executable file
 {\tt reduma*}) under which all editor files are being allocated.
 If not yet existing, this directory must be opened up with
the UNIX command {\tt mkdir ed}, otherwise no new editor files can be created under the output mode.}.
\end{itemize}

The first item applies equivalently if \pired is installed under another
 user's directory which you
are allowed to access. Then you may call it with
$$
{\tt <unix\_prompt>:\sim/other\_user/red/red.o/reduma}\;\;,
$$
but all files which you create are again allocated under your own
{\tt user} directory. 

There are several {\mys file types}\index{file type} which are distinguished by appropriate
{\mys default extensions}\index{default extension} to the file names. These extensions are automatically 
added by the editor in compliance with the modus of creation. 
 
You know already about the most important file type (or format),
 so-called {\mys editor files}\index{editor file}, whose names are extended by the suffix 
{\tt .ed}. This file type may be used to create {\mys libraries}\index{\kir library} of 
\kir expressions, e.g., of complete programs or
frequently used function definitions, 
or as an intermediate store for program 
components in the process of designing large programs. Creating
and accessing edit files is controlled by the function keys
F10 and F2. The former enables the {\mys output mode}\index{output mode} under which the
current cursor expression (which
may be different from the {\sc fa}-field) may be written into a
 new or an existing edit file;
the latter enables the {\mys input mode}\index{input mode} under which the current
cursor expression may be replaced by the \kir expression held in
an existing edit file. 

Assembling  large programs from several \kir expressions that are stored away 
in edit files can be elegantly supported by yet another input mode. Rather
than explicitely copying the files, they may simply be referenced from within
a program by so-called {\mys dollar variables}\index{dollar variable}\index{referencing edit files} of the form 
{\tt \$file\_name}, i.e., by their names preceded by a dollar sign. Before
actually executing a  program, 
the pre-processor of \pired substitutes all dollar variables
by the respective file contents. This is done systematically
from outermost to innermost since all
expressions thus inserted may contain further references to
other edit files. Note that some care must be taken on the part
 of the programmer to avoid {\mys cyclic references}\index{cyclic 
edit file references}.

You may test this feature on your {\mys greatest common divisor program}\index{greatest common divisor program}, say,
by first copying the function body of
{\tt Gcd} into a file named {\tt Body}, and then replacing it with 
{\tt \$Body}. The program then looks like this: 
%--------------------------------------------------------------------
\begin{verbatim}

def 
  Modulo [ X , Y ] = if ( X le Y )
                     then X
                     else Modulo [ ( X - Y ) , Y ] ,
  Gcd [ X , Y ] = $Body
in Gcd [ 6 , 4 ]

\end{verbatim}
%----------------------------------------------------------------------
If you now trigger the system to
 perform one reduction step, you will see that the complete body of
{\tt Gcd}  again shows up in  all occurrences of the {\tt def} construct,
i.e., the program behaves exactly as if the body would have been
directly included.

At this point you may wonder why there isn't a dynamic version of this input
mode which expands dollar variables by the respective edit file contents
 at run\_time, and only if and when absolutely necessary. Though this mode may
save a lot of memory space, there are severe problems with it
which largely arise from potential occurrences of free variables in expressions
that are filed away. They cannot be naively substituted into a running program
without a high risk of changing binding scopes and thus its meaning 
(or semantics). 

Another type of input/output files are so-called
{\mys red files}\index{red file}. The \kir expressions stored in them are treated as 
{\mys macros}\index{macro expression} in the following sense: whenever they are reduced in the 
context of a large program, they decrement the {\mys reduction counter}\index{reduction counter}
just by one (or count as one instance of reduction), irrespective of the
actual number of reduction steps that must 
be performed. They are distinguished by the suffix {\tt .red} 
following the file name. You may use (under the command mode F8) 
the {\mys editor commands}\index{editor command}
\begin{description}
\item[{\tt store[file\_name]}] to store the cursor expression under
the file {\tt file\_name.red};
\item[{\tt load[file\_name]}] to  overwrite the cursor expression with
 the contents of the file {\tt file\_name.red} .
\end{description}
Similar to edit files, these red files may be referenced from within a
 program with
{\tt \_file\_name}, i.e., with an underscore rather than a {\tt \$}
preceding the file name.

You may also copy your cursor expression into an {\mys ASCII file}\index{ASCII file}, using the
command {\tt print[file\_name]}, or overwrite the cursor expression with
the contents of an existing ASCII file for \kir expressions, using
{\tt read[file\_name]}. ASCII file names have the suffix {\tt .asc}
appended to them.

There is yet another file type called {\mys prettyprint}\index{prettyprint file} in  which 
\kir expressions may be stored in the same layout and fond in which
you can see them on your screen. These files may either be directly
printed or included in other text files. In fact, all the example programs  
in this manual are, through prettyprint files, copied from the 
screen.  The command {\tt pp file\_name} creates a new (or overwrites
an existing) prettyprint file with the cursor expression and appends the
suffix {\tt .pp} to the file name.

\subsection{Setting the System Parameters}

In order to be able to deal with certain {\mys memory space problems}\index{memory space management}
that may occur when running large programs, you need to have at least 
some rudimentary understanding of the internal workings of \pired.
There are also some particularities about the {\mys arithmetic subsystem}\index{arithmetic subsystem} 
 used in \pired that you ought to know about. 

\pired is an {\mys applicative order graph reduction system}\index{graph reduction system} which uses
the following {\mys run\_time data structures}\index{\pired run\_time environment}~\cite{schm91b,gaer91}:
\begin{itemize}
\item a {\mys program graph}\index{program graph} (held in a {\mys heap
segment}\index{heap}) whose inner and outer {\mys nodes}\index{graph node} roughly
correspond to {\mys constructors}\index{constructor} and string representations of atomic 
expressions (such as variables, constants, primitive function symbols etc.),
respectively, and whose edges correspond to pointers connecting these
nodes;
\item an array of {\mys node descriptors}\index{node descriptor} which, as an integral part
of the program graph, contains information
about {\mys class}, {\mys type} and {\mys shape} of the (sub-) expressions represented by
the graph nodes;
\item one or several {\mys run\_time stacks}\index{run\_time stack} which accommodate the {\mys activation records}\index{activation record}, {\mys workspaces}\index{workspace} etc. for
instantiations of defined functions; 
\item (as part of the interpreter version \pired$^*$)
a {\mys shunting yard}\index{shunting yard} comprising three stacks
about which graph structures are {\mys traversed}\index{traversing graphs} in search for instances
 of reductions.
\end{itemize}
Both the heap and the {\mys descriptor array}\index{descriptor array} are managed based on a 
{\mys reference counting scheme}\index{reference counting} which releases as early as possible
 the space occuppied by graph structures that are no longer needed.
 However, as heap space is allocated and de-allocated in pieces of variable
 lenghts, it may require
compaction from time to time in order to satisfy a particular space
demand for the placement of a new (sub-) graph.
 All descriptors have the same size 
and may therefore be placed into any array entry.

When calling \pired, the sizes of the memory segments which accommodate
these run\_time structures are {\mys initialized}\index{\pired initialization}\index{initialization file} from a {\tt red.init} file
with pre-specified values. The {\mys editor command}\index{editor command} {\tt redparms} will display
on the screen the {\mys memory configuration}\index{memory configuration} (together with some other system
parameters) with which you are working.

You have several commands available to modify this configuration, say,
if one of your programs has been aborted because it has run
 out of space\footnote{You will always be notified in the message line
of the exact cause of the problem, i.e., in which run\_time structure the
overflow has occured.}: 
\begin{description}
\item[{\tt heapsize no\_of\_bytes}] specifies, in numbers of bytes,
the size of the heap segment; 
\item[{\tt heapdes no\_of\_entries}] specifies the number of entries in
the descriptor array;
\item[{\tt qstacksize no\_of\_el}] specifies, in numbers of stack elements
(of four bytes), the size of the larger stacks of the shunting yard;
\item[{\tt mstacksize no\_of\_el}] specifies the size of the smaller stack
of the shunting yard;
\item[{\tt istacksize no\_of\_point}] specifies, in numbers of pointers
(again of four bytes length), the size of the run\_time stack.
\end{description}
You should change these parameters if and only if you really understand
what you are doing, i.e., you know about the internal program representation
and the way programs are reduced in \pired. However, as a rule of thumb,
you should have at least 500 kBytes of heap space allocated to run 
programs of modest size, and the ratios between all other
parameters should be roughly as follows:
$$
\begin{array}{l}
heapsize\;:\;heapdes\;:\;qstacksize\;:\;mstacksize\;:\;istacksize\;=
\phantom{xxxx}
\\ \multicolumn{1}{r}{500\;:\;5\;:\;20\;:\;10\;:\;20\;.}
\end{array}
$$
As already indicated, another set of {\mys system parameters}\index{system parameters} relates to
the way {\mys arithmetic operations}\index{arithmetic operation} are performed in \pired. You have the
choice between a {\mys decimal mode}\index{decimal mode}\index{binary mode} which works with variable length
number representations\footnote{Internally, all numbers
are represented relative to a base of 10000 (as opposed to
 BCD codes) for efficiency reasons.} and a {\mys binary mode}\index{binary mode} which,
 as usual, works with the fixed integer and floating point formats of
 your host system. The command
\begin{description}
\item[{\tt varformat}] enables the decimal mode which performs all arithmetic
operations with potentially unlimited {\mys precision}\index{precision of arithmetic operations} (it is the mode that
you get as a default option when starting \pired);
\item[{\tt fixformat}] enables the binary mode, under which all decimal
numbers specified in your high-level \kir program are automatically 
converted to floating point formats if they include a decimal point, 
and to integer formats if they don't.
\end{description}
In the decimal mode you can compute
breathtakingly long numbers which fill the entire {\sc fa}-field,
e.g., if you run the factorial program with an argument value of 1000
or more. However, you may also wish to, and in the case of division you have
to (!), put an upper limit on the precision of your result values,
 i.e. primarily on the number of digits to the right of the decimal point.
So, \pired has parameterized precisions for the results of 
multiply and divide operations, but also with respect to the maximal number of
digits that are displayed on the screen. The commands which may be used to
re-define them other than initialized at start-up time are:
\begin{description}
\item[{\tt div\_prec no\_of\_digits}] for the divide precision;
\item[{\tt mult\_prec no\_of\_digits}] for the multiply precision;
\item[{\tt trunc no\_of\_digits}] for the upper limit on
the number of digits
displayed on the screen.
\end{description}
There is one more parameter called {\tt beta\_count} by which you may
determine what the {\mys reduction counter}\index{reduction counter} is supposed to keep track of. You have
the choice between
\begin{description}
\item[{\tt beta\_count on}], in which case are counted  only reductions of
 defined function applications which require
 the instantiation of formal by actual parameters (or $\beta$-reductions);
\item[{\tt beta\_count off}], in which case are counted all instances of
reduction (i.e., including applications of primitive functions).
\end{description}
All changes effecting the system parameters are immediately confirmed
on the message line. Alternatively, you may inspect all parameters at
once by means of the command {\tt redparms}. It will also show you
the times used up by your last sequence of reductions for pre-processing,
processing and post-processing.

\subsection{Defining the Function Keys on your Keyboard}\index{function key definition}

As pointed out in chapter 3, the function keys used by \pired are
{\mys logical}\index{logical function key}, i.e., the functions they are supposed to perform
are not necessarily assigned to the {\mys physical 
function keys}\index{physical function key} that you find on
the keyboard of your workstation or terminal. Some of these keys
may be used otherwise by the underlying host system, and there
may not even be as many physical keys as \pired has logical keys.
The easiest way to get around this problem is to use only the
logical keys by typing {\tt :n } (where {\tt n} is the logical key
number), which echoes in the input line as {\tt Fn}.

However, if you wish to assign at least some of the logical keys, e.g.,
those most frequently used, to physical keys on your keyboard, you may
do so by entering the command mode and then typing the command 
{\tt defkeys}. It will bring up on your display screen the actual table
of {\mys key definitions}\index{key definition table} as it is currently available on your system.
This table comprises three columns, of which 
\begin{itemize} 
\item the first simply enumerates all logical keys of \pired, with
indices running from {\tt 0} to {\tt 20};
\item the second lists the logical keys associated with these indices,
with the first five keys controlling the cursor movement, and the 
remaining 16 keys defining the function keys (of which {\tt SHIFT\_F1
.. SHIFT\_F8} stand for {\tt F9 .. F16}, respectively);
\item the entries of the third column may or may not 
contain some weird codes which identify the physical keys assigned
 to the particular logical keys (if an entry is empty, then no
physical key is assigned).
\end{itemize}
Typical table entries on a SUN workstation look like this:
\begin{verbatim}

  0 :       UP : 0x1b 0x5b 0x41

  8 :       F4 : 0x1b 0x5b ox32 0x32 0x37 0x7a

 14 : SHIFT_F2 : 0x1b 0x5b 0x32 0x33 0x32 0x7a

\end{verbatim}
 
On top of this table you are asked to enter the index of the logical
key to which you wish to assign a physical key. Having done so,
you are asked to hit the particular physical key, followed by two 
blanks, which will enter the code for this key in the table (thereby
 possibly overwriting an existing assignment). This works only for
physical function keys (including those reserved for cursor movement)
that are not yet otherwise pre-assigned by the host system.

 You may
do these assignments for several keys in succession. If you are
 finished, 
simply hit the return key, and you will get back to the standard
\pired screen layout. The key assignments, together with all other
system parameters,  are entered into the \pired 
{\mys initialization file}\index{initialization file} {\tt red.init}. 

