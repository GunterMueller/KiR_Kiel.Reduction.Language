\section{A Kernel of \kir}

Non-trivial functional programs are {\mys expressions}\index{expression} composed of 
(mutually recursive) {\mys function definitions}\index{function definition} and {\mys function applications}\index{function application}~\cite{abel85,macq87,bird88}.
They are {\mys pure algorithms}\index{pure algorithm} which, in contrast
to conventional imperative programs, just specify what is to be
computed rather than including detailled schedules of how to actually
execute them on some underlying machinery~\cite{back78}. 
The elementary operations to be performed are only partially ordered,
merely reflecting the 
logical structure of the appplication
problem. There is no notion of a memory either: as in algebra, 
variables are `unknowns' which may serve
as placeholders for other expressions to be substituted
for them, but they do not represent changeable values. 

A functional program written in \kir typically has the syntactical form
$$
\begin{array}{lclcl}
&& {\tt def}
\\
&& \phantom{{\sc in}\;\;}\vdots
\\
&& \phantom{{\sc in}\;\;}{\tt F[X\_1, \ldots, X\_n]}&=&{\tt F\_expr}
\\
&& \phantom{{\sc in}\;\;}\vdots\\
&& \phantom{{\sc in}\;\;}{\tt H[Y\_1, \ldots, Y\_m]}&=&{\tt H\_expr}
\\
&& \phantom{{\sc in}\;\;}\vdots \\
&& {\tt in\;\; S\_expr\;}.
\end{array}
$$
This {\sc define}-construct specifies a set of {\mys recursive functions}\index{recursive function}
in terms of which the value of the so-called {\mys goal-expression}\index{goal expression}
{\tt S\_expr} is to be computed.
We refer to the left-hand sides of the function
definitions as {\mys function headers}\index{function header}
and to the right-hand sides as {\mys function body
expressions} (or {\mys function bodies}\index{function body} for short).

For instance,
the function header {\tt F[X\_1,\ldots, X\_n]} declares {\tt F} to be a
function of {\tt n} {\mys variables}\index{variable}
(or {\mys formal parameters}\index{formal parameter})
{\tt X\_1,\ldots, X\_n}. The function body
expression {\tt F\_expr}
specifies an algorithm for the computation of {\mys function
values}\index{function value} of {\tt F} in terms of
\begin{itemize}
\item the variables {\tt X\_1,\ldots, X\_n} and {\tt F} (in which case the
function is recursive);
\item applications of subfunctions of {\tt F};
\item applications of functions declared at the same level as {\tt F},
e.g., the function {\tt H}.
\end{itemize}

These {\sc define}-{\mys constructs}\index{{\sc define}-construct}
may recursively appear as function body
expressions to define local subfunctions.
For instance, {\tt F\_expr} may be specified as
$$
\begin{array}{lcl}
{\tt def} \\
\phantom{{\sc in}\;\;}\vdots \\
\phantom{{\sc in}\;\;}{\tt G[V\_1, \ldots, V\_m]}&=&{\tt G\_expr} \\
\phantom{{\sc in}\;\;}\vdots \\
{\tt in}\;\; {\tt F\_expr'}\;,
\end{array}
$$
where the function {\tt G} and all other functions defined at this
and lower
levels are {\mys subfunctions}\index{subfunction}
of {\tt F}. All variables defined in a
particular function are also defined in its subfunctions.
All {\mys function names}\index{function name} (or {\mys function identifiers})\index{function identifier} specified under a particular {\sc define}-construct
should be unique\footnote{ In fact, \pired allows that some or all functions be named 
identically. Applied occurrences of function identifiers must then be 
preceded
by backslashes whose numbers uniquely identify the positions of the respective
function definitions in the {\sc define} constructs.}.

%-------------------------
\arraycolsep 5pt
%-------------------------
A more appropriate notion for the status of a variable is
that of its {\mys binding scope}\index{binding scope}.
With respect to a program specified in the above form, scopes may
be roughly defined as follows:
\begin{itemize}
\item the
formal parameters {\tt X\_1,\ldots, X\_n} are {\mys bound}
in the function {\tt F} but {\mys free}\index{bound variable}\index{free variable}
in its subfunctions, e.g., in the function {\tt G};
\item the formal parameters {\tt V\_1,\ldots, V\_m} are bound in
{\tt G} but, again, free in the subfunctions of {\tt G};
\item function identifiers are mutually recursively bound in all
functions defined at the same hierarchical level; they are
also bound in the goal expression: {\tt G} is bound in
{\tt F\_expr'}, and {\tt F} and {\tt H} are bound in {\tt S\_expr}.
\end{itemize}
We will henceforth refer to
variables which are free in subfunctions but bound higher up
as being {\mys relatively free}\index{relatively free variable}.

In addition to functions that are given names, \kir also
admits non-recursive {\mys nameless} (or {\mys anonymous)
functions}\index{anonymous function} of the form
$$
{\tt sub\; [X\_1,\ldots, X\_n]\; in\;Expr}\;\;\;,
$$
where {\tt X\_1,\ldots, X\_n} again denote formal parameters
and {\tt Expr} denotes the function body expression.

Other expressions may be recursively constructed from
\begin{itemize}
\item {\mys function applications}\index{function application},
which generally are denoted as
$$
\begin{array}{rl}
{\tt ap} & {\tt Func} \\
{\tt to} & {\tt [\;Arg\_1,\;\ldots\;,\;Arg\_n]}
\end{array}
$$
or alternatively as
$$
 {\tt Func[Arg\_1,\ldots, Arg\_r]}\;\;\;,
$$
where {\tt Func} may be (any legitimate \kir expression which
is expected to evaluate to)  the name of a function defined
by an equation, a nameless function (in which case only the
former representation will be accepted),
or a symbol representing a primitive function, and
{\tt Arg\_1,\ldots, Arg\_r} are argument expressions\footnote{\pired sometimes exchanges, in the course of evaluating a program,
 these representations against each other as a matter of convenience:
 it usually prefers the former if {\tt Func} itself is an application, a
nameless function, or a composite expression, and the latter if it is a function name or
a variable.};
\item
%-------------------------
\arraycolsep 2pt
%-------------------------
{\sc let} {\mys clauses}\index{{\sc let}-clause} of the general form
$$
\begin{array}{llcl}
{\tt let} & \; & \; & \;
\\ &\phantom{{\sc in}\;\;}\vdots \\
&\phantom{{\sc in}\;\;}{\tt X}&=&{\tt X\_expr} \\
&\phantom{{\sc in}\;\;}\vdots\\
&\phantom{{\sc in}\;\;}{\tt Y}&=&{\tt Y\_expr} \\
&\phantom{{\sc in}\;\;}\vdots \\
{\tt in} \; & {\tt Expr} & \;&\;
\end{array}
$$
which specify the (values of) {\tt X\_expr} and {\tt Y\_expr} as substitutes
for free occurrences of the variables {\tt X} and {\tt Y}, respectively,
in {\tt Expr};
%-------------------------
\arraycolsep 5pt
%-------------------------
\item {\sc if-then-else} {\mys clauses}\index{{\sc if-then-else}
clause} of the form
$$\begin{array}{ll}
{\tt if} & {\tt Pred\_expr}\\
{\tt then} & {\tt True\_expr}\\
{\tt else} & {\tt False\_expr}
\end{array} $$
of which {\tt Pred\_expr}, {\tt True\_expr} and {\tt False\_expr} are
also referred to as the {\mys predicate}, the {\mys consequence} and
the {\mys alternative}\index{predicate}\index{consequence}\index{alternative}, respectively.
\item {\mys lists} or {\mys sequences of
expressions}\index{list}\index{sequence}, denoted as
$$
\mbox{{\tt $<$Expr\_1,\ldots, Expr\_n$>$}}
$$
as a means for representing {\mys structured objects}\index{structured object}.

\end{itemize}

%-----------------
\sloppy % o.k. mit
%-----------------
The set of {\mys primitive functions}\index{primitive function} of \kir includes
the familiar binary {\mys arithmetic}, {\mys logical} and {\mys relational
operators}\index{arithmetic operator}\index{logical operator}\index{relational operator}, applications of which are usually denoted in (or by \pired
turned into)
parenthesized form as\footnote{Note that {\tt Op} need not necessarily
be a binary operator right away: it may be any legitimate \kir expression which eventually reduces to one.}
$$
{{\tt (Arg\_2}\;{\tt Op}\; {\tt Arg\_1)}}\;\;\;.
$$
There are also a number of primitive
structuring and predicate functions applicable to lists which
we will introduce later on.
 
As a first example, here is a simple \kir program which computes the
{\mys greatest common divisor}\index{greatest common divisor program} of two integer numbers:
$$
\begin{array}{lclll}
{\tt def} && \\
\phantom{{\sc in}\;\;}{\tt Gcd[X, Y]}&=& {\tt if}&
\multicolumn{2}{l}{
{\tt (X\; eq\;Y)}} \\
&& {\tt then}& {\tt X} \\
&& {\tt else}& {\tt if}&{\tt (X \; gt\; Y)}\\
&& & {\tt then}&{\tt Gcd[Y, Modulo[X, Y]]}\\
&& & {\tt else}&{\tt Gcd[X, Modulo[Y, X]]}\\
\phantom{{\sc in}\;\;}{\tt Modulo[X, Y]}&=& {\tt if}&
\multicolumn{2}{l}{
{\tt (X\; le\; Y)}
}\\
&& {\tt then}& {\tt X} \\
&&  {\tt else}&
\multicolumn{2}{l}{{\tt Modulo[(X - Y), Y]}}
\\ \multicolumn{3}{l}{{\tt in\;\;Gcd[4,6]}}
\end{array}
$$
It consists of two recursive functions, of which {\tt Gcd[X,Y]} 
is directly called in the goal-expression to
return the greatest common divisor of actual integer
values substituted for {\tt X} and {\tt Y} (which in this 
particular example are {\tt 4} and {\tt 6}, respectively),
and {\tt Modulo[X,Y]} is called
as a subfunction of {\tt Gcd[X,Y]} to compute the remainder of {\tt X}
after division by {\tt Y}.

As a convention, \kir uses lower case letters to denote {\mys keywords}\index{keyword}
such as {\tt def, if ,then, else, in } etc., whereas all identifiers
introduced by the programmer (function names and formal parameters) 
start with upper case letters. As we will see in the next section,
this is nothing that you need to worry about for this is taken care of
by the \pired editor\footnote{However, be warned that if you accidently
type any of the keywords with upper case letters, say {\tt If}, {\tt Then}, etc.,
they are taken as variables.}.

It may be noted that \kir programs (as programs of any other functional
language, for that matter) exhibit the same overall structure
as imperative
programs. The set of function definitions replaces the set of
procedure declarations, and the goal expression replaces the statement
part. This structure recursively extends into all local definitions (or
declarations). {\sc let}-expressions combine local {\mys variable declarations}\index{variable declaration}
with {\mys single assignments}\index{single assignment} of values (of expressions). 

Since \kir is an {\mys untyped language}\index{untyped language}, its programs do not include
explicit type declarations.
Type compatibility checks between primitive functions and the
operands to which they are actually applied are dynamically
performed by the run-time
system of \pired based on type-tags that are carried along with all
objects of a program.   

With this basic knowledge about \kir at hand, we are now ready to have
a closer look at the workings of \pired.
