\section{Full $\beta$-Reductions and Higher-Order Functions}
Let us now return to writing programs in \kir since there is lots more
to learn about its constructs and features, particularly about those 
that are not available in most other functional or function-based languages.

One concerns the implementation of a 
{\mys $\lambda$-calculus}\index{$\lambda$-calculus} as an essential pre-requisite for
simple {\mys symbolic computations}\index{symbolic computation}, particularly when it comes
to exploiting the full 
potential of {\mys higher-order functions}\index{higher-order function}, including {\mys self-applications}\index{self-application}
and {\mys partial applications}\index{partial application}. Parameter passing by full
{\mys $\beta$-reductions}\index{$\beta$-reduction} guarantees the orderly
resolution of {\mys naming conflicts}\index{naming conflict} between (relatively) free
and bound variable occurrences.  This problem potentially  arises
whenever, in the course of reducing defined function applications,
argument terms other than basic values are being substituted into the scopes
of other functions (or abstractions). If this is done naively, occurrences
of free variables in the argument terms may get {\mys parasitically
bound}\index{parasitic binding} and thus change the meaning of the entire program. Even worse,
depending on the order in which reductions are
performed in a particular program, this phenomenon may 
or may not occur, i.e., the {\mys referential transparency}\index{referential transparency} which allows you
to perform reductions in any part of a program would be lost.  

Two simple examples may help to illustrate this problem.

The first one relates to the use of the function
$$
{\tt twice[f,x]\;=\;f[f[x]]} 
$$
which applies its first argument (the one substituted for {\tt f}) twice
to its second argument (the one substituted for {\tt x}). If this function
of two parameters is just applied to itself, i.e., we have both
a {\mys self-application} and a {\mys partial application}, we eventually
 get another
function of two parameters which in fact doubles the application
 of {\tt twice},
i.e., it applies its first argument four times to its second argument.

A \kir program which first computes this function and then
applies it to another function, say {\tt square[x] = (x * x)},
and to an argument value, say {\tt 2}, looks like this:
%------------------------------------------------------------------
\begin{verbatim}

let Double_twice = def 
                     Twice [ F , X ] = F [ F [ X ] ]
                   in Twice [ Twice ] ,
    Square = sub [ X ]
             in ( X * X )
in Double_twice [ Square , 2 ]

\end{verbatim}
%-------------------------------------------------------------------
Note that this program has
the functions {\tt Double\_twice} and {\tt Square} defined
as {\tt let} variables since both are
non-recursive. This ensures that the right-hand side
of the definition for {\tt Double\_twice} is evaluated, i.e., a new
 function is actually computed, before it is applied in the outermost
 goal expression\footnote{Remember that the semantics of {\tt let} constructs
prescribes the substitution of {\tt let}-defined variables
 by the {\mys values} (or {\mys normal forms}\index{normal form}) 
of the respective defining expressions, whereas the recursively
bound variables of a {\tt def}
construct are always substituted by the function bodies as they are.}. 
Moreover, when defining {\tt Double\_twice} in this form,
you don't need to think about names for its formal parameters. As
you will see in just a moment, they are implicitely already defined. 

If you now select the right-hand side of the definition for
{\tt Double\_twice}, i.e., the {\tt def}-construct for {\tt Twice},
as the actual cursor expression and reduce it step-by-step, you
 will end up, after six reductions, with a new defining expression for {\tt Double\_twice}
(don't worry for the moment what happens in between for the expressions
that come up may be rather confusing on first sight):
%-------------------------------------------------------------------
\begin{verbatim}

let Double_twice = sub [ X ]
                   in sub [ X ]
                      in \X [ \X [ \X [ \X [ X ] ] ] ] ,
    Square = sub [ X ]
             in ( X * X )
in Double_twice [ Square , 2 ]

\end{verbatim}
%---------------------------------------------------------------------
{\tt Double\_twice} is here returned as
 a so-called {\mys curried function}\index{curried function} of two parameters, i.e.,
two nested functions of one parameter each,  since its body is embedded
in two nested {\tt sub} constructs. But strangely enough, both {\tt sub}s
bind the same variable name {\tt X}. Moreover, if you have a close look at the
 body of this function,
you will notice that there are four occurrences of {\tt $\setminus$X}, i.e., of
the variable {\tt X} preceded by a {\mys back\_slash}\index{back\_slashing variables}, and one occurrence of
{\tt X} without it. Now, just what is this supposed to mean? 


The answer to this is very simple: The back\_slashes are
generated as an integral part of the {\mys  $\beta$-reduction rule}\index{$\beta$-reduction}
 as it is implemented in \pired.
Rather than introducing other variable names (as the
classical definition of $\beta$-reduction requires~\cite{bar81,hind86}), this rule
realizes a {\mys dynamic indexing scheme}\index{dynamic indexing scheme} which renders conflicting
variable occurrences distinguishable and sustains a {\mys static} (or lexical)
{\mys scoping rule}\index{static scoping}\index{lexical scoping}.
The back\_slashes preceding the occurrence of a particular variable,
 say {\tt X}, 
simply  count the number of {\mys binders}\index{binder} {\tt sub[X]} for the same
variable name that, on a straight path in the syntax tree, are between this
occurrence and the binder {\tt sub[X]} that actually binds it.

In our example, this means that
\begin{itemize}
\item the variable occurrence {\tt X} is bound to the innermost {\tt sub[X]}
for there must be no other {\tt sub[X]} in between;
\item all variable occurrences {\tt $\setminus$X} are bound to
 the outermost {\tt sub[X]}
for there must be one binder {\tt sub[X]} in between (which is the
innermost {\tt sub[X]}).
\end{itemize}
Thus we have two identically named parameters {\tt X} in this function whose
occurrences in the body can nevertheless be uniquely related
 to different {\mys binding levels}\index{binding level} (or occurrences of binders {\tt sub[X]})\footnote{You will find  a precise definition of binding levels and of 
the $\beta$-reduction rule in terms of these back\_slashes (or 
{\mys protection keys}\index{protection key}) further down in this chapter, after we have worked
our way through the second example and thus developed a better
intuitive understanding as to how back\_slashes are being manipulated.}.
	
When continuing with further step-by-step reductions of the entire
program, i.e., after having moved the cursor back underneath the
{\tt let}-constructor, \pired first
substitutes in the goal expression all formal by actual parameters,
returning the application: 
%-----------------------------------------------------------------
\begin{verbatim}

ap sub [ X ]
   in sub [ X ]
      in \X [ \X [ \X [ \X [ X ] ] ] ]
to [ sub [ X ]
     in ( X * X ) , 2 ]

\end{verbatim}
%-----------------------------------------------------------------
The next step reduces the outermost application and thereby substitutes
all occurrences of {\tt $\setminus$X} which are bound to the outermost {\tt sub[X]}
by the {\tt Square} function, returning the application of a unary
function to the remaining argument {\tt 2}:
%---------------------------------------------------------------------
\begin{verbatim}

ap sub [ X ]
   in ap sub [ X ]
         in ( X * X )
      to [ ap sub [ X ]
              in ( X * X )
           to [ ap sub [ X ]
                   in ( X * X )
                to [ ap sub [ X ]
                        in ( X * X )
                     to [ X ] ] ] ]
to [ 2 ]

\end{verbatim}
%-----------------------------------------------------------------------
Another reduction step substitutes by the argument {\tt 2}
 all occurrences of {\tt X} in the function body that are bound
to the remaining {\tt sub[X]}. This is just the {\tt X} in the argument
position of the innermost of the nested applications:
%-----------------------------------------------------------------------
\begin{verbatim} 

ap sub [ X ]
   in ( X * X )
to [ ap sub [ X ]
        in ( X * X )
     to [ ap sub [ X ]
             in ( X * X )
          to [ ap sub [ X ]
                  in ( X * X )
               to [ 2 ] ] ] ]


\end{verbatim}
%-------------------------------------------------------------------
From this point on, the applications of {\tt Square} are reduced 
from innermost to outermost, yielding {\tt 65536} as the final result.

If the system would perform {\mys naive substitutions}\index{naive substitution} instead of full
$\beta$-reductions, things would turn out quite  differently. Starting 
with the above program and again making the defining expression for
{\tt Double\_twice} the cursor expression, we would get, after six reduction steps,
another function body for {\tt Double\_twice}\footnote{Note that this 
function has been manipulated by hand to what it is. \pired could
not derive it by regular means from the original program.}:
%--------------------------------------------------------------------
\begin{verbatim}

let Double_twice = sub [ X ]
                   in sub [ X ]
                      in X [ X [ X [ X [ X ] ] ] ] ,
    Square = sub [ X ]
             in ( X * X )
in Double_twice [ Square , 2 ]

\end{verbatim}
%----------------------------------------------------------------------
As all  occurrences of {\tt X} in its body would be bound to the innermost
{\tt sub[X]}, and there would be no variable occurrence bound to the
outermost {\tt sub[X]}, the program would produce, after three more
reduction steps, the nested application
%----------------------------------------------------------------------
\begin{verbatim}

ap 2
to [ ap 2
     to [ ap 2
          to [ ap 2
               to [ 2 ] ] ] ]


\end{verbatim}
%-----------------------------------------------------------------------
which, of course, is dead wrong.

Note that an entirely different course of actions takes place if the
program is reduced strictly from outermost to innermost, i.e., with
the cursor always remaining under the outermost constructor. As you
 may check out for yourself, \pired then does not fully reduce the
partial application {\tt Twice[Twice]}. Instead, it just substitutes the 
defining expression
of {\tt Twice} in itself and applies the abstraction thus obtained
directly to the
arguments {\tt Square} and {\tt 2}, as it always continues with what
is actually the topmost application. However, irrespective of this
and all subsequent 
intermediate expressions that are being brought about, the computation 
again terminates with the correct value {\tt 65536}.

Let us now turn to a more elaborate example which shows you how the number
of back\_slashes dynamically changes in the course of performing
$\beta$-reductions that produce name clashes over several binding levels.
%-----------------------------------------------------------------------
\begin{verbatim}

ap sub [ U , V ]
   in ap sub [ V ]
         in ap sub [ V ]
               in ap sub [ V ]
                     in ap sub [ V ]
                           in ap sub [ V ]
                                 in ap V
                                    to [ U ]
                              to [ ap V
                                   to [ U ] ]
                        to [ ap V
                             to [ U ] ]
                  to [ ap V
                       to [ U ] ]
            to [ ap V
                 to [ U ] ]
      to [ ap V
           to [ U ] ]
to [ V ]

\end{verbatim}
%-----------------------------------------------------------------------
This program simply applies a nameless function of two parameters
{\tt U} and {\tt V} to a single argument {\tt V}, i.e., we have a {\mys partial
 application}\index{partial application} to a {\mys globally free variable}\index{globally free variable}. When substituting this
variable for free occurences of {\tt U} in the body of this function, it
 is going to penetrate the scopes
of several nested binders {\tt sub[V]},
including the one that is left over from the outermost binder {\tt sub[U,V]}. 

A full $\beta$-reduction of this application yields the expression
 shown on top of the opposite page.
%------------------------------------------------------------------------
\begin{figure}
\begin{verbatim}

sub [ V ]
in ap sub [ V ]
      in ap sub [ V ]
            in ap sub [ V ]
                  in ap sub [ V ]
                        in ap sub [ V ]
                              in V [ \\\\\\V ]
                           to [ V [ \\\\\V ] ]
                     to [ V [ \\\\V ] ]
               to [ V [ \\\V ] ]
         to [ V [ \\V ] ]                       <=
   to [ V [ \V ] ]

\end{verbatim}
\end{figure}
%--------------------------------------------------------------------------
Here you can clearly see that each occurrence of what originally was the
free variable {\tt V} now carries as many back\_slashes as there are
binders {\tt sub[V]} into the scopes of which it has been substituted. This is to say
that there is again no {\tt sub[V]} to which these occurrences are bound,
i.e., the variable maintains its status of being globally free (Note also
that the editor has changed all innermost applications originally 
denoted as {\tt ap V to [ U ]} into {\tt V [$\setminus$\ldots$\setminus$V]}).

Further reductions will systematically eat up from outermost to innermost
the applications inside the body of the new function, unless you change 
this order by selecting an inner application as the actual 
cursor expression. So, let's do first  what is currently 
the outermost application. It
is going to substitute the argument term {\tt V[$\setminus$V]} for  
occurrences of {\tt V} that are bound to the outermost sub[V] of
 the abstraction that is in function position of this application
(which is just the {\tt V} in the line marked by a {\tt <=}). Thus,
what we get next is this:
%-----------------------------------------------------------------
\begin{verbatim}

sub [ V ]
in ap sub [ V ]
      in ap sub [ V ]
            in ap sub [ V ]
                  in ap sub [ V ]
                        in V [ \\\\\V ]
                     to [ V [ \\\\V ] ]      <<==
               to [ V [ \\\V ] ]
         to [ V [ \\V ] ]
   to [ ap V [ \V ]
        to [ \V ] ]

\end{verbatim}
%------------------------------------------------------------------

The substitution has taken place as expected, but you may also notice that 
all occurrences of {\tt $\setminus$ \ldots $\setminus$V} have lost one of
their back\_slashes. This is due to the fact that,
 together with the application, a binder {\tt sub[V]} has disappeared too,
i.e., the number of nested scopes has been reduced by one with respect
to all occurrences of back\_slashed {\tt V}'s.

You are now invited  to move the cursor to the innermost application
 for a change and perform a reduction there. Other than substituting 
{\tt V[$\setminus\setminus\setminus\setminus$V]} (which is the
argument term in the line marked {\tt <<==}) for the innermost {\tt V}, it
just decrements by one the number of back\_slashes of the variable
occurrence {\tt $\setminus\setminus\setminus\setminus\setminus$V},
but leaves all other back\_slashed occurrences of {\tt V} as they are. This
reflects the fact that only the innermost binder {\tt sub[V]} has disappeared.

Continuing in any order will finally result in an abstraction
%---------------------------------------------------------------------
\begin{verbatim}

sub [ V ]
in ap ap ap ap ap V [ \V ]
               to [ \V ]
            to [ \V ]
         to [ \V ]
      to [ \V ]
   to [ \V ]

\end{verbatim}
%-------------------------------------------------------------------------
whose body is composed of nested applications of the bound variable {\tt V}
to occurrences of the free variable {\tt $\setminus$V}. 

If you now apply this abstraction to another free variable, say {\tt U},
you get
after one more $\beta$-reduction
%---------------------------------------------------------------------
\begin{verbatim}

ap ap ap ap ap U [ V ]
            to [ V ]
         to [ V ]
      to [ V ]
   to [ V ]
to [ V ]

\end{verbatim}
%-----------------------------------------------------------------------
in which no more binders and hence no more back\_slashed variables occur.
 
The back\_slashes formally define the {\mys binding status}\index{binding status}
 (or {\mys binding level}\index{binding level}) of a variable occurrence
as follows~\cite{brui72,berk82b}:

Let {\tt $\setminus_{\tt (n)}$ V} denote a variable occurrence preceded
by $n$ back\_slashes,  then with respect to a binder {\tt sub[V]} which is $k$
intervening binders {\tt sub[V]} away from it
(counted upwards in the syntax tree), this variable is said to be
\begin{itemize}
\item {\mys protected}\index{protected variable} if $k\;<\; n$;
\item {\mys free}\index{free variable} if $k\;=\;n$;
\item {\mys bound}\index{bound variable} if $k\;>\;n$.
\end{itemize}
 
Since the purpose of the back\_slashes obviously is to protect a
variable occurrence against parasitic bindings, we will also refer to
them from now on more appropriately as {\mys protection keys}\index{protection key}, and
to the number of protection keys preceding a variable also as
its {\mys protection index}\index{protection index}.

Of course, what has been defined here for the binding status of
a variable relative to a {\tt sub}-binder applies equivalently to
variables bound in a recursive function definition of the form
$$
{\tt F[X\_1,\;\ldots\;,X\_n]\;=\;Expr}\;\;.
$$
Following standard $\lambda$-notation, this definition can be equivalently 
written as
$$
{\tt F\;=\;sub[X\_1,\;\ldots\;,X\_n]\;in\;Expr}\;\;,
$$
which is also the syntax used for the internal graph representation.
 
The {\mys $\beta$-reduction}\index{$\beta$-reduction rule} rule for an application of the general form
$$
{\tt ap\;sub[V]\;in\;Expr\;to\;[Arg]}
$$
as it is implemented in \pired can now be defined thus:

An occurrence of a variable{\tt $\setminus_{\tt (n)}$V} in the body expression
{\tt Expr}
\begin{itemize}
\item decrements its protection index $n$ by one if it is protected;
\item is substituted by the argument {\tt Arg} if it is free;
\item is left unchanged if it is bound.
\end{itemize}
An occurrence of {\tt $\setminus_{\tt (n)}$V} in the argument term {\tt Arg}
\begin{itemize}
\item is left unchanged if it is free or bound;
\item increments its protection index $n$ by $m$ if it is
protected and substituted into
the scopes of $m$ nested binders {\tt sub[V]} inside {\tt Expr}.
\end{itemize}

There is an interesting facet to this dynamic indexing scheme that you
should be aware of,  even though it may be not too relevant from a
high-level programmer's point of view: 
all formal parameters of a defined function may be named
 identically (or mechanically transformed into identical names) 
since occurrences of these parameters in the function body
 can be uniquely related to their correct binding levels solely by means
of their protection indices. 

To get this point across, consider first the following 
 application of a simple function that is defined in terms of three
distinctly named formal parameters {\tt X}, {\tt Y} and {\tt Z}:
%------------------------------------------------------------------------
\begin{verbatim}
  
def 
  F [   ] = sub [ X , Y , Z ]
            in X [ Y [ Z [ X , Y ] ] ]
in F [ A , B , C ]

\end{verbatim}
%----------------------------------------------------------------------------
 Thus it is pretty obvious where
the argument terms {\tt A}, {\tt B} and {\tt C} of this application will
be substituted. When reducing it, we get:
%------------------------------------------------------------------------
\begin{verbatim}

A [ B [ C [ A , B ] ] ]

\end{verbatim}
%--------------------------------------------------------------------------
Since occurrences of formal parameters in a function body are merely
{\mys placeholders}\index{placeholder} for things that eventually will be substituted for them,
it is, of course,
 completely irrelevant which parameter names you actually choose. The result
of applying the function will always be the same.
So, you may change any of the parameter names as long as the changes are
made consistently and do not get into  conflicts with other names.

A renaming scheme which accomplishes this is known from the theory
of the $\lambda$-calculus as so-called {\mys $\alpha$-conversion}\index{$\alpha$-conversion}. It is
based on an abstraction
$$
{\tt Alpha\;=\;sub[\;U\;,\;V\;]\;in\;U[\;V\;]}  
$$
which, when applied to another abstraction, converts the outermost parameter
of the latter to {\tt V}. You may convince yourself of this by means
of the following example:
%---------------------------------------------------------------------
\begin{verbatim}

def 
  Alpha [   ] = sub [ U , V ]
                in U [ V ]
in Alpha [ sub [ X ]
           in X [ X ] ]

\end{verbatim}
%-------------------------------------------------------------------
Here we have a {\mys partial application}\index{partial application} of {\tt Alpha} which, in a first step,
reduces to an abstraction in one parameter {\tt V}. The application that has
come about in its body inserts, in a second step, the variable {\tt V}
for occurrences of {\tt X} in the body of {\tt sub[ X ] in X [ X ]},
as a result of which you get this abstraction with {\tt X} replaced
by {\tt V}. Note that this is a special case of  a higher-order function: {\tt Alpha} is some kind of {\mys identity function} which returns the
functions to which it is applied with just the outermost 
formal parameter name 
exchanged, but otherwise in the same syntactical form. 

Things are getting decidedly more interesting if you wish to convert
all formal parameters of an abstraction into the same name, say 
again {\tt V}. Since the $\alpha$-conversion scheme employs full
$\beta$-reductions, nothing should go wrong. So, let's try
it with the function of three parameters introduced in the
above example.

In order to have all three binders of this function converted in one pass,
you should modify the $\alpha$-conversion function so that it recursively
eats its way through them from outermost to innermost.
 The following program will do this job: 
%--------------------------------------------------------------------
\begin{verbatim}

def 
  Alpha [ N , U , V ] = if ( N gt 1 )
                        then Alpha [ ( N - 1 ) , U [ V ] ]
                        else U [ V ]
in Alpha [ 3 , def 
                 F [   ] = sub [ X , Y , Z ]
                           in X [ Y [ Z [ X , Y ] ] ]
               in F ]

\end{verbatim}
%-------------------------------------------------------------------
The $\alpha$-conversion function now has a third parameter {\tt n}
which must be set to the number of  parameters that you want to be converted
 (this parameter simply terminates recursive calls of {\tt Alpha}),
which in this particular case are three. If you reduce this program step-by-step, you may get a little confused about the intermediate expressions that
build up (that's just the way reduction works by the books), but you will end up
with an abstraction of the form:
%--------------------------------------------------------------------
\begin{verbatim}

sub [ V ]
in sub [ V ]
   in sub [ V ]
      in \\V [ \V [ V [ \\V , \V ] ] ]

\end{verbatim}
%-----------------------------------------------------------------------
which has all variables converted to {\tt V} and different binding levels
distinguished by protection keys. All occurrences of
\begin{itemize}
\item {\tt V} are bound to the innermost {\tt sub[V]};
\item {\tt $\setminus$V} are bound to the {\tt sub[V]} in the middle;
\item {\tt $\setminus\setminus$V} are bound to the outermost {\tt sub[V]}.
\end{itemize}
To convince yourself that this is the same function as before, you may simply
type as input {\tt ap \% [a,b,c} and hit the return key, which creates the 
application:
%---------------------------------------------------------------------
\begin{verbatim}

ap sub [ V ]
   in sub [ V ]
      in sub [ V ]
         in \\V [ \V [ V [ \\V , \V ] ] ]
to [ A , B , C ]


\end{verbatim}
%----------------------------------------------------------------------
Three more reduction steps will produce the same resulting expression as
before:
%--------------------------------------------------------------------
\begin{verbatim}

A [ B [ C [ A , B ] ] ]

\end{verbatim}
%---------------------------------------------------------------------
So, you are free to apply any function to any other function in order to
compute new functions without worrying about variable names and binding
 scopes. The $\beta$-reduction rule untangles correctly the messiest
situations of name clashes you can think of, i.e., it maintains all 
scopes as originally defined and thus guarantees {\mys 
referential transparency}\index{referential transparency} and the {\mys Church-Rosser property}\index{Church-Rosser property}.
It also avoids the alienation of intermediate programs beyond recognition as
it maintains the variable names of the original
program unless you explicitely change them into others by $\alpha$-conversion.
However, even more important is that functions are thus truly treated as
first class ojects: they cannot only be passed as parameters but also be 
returned in intelligible form, i.e., in \kir notation, as results of
rather complex computations. 

This is in sharp contrast to most other implementations of functional
languages which, contrary to what is claimed, separate the world of
functions from the world of objects they can be applied to~\cite{burs80,harp86,turn85a,huda88}. This is
primarily a consequence of compiling functions, in one way or another, to
 static code, which for all practical purposes precludes full
$\beta$-reductions~\cite{augu84,john87,appe87}. Thus, rather than computing new functions as results, say, of
partial applications, the respective run-time systems internally form so-called
{\mys closures}\index{closure}, i.e., packages composed of pieces of code and (references to)
argument objects,
and notify  you merely of the fact that the computation has terminated
with a function of some number of arguments, but they cannot tell you
what the function looks like. Also, since these systems are based on
 {\mys polymorphically typed languages}\index{polymorphic typing}, accepting only programs that
are well typed, they cannot deal with globally free variables either. 
 


