%=======================windows===========================================
\section*{Appendix 3: The X-Window Frontend for the Editor}
\addcontentsline{toc}{section}{Appendix 3: The X-Window Frontend for the Editor}


For some UNIX-based systems there is also available an X-window frontend which
may be called with {\tt Xred}.
It facilitates handling the editor in that it realizes many of the editor
controls, parametrization of the run-time system, file handling etc. as
menue functions, i.e., you can mainly work with a mouse rather than typing editor
commands.  There are also two important extensions that come with 
the X-windows implementation: it is now possible to load ASCII files
and to switch between syntax-oriented and line-oriented editing. Since the user
is guided by menues, there is also no need to learn the syntax of the
editor commands. 

The following system versions are required to support {\tt Xred}:
\begin{itemize}
\item UNIX system V, release 3.2 with IPC message queues enabled;
\item X window system X11R5 or higher;
\item X-view 3.2 or higher toolkit;
\item colour work station or X-terminal.
\end{itemize}

\begin{figure}[htb]
\let\picnaturalsize=N
\def\picsize{12cm}
\def\picfilename{pics/xred-main.eps}
\centerline{\ifx\picnaturalsize N\epsfxsize \picsize\fi \epsfbox{\picfilename}}
\caption{\label{window} Screen layout of the Xred main window}
\end{figure}

Figure ~\ref{window} shows the main window of the X-window frontend.
It consists of the following three parts (from top to bottom):
\begin{itemize}
\item  the command panel which includes eight menue buttons and also
displays the (file) name of the program that is actually in use;
\item the usual alphanumerical interface of \pired~, as described in
chapter 3;
\item the window footer which displays various short help and editor messages.
\end{itemize}
The X-frontend is just another interface sitting on top of
 the editor program that controls the alphanumerical interface. All features 
and functions, other than those that have been added,
 are working in the same way as before.

Under one of the menues, you may also call a text editor object which opens up
another window, in which you can use a simple line-oriented mode 
to edit (parts of) \kir~ programs. Provided they are syntactically correct,
they may be inserted in any syntactical position
of the expression that is under the control of the main window. The layout of
the text editor object is shown in fig. ~\ref{textobject}. It too features 
a command panel with eight function buttons, a main field which holds the text to be
edited, and a window footer for various messages.  

\begin{figure}[htb]
\let\picnaturalsize=N
\def\picsize{12cm}
\def\picfilename{pics/texted.eps}
\centerline{\ifx\picnaturalsize N\epsfxsize \picsize\fi \epsfbox{\picfilename}}
\caption{\label{textobject} Screen layout of the text editor object}
\end{figure}

All system functions of the main window are grouped into eight menues which may be accessed
through the menue buttons of the command panel.  Following
standard conventions, the menue entries that show up when clicking one of the
menue buttons may be followed by three dots or by arrows. 
Selecting one of the former menues opens up a user request box, selecting one of the latter
opens up another sub-menue. All menues may be pinned on the screen in order to
 use its functions repeatedly.

Here is a brief explanation of the available menues:
\begin{itemize}
\item the menue {\sc file} includes all the file handling functions. These are
\begin{description}
\item[{\tt clear}] which clears the {\sc fa}-field of the editor and the backup buffer;
\item[{\tt load ...}] which opens a file request box to load a file of one of the formats
{\tt .ed} (for input/output files) {\tt .red} (for red expression files) or
{\tt .pp} (for pretty print (ASCII) files). If you try to load a {\tt .pp} file
which contains a syntax error, the frontend will switch to a text editor object
for corrections, otherwise the frontend asks you whether you want the expression
stored away as an {\tt .ed} file.
\item[{\tt include}] which opens a file request box to insert into the actual cursor position
an expression read from a file of one of the three legitimate formats;
\item[{\tt save}] which saves the actual cursor expression under the file name last used;
\item[{\tt save as ...}] which opens a file request box to specify a file name under which
the complete actual expression is to be saved; 
\item[{\tt save expression ...}] which opens a file request box to specify  a file name
under which the actual cursor expression is to be saved;
\item[{\tt load setups ...}] which loads the setup file for the system parameters for 
inspection and changes;
\item[{\tt save setups ...}] which stores the (updated) parameter set shown on the screen
in the setup file;
\item[{\tt exit}] which quits the X-window frontend;
\end{description} 
All functions followed by ... bring up a file request box which displays the
files or directories (folders) that are available under the directory in which you
called {\tt Xred}. You may select one of the files or directories, preferably {\tt red.o} in
the latter case,
to which you wish to store or from which you wish to  fetch the actual cursor expression.
Clicking one of the directories brings up the
next lower level of files or directories, clicking the name of a file of type
 {\tt .ed} actually does the copying of the file contents 
into the actual cursor position (or vice versa).
\item the menue {\sc edit} includes some functions which support syntax-oriented
editing:
\begin{description}
\item[{\tt undo}] undoes the last modification in that it copies the
contents of the back-up buffer into the actual cursor position;
\item[{\tt copy}] copies the actual cursor expression into the backup buffer;
\item[{\tt paste}] inserts the expression held in the backup buffer into the actual
cursor position;
\item[{\tt cut}] moves the actual cursor expression into the backup buffer,
leaving an empty symbol {\tt \#} in its syntactical position;
\item[{\tt texteditor ...}] calls the text editor object 
which shows up on your screen as another window with essentially
the same layout. In its expression field you will find the cursor
expression of the main window set up for line-oriented editing, while the cursor
position in the main window is now replaced by a system-generated identifier for
the text editor object. You may now directly use the keyboard of your workstation
to modify or edit your program text. You may insert into this text
all expressions stored under dollar variables. When clicking the
{\tt done} button in the command field, the text editor window disappears 
and the edited expression replaces the identifier in the expression field
of the main window. Note that returning in this way to the main window is only possible if
the expression is syntactically correct, otherwise the text editor window stays in place
and displays an error message at the bottom. Thus it is  now
your responsibility to construct syntactically correct expressions.   
\end{description}
\item the menue {\sc buffers} includes the functions by which the actual
cursor expression is written to or read from one of the buffers (using
self-explaining inscriptions of the buttons);
\item the menue {\sc reduce} includes the functions which control the reduction
process:
\begin{description}
\item[{\tt undo last reduction}] re-installs the cursor expression as before the last sequence of reductions;
\item[{\tt base expression}] sets the cursor to the top of the entire expression;
\item[{\tt one step}] performs at most one reduction step on the cursor expression;
\item[{\tt set reduction counter ...}] opens a request box for the specification of an
initial reduction counter value (which may be done by clicking the up/down arrows to
the right of the counter field or by directly typing it into this
field);
\item[{\tt use reduction counter}] performs on the cursor expression at most as many
reduction steps as specified by the last counter setting; 
\item[{\tt reduce expression}] performs an unlimited number of reduction steps and should
therefore be used only if it is absolutely certain that the  cursor expression 
 terminates eventually;
\end{description}
\item the {\sc options} menue allows you to set up certain system parameters
such as the parameters of the run-time system, some properties of the frontend, and
the layout of the expressions under the syntax-oriented editor mode;
\item the {\sc extra} menue contains some special functions like tool boxes for
frequently used commands;
\item the {\sc help} menue includes the help files, with the various help items
classified in the same way as under the alphanumerical frontend. 
\end{itemize}



