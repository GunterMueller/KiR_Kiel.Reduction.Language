\section*{Appendix 2: Summary of the \pired Controls}
\addcontentsline{toc}{section}{Appendix 2: Summary of the \pired Controls}

For quick references, we will in this appendix give a brief 
description of all the control functions of \pired
as they can also be looked up under the {\tt help} 
feature that may be called with the function key F4. 
\\

\begin{description}

\item[Calling and Terminating \pired]$\;$\\
\pired, if properly installed on your UNIX system, should be available under your own or someone elses user directory
in a file directory {\tt red}. The executable file {\tt reduma} 
may be found under the path name {\tt red/red.o}. These are the commands to call and exit from \pired:
\begin{description}
\item[{\rm call}]: {\tt reduma [-options] [ file [, file ] ]}
\\ (the material enclosed in the brackets [,] is optional) 
\item[{\rm exit}]: {\tt CTRL-c, CTRL-z} or editor command {\tt exit}
\item[{\rm options}]:
\\ t: reduces the specified file and displays the time required; 
\\ i: instead of {\tt red.init} uses the specified file as initialization file;
\\ c: reduces the first file and compares it with the second file (if
 specified);   
\\ p: prints all specified editor files in the prettyprint format, using
the extension {\tt .pp}.
\end{description}

With any of the options enabled, \pired runs in a batch mode, 
otherwise it runs in the interactive mode.
\\

\item[The Line Editor] $\;$\\
The line editor handles what you type in the input line displayed at 
your screen. There is a  bound on the length of the input line 
which is shown in 
the right-hand corner of the message field. An input may exceed this length,
in which case you can see only part of it in the input field.
The functions that are available to manipulate 
 the contents of the input line other than
 just typing it are the following:
\begin{description}
\item[{\tt F3}]: turns on the insert mode in which existing input 
may be modified;
\item[{\tt CTRL-e}]: turns the insert mode off;
\item[{\tt DEL}]: deletes the character to the left of the current
cursor position;
\item[{\tt F4}]: deletes the character at the current cursor position;
\item[{\tt CTRL-d}]: deletes all characters from (and including) the current cursor position to the end of the input line;
\item[{\tt cursor-home}]: positions the cursor at the first input character;
\item[{\tt cursor-up}]: moves the part of the input line actually displayed
 half a line to the left;
\item[{\tt cursor-down}]: moves the part of the  input line  actually displayed 
 half a line to the right;
\item[{\tt tab}]: moves the cursor to the next tab stop.
\end{description}

\item[Function Keys] $\;$\\
\pired supports 16 logical function keys F1 .. F16 which may or
may not be assigned to (some of) the physical function keys that 
are available on your keyboard. You may define these assignments
 by yourself with the help of a {\tt defkeys} command as described
under the command interpreter. Alternatively, you may type in the
input line (with the message line cleared) {\tt :n}, where {\tt n}
 is the number of the logical function key, which will be 
echoed as {\tt Fn}.

\begin{description}
\item[{\tt F1}]: performs at most one reduction step on the current cursor expression\footnote{This is a default value which can be changed using the editor command {\tt redcnt}.};
\item[{\tt F2}]: enters the read mode for {\tt .edit} files; 
\item[{\tt F3}]: displays in the {\sc fa} field of your screen the
actual cursor expression;
\item[{\tt F4}]: calls the \pired help file;
\item[{\tt F5}]: reads an expression from a backup buffer\footnote{Note that this buffer by default always contains the expression held under the
current cursor position prior to the last shift of reductions.};
\item[{\tt F6}]: reads an expression from a first auxiliary buffer\footnote{In conjunction with special editor commands, this buffer may
be used for dedicated purposes.};
\item[{\tt F7}]: reads an expression from a second auxiliary buffer;
\item[{\tt F8}]: enters the command mode;
\item[{\tt F9}]: performs at most 1000 reduction steps on the current 
cursor expression;
\item[{\tt F10}]: enters the write mode for {\tt .edit} files; 
\item[{\tt F11}]: displays in the {\sc fa} field the father expression of the current cursor expression; 
\item[{\tt F12}]: unassigned;
%
\addtocounter{footnote}{-3}
%
\item[{\tt F13}]: writes the current cursor expression into a backup
buffer\footnotemark;
\item[{\tt F14}]: writes the current cursor expression into a first auxiliary buffer;
\item[{\tt F15}]: writes the current cursor expression into a second auxiliary buffer;
\item[{\tt F16}]: terminates \pired.
\end{description}


\item[The Command Interpreter] $\;$\\
The command mode may be entered by means of the function key F8. 
Under this mode, only one command may be executed at a time. \pired
returns to the normal edit mode immediately afterwards.

The following commands are available:
\\
\begin{description}

\item[{\rm The Editor Commands}] $\;$\\
\begin{description}
\item[{\tt abb\_def}]: abbreviates all {\tt def} constructs;
\item[{\tt abb\_idef}]: abbreviates only all inner {\tt def} constructs;
\item[{\tt append file}]: appends prettyprint to {\tt file};
\item[{\tt arity}]: displays the arity of the constructor that is in
cursor position;
\item[{\tt cat file}]: displays the contents of the file {\tt file};
\item[{\tt cmd shell\_cmd}]: issues the shell command {\tt shell\_cmd} to a shell;
\item[{\tt comp}]: compares the expression in the first auxiliary buffer with the current cursor expression;
\item[{\tt curmode[par]}] enables the line-oriented cursor mode with {\tt par = on } and
the syntax-oriented cursor mode with {\tt par = off}, the latter also being the
default option;
\item[{\tt defkeys}]: enters the mode under which logical function keys and cursor movements can be assigned to physical function keys on 
your keyboard;
\item[{\tt dfct}]: shows the function headers of abbreviated {\tt def}
constructs;
\item[{\tt dgoal}]: shows the goal expressions of abbreviated {\tt def}
constructs;
\item[{\tt dmode[ n ]}]: defines the number of nesting levels up to 
which an expression is being displayed (with {\tt n = -1} showing all
 levels);
\item[{\tt editparms}]: shows all editor parameters;
\item[{\tt exit}]: terminates the editor;
\item[{\tt ext[ i [extension]]}]: defines the extensions of the various file types that can be used under the \pired editor, with {\tt i} being assigned as follows:
\\
\\0: input/output file   {\tt .ed};
\\1: prettyprint file    {\tt .pp};
\\2: document file       {\tt .doc};
\\3: protocol file       {\tt .prt};
\\4: ASCII file (portable) {\tt .asc};
\\5: red expression file {\tt .red};
\\6: pre-processed expr  {\tt .pre};          
\\
\item[{\tt find}]: searches, from the current cursor position downwards, for the first
occurrence of a sub-expression that matches the expression stored in the topmost position
of the first auxiliary buffer, if there is one;
\item[{\tt findexpr[expr]}]: searches, from the current cursor position downwards, 
for the first occurrence of of a sub-expression that matches {\tt expr};
\item[{\tt help[text]}]: searches for help;
\item[{\tt initfile[file]}]: changes the name of the initialization 
file to {\tt file};
\item[{\tt initparms}]: reads the parameters held in the initialization file;
\item[{\tt load[file]}]: loads into the current cursor position an 
expression from {\tt file} (which must be of type {\tt .red});
\item[{\tt next}]: tries to find under the current cursor expression 
the next occurrence of the expression
held in the first auxiliary buffer;
\item[{\tt pp[file]}]: prettyprints the current cursor expression
to {\tt file.pp};
\item[{\tt print[file]}]: prints the current cursor expression to
{\tt file.asc};
\item[{\tt protocol file}]: appends to the file {\tt file.prt}
the (intermediate) expressions obtained after every shift of reductions;
\item[{\tt protocol}]: terminates protocolling;

\item[{\tt read[file]}]: reads an ASCII text expression from {\tt file.asc};
\item[{\tt redcnt n}]: sets the default reduction counter assigned to
 the function key {\tt F1} to the value {\tt n};
\item[{\tt reduce n}]: performs at most {\tt n} reduction steps on the
current cursor expression;
\item[{\tt redparms}]: shows the reduction parameters;
\item[{\tt refresh}]: refreshes the screen;
\item[{\tt replace[expr\_1; expr\_2; mode]}] searches, from the current cursor position
downwards, for occurrences of sub-expressions that match {\tt expr\_1}, and replaces
them by {\tt expr\_2}, with {\tt mode} specifying whether all replacements are to
be made in one go {\tt mode = all} under interactive control {\tt mode = ask}; 
\item[{\tt saveparms}]: saves reduction and editor parameters;
\item[{\tt shell}]: calls another UNIX shell;
\item[{\tt small on|off}]: sets the display mode;
\item[{\tt stack}]: displays the contents of the \pired run-time stacks;
\item[{\tt store[file]}]: saves an expression on {\tt file.red};
\item[{\tt time}]: shows the time used up for the last shift of
reductions;
\end{description}
For the most frequently used of these commands there exist abbreviations.
\\
\item[{\rm Setting System Parameters}]$\;$\\ 
\begin{description}
\item[{\tt pagesize no\_of\_bytes}]: sets the size of the communication page
to the \pired interpreter;
\item[{\tt heapdes no\_of\_entries}]: specifies the number of entries in
the descriptor array;
\item[{\tt heapsize no\_of\_bytes}]: specifies the size of the heap segment in numbers of bytes;
\item[{\tt qstacksize no\_of\_el}]:  specifies, in numbers of stack elements (of four bytes), the sizes of the stacks E,A,H of the run-time
environment of \pired;
\item[{\tt mstacksize no\_of\_el}]: specifies the size of stack M of
the
run-time environment;
\item[{\tt istacksize no\_of\_el}]: specifies the size of the argument
stack I of the run-time environment;
\item[{\tt fixformat}]: converts all decimal numbers into binary coded
integer or floating point formats of the host system;
\item[{\tt varformat}]: performs all arithmetic operations on decimal
number representations with unlimited precision (this is the defasult
option of \pired);
\item[{\tt base}]: defines the base for the decimal number representation under the {\tt varformat};
\item[{\tt trunc no\_of\_digits}]: specifies an upper limit on the number of digits displayed for any decimal number;
\item[{\tt mult\_prec no\_of\_digits}]: specifies, in numbers of digits
after the decimal point, the precision for decimal multiplications;
\item[{\tt div\_prec no\_of\_digits}]: specifies, in numbers of digits
after the decimal point, the precision for decimal divisions;
\item[{\tt betacount on|off}]: when on, counts only reductions of defined function applications, otherwise all $\delta$-reductions as well.

\end{description}
\end{description}

\item[The File System] $\;$\\
\pired supports several types of files to store and communicate 
\kir expressions. It makes use of the underlying UNIX file system.
All files are stored under the same directory under which the
executable file {\tt reduma} is being called.
The file types are distinguished by appropriate default extensions
which are being appended to the file names upon file creation by
type-specific commands. The following types are available:

\begin{description}
\item[{\rm input/output files (type 0)}]$\;$\\$\;$\\
This file type carries the default extension (suffix) {\tt .ed}. It
 is used to create libraries of \kir expressions in the
 internal editor format. \\$\;$\\
Commands:\\$\;$\\
Writing an {\tt .ed} file: function key F10 followed by {\tt file\_name};\\
Writes the current cursor expression into the file {\tt file\_name.ed}. If this file already exists, its actual contents are overwritten.\\$\;$\\
Reading an {\tt .ed} file: function key F2 followed by {\tt file\_name};\\
Overwrites the current cursor expression with the expression held
 in the file {\tt file\_name.ed} if it exists.\\$\;$\\
{\tt .ed} files may also be referenced from within \kir programs
by {\tt \$file\_name} (which must start with the first letter typed as
upper case), in which case the full expressions held in the files are
 literally substituted for the references before executing the programs.
\item[{\rm prettyprint files (type 1)}]$\;$\\
This file type carries the default extension {\tt .pp}. \kir
 expressions are in these files stored in the same layout in which
 they appear on the screen. In this form, they may be either directly
 printed or included in other text files.\\$\;$\\
Commands:\\$\;$\\
Writing a new or overwriting an existing {\tt .pp} file: {\tt pp[file\_name]}.
\item[{\rm document files (type 2)}]:$\;$\\
The default extension for this file type is {\tt .doc}. The command 
{\tt doc file\_name} is an abbreviation for the UNIX system command
{\tt \$EDITOR file\_name.doc}. 
\item[{\rm protocol files (type 3)}]$\;$\\
The default extension for this file type is {\tt .prt}. It may be used to store a succession of intermediate \kir expressions while executing
 a program in a stepwise mode.\\$\;$\\
Commands:\\$\;$\\
Opening a protocol file: {\tt prot file\_name}\\
Closing a protocol file: {\tt prot}\\ 
\item[{\rm ASCII files (type 4)}]$\;$\\
The default extension is {\tt .asc}. This file type uses a special 
external format for \kir expressions which is portable to 
other (text) editors.\\$\;$\\
Commands:\\$\;$\\
Writing an ASCII file: {\tt print[file\_name]}\\
(Over)writes the file {\tt file\_name.asc} with the current cursor expression;\\$\;$\\
Reading an ASCII file: {\tt read[file\_name]}\\
Overwrites the current cursor expression with the expression held
in the file {\tt file\_name.asc}, if it exists.\\
\item[{\rm red files (type 5)}]$\;$\\
The default extension of this file type is {\tt .red}. This is another
input/output file type which, in difference to files of type 0, treats
\kir expressions as macros: reducing them, e.g., in the context of
a larger program into which they are being inserted, decrements the 
reduction counter by just one, irrespective of the actual number
of reduction steps that are carried out.\\$\;$\\
Commands:\\$\;$\\
Writing a {\tt .red} file: {\tt store[file\_name]}\\
(Over)writes the file {\tt file\_name.red} with the current cursor expression;\\$\;$\\
Reading a {\tt .red} file: {\tt load[file\_name]}\\
Overwrites the current cursor expression with the expression in
{\tt file\_name.red}.

\end{description}

\end{description}
