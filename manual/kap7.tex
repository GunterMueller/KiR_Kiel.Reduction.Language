\subsection{Pattern Matching with User-Defined Constructors}
You may have askes yourself why pattern matching as introduced in the preceding section
is confined to list structures, but is not applicable to other non-atomic \kir expressions,
e.g., applications or user-defined functions (abstractions). Though one can immediately
think of many interesting applications of such a feature, there is a very simple and good reason
for not supporting it: it would render the results of computations dependent on execution
orders and thus violate the {\mys Church-Rosser property}\index{Church-Rosser property} and also {\mys referential transparency}\index{referential transparency}. As
a consequence, we would end up having a  rather versatile {\mys term rewriting system}\index{term rewriting system}, but not
anymore a purely functional reduction system. 

In order to `fake' at least syntactically the applicability
 of pattern matching
to objects other than just lists (sequences), \kir supports
 {\mys user-specified constructors}\index{user-defined constructor}. They may be used to construct 
expressions of a {\mys meta-level language}\index{meta-level language} that sits on top of \kir and can
be interpreted only by {\mys rewrite rules}\index{rewrite rule} specified as pattern matches. This
feature, among many other applications, greatly facilitates rapid
 prototyping and testing of abstract machines, compilation schemes, type
checkers etc.  for other programming language, imperative or declarative
 (including \kir itself).

A {\mys meta-level expression} generally  has the syntactical form:
$$
{\tt Constr\{Expr\_1\;,\; Expr\_2\;,\; \ldots\;,\; Expr\_n\}}
$$
where {\tt Constr} is a character string beginning with a letter
 which, in conjunction with the
curly brackets {\tt \{} and {\tt \}} that enclose the 
sequence of $n$ \kir expressions {\tt Expr\-1} , \ldots , {\tt Expr\_n},
is taken as an $n$-ary user-specified constructor. In particular, 
the \kir expressions
may be recursively constructed from other meta-level expressions.
  Note that the constructor symbol, though the editor converts its 
first letter to upper case, is not treated as a variable, and that
the sequence
within the brackets may be empty, in which case you have a $0$-ary
constructor.

A pattern that matches against such an expression 
must feature the same constructor name and the same arity\footnote{except
in those cases where the pattern contains wild card synbols}, but may
only have legitimate pattern components in the places of the expressions.
These are variables, constants, wild cards, 
type specifications, and recursively 
non-atomic patterns made up from ordinary list constructors 
or user-specified constructors. 

The following is a simple example which shows how pattern matching with
user-specified constructors works:
%----------------------------------------------------------------
\begin{verbatim}

ap case 
   when C_1{< U , V > , 'abc` , C_2{W}} guard true do C_2 [ U [ V , W ] ]
   end
to [ C_1{< + , 6 > , 'abc` , C_2{4}} ]

\end{verbatim}
%-----------------------------------------------------------------
Here the pattern obviously matches against the argument term in all components
that matter, i.e., the application reduces to the body of the {\tt when}
clause in which the variables {\tt U, V, W} are replaced by {\tt +, 6, 4},
respectively:
%--------------------------------------------------------------------
\begin{verbatim}

C_2 [ ( 6 + 4 ) ] .

\end{verbatim}
%-------------------------------------------------------------------
One more reduction step yields {\tt C\_2 [ 10]} .

You may convince yourself that changing the names of the user-specified
constructors or of their arities either in the pattern or in the argument
(but not identically in both (!)) leads to a mismatch, i.e., the application
itself is returned as its normal form.

The particular syntax for meta-level expressions and for the corresponding
 patterns, of course, is more or less syntactic sugar. The same ends may be achieved
by expressions of the form 
$$
{\tt <\;'constr`\;,\;Expr\_1\;,\;\ldots\;,\;Expr\_n\;>}
$$
and equvalent patterns, in which the user-introduced 
constructors are represented as
ordinary list constructors in combination with 
dedicated character strings (i.e., constants) in the first (or any other)
index positions of the respective sequences. In fact, \pired uses internally
this representation as it reduces exactly as desired. There is one significant
 difference to ordinary lists though: structures made up from user-defined constructors
can only be re-structured by pattern matching; primitive structuring functions are not
applicable to them.

\subsection{A Case Study : the SECD Machine}
Let us now look at an interpreter for Landin's {\mys SECD machine}\index{SECD machine}~\cite{land64} 
as a perfect example to demonstrate the expressive power of pattern matching
with both ordinary lists and user-defined constructs.
In its original form, the SECD machine was conceived as
an abstract applicative order
evaluator for expressions of a pure $\lambda$-calculus. It derives its name
from the four data structures it uses as a run-time environment. These 
data structures are operated as {\mys push-down stacks}, of which
\begin{description}
\item[{\mys stack S}] holds evaluated $\lambda$-terms in essentially the
order in which they are encountered during the traversal of the original
$\lambda$-expression;
\item[{\mys stack E}] is an environment stack which holds name/value pairs
 representing instantiations of free occurrences of $\lambda$-bound
variables;
\item[{\mys stack C}] is a control structure which 
accommodates the expression that still needs to be processed;
\item[{\mys stack D}] serves as a dump for actual machine states that need to be saved when entering into the evaluation of a new application.
\end{description}
The operation of this machine is specified in terms of a state transition
function
$$
\tau\;:\;(\;S,\;E,\;C,\;D)\;\rightarrow\;(\;S',\;E',\;C',\;D')
$$
which maps a current into a next machine state. To do so, the function $\tau$
must distinguish among six basic constellations on the stack-tops
 of components of (partially or fully evaluated) $\lambda$-terms. These
constellations can be readily expressed in terms of pattern matches
collected in a {\tt case} construct. They include the occurrence of
\begin{itemize}
\item an empty control structure C in conjunction with a non-empty dump,
in which case a new machine state must be retrieved from the dump;
\item a variable on top of the control structure, in which case the associated
value must be looked up in the environment E and pushed into the value stack
S;
\item a $\lambda$-abstraction on top of the control structure, in which case
a closure representing its value in the actual environment E must be created
on top of stack S;
\item an applicator on top of C in conjunction with a closure on top
of S, in which case a new name/value pair must be added to the environment E;
\item an applicator on C in conjunction with any other term on on top of
S, in which case the value of applying the topmost to the second-to-top 
item on stack S must be computed and pushed into S;
\item an application on top of the control structure which, in compliance with
the applicative order regime, must be reconfigured on C so as to have its
argument and function terms in this order precede an explicit applicator;
\end{itemize}

The $\lambda$-terms to be processed by the interpreter that we wish to
specify for this machine may be defined in the following meta-syntax in order
to distinguish them from the $\lambda$-terms of \kir proper:
$$
{\tt Term\;\rightarrow\;Var\{X\}\;|\;Lambda\{X\;,\;Term\}\;|\;Apply\{Term\;,\;Term\}} \;\;.
$$
Thus, we consider only the terms of a pure $\lambda$-calculus, treating
variables, abstractions and applications as special constructor expressions
which in fact define a meta-language on top of \kir. In addition we have 
{\mys closures} of the general form {\tt Clos\{ E, X, Expr\}} which define {\tt Expr} as the value to be substituted for occurrences of the variable {\tt X} in the environment {\tt E}.

The specification of the interpreter may be based on the representation of the run-time environment as a sequence
of four subsequences which correspond to the four push-down stacks.
The state transition
 function $\tau$ can then be implemented straightforwardly
by means of the {\tt case} construct shown below. It is composed of six pattern
matches which realize the above rewrite rules.

As you can see, all rules straightforwardly
map old into new states. The second rule is slightly
more complicated than the others for it must call upon a recursive
function {\tt Search} to search the environment {\tt E} for
the value with which a variable appearing on top of the control
structure is instantiated.

%-----------------------------------------------------------------------
%\begin{figure}
\begin{verbatim}

case 
when < < X , S > , E , <> , < < Ss , Se , Sc , Sd > , Rest > >
 guard true
 do < < X , Ss > , Se , Sc , Sd > ,
when < S , E , < Var{X} , C > , D >
 guard true
 do let Val = def 
                Search [ E ] =
                 let First = lselect( 1 , E )
                 in if empty( E )
                    then Var{X}
                    else if ( lselect( 1 , First ) eq X )
                         then lselect( 2 , First )
                         else Search [ lcut( 1 , E ) ]
              in Search [ E ]
    in < < Val , S > , E , C , D > ,
when < S , E , < Lam{X , Expr} , C > , D >
 guard true
 do < < Clos{E , X , Expr} , S > , E , C , D > ,
when < < Clos{Env , X , Expr} , < Hed , S > > , E , < Ap{ } , C > , D >
 guard true
 do < <> , ( < < X , Hed > > ++ Env ) , < Expr , <> > ,
           < < S , E , C , D > , D > > ,
when < < Hed , < Heta , Rest > > , E , < Ap{ } , C > , D >
 guard true
 do < < Eval [ Hed , Heta ] , Rest > , E , C , D > ,
when < S , E , < Apply{E_1 , E_2} , C > , D >
 guard true
 do < S , E , < E_2 , < E_1 , < Ap{ } , C > > > , D >
otherwise < Env , 'irreg termination` >
end

\end{verbatim}
%\end{figure}
%---------------------------------------------------------------------
Since the function $\tau$ must be recursively applied to the
run-time environment
(or machine state) until both the control structure and the dump
 are empty, we simply define a recursive function {\tt Secd} whose
body includes another {\tt case} construct composed of
\begin{itemize}
\item a {\tt when} clause which handles the termination condition;
\item an {\tt otherwise} clause under which {\tt Secd} is recursively
applied to the new state (or environment) resulting from the
 application to the current state, represented by the variable {\tt Env}, of the {\tt case } construct that realizes $\tau$.
\end{itemize}
 
We thus get the following complete program specification for the
interpreter function  {\tt Secd}, in the body of 
which the abbreviated inner {\tt case} construct is the one
shown on the preceding page which  realizes the state transition rules.
%--------------------------------------------------------------------
\begin{verbatim}
 
def 
  Secd [ Env ] =
   ap case 
      when < S , E , <> , <> >
       guard true
       do < S , E , <> , <> , Done >
      otherwise Secd [ ap case 
                          ? 
                          ?
                       to [ Env ] ]
      end
   to [ Env ]
in Secd

\end{verbatim}
%-----------------------------------------------------------------------

Let us now try to apply the interpreter function {\tt Secd} to a
concrete $\lambda$-expression which in \kir would look like this:
%-----------------------------------------------------------------
\begin{verbatim}

ap ap sub [ U ]
      in sub [ V ]
         in U [ V ] 
to [ sub [ X ] in X , W ]

\end{verbatim}
%----------------------------------------------------------------
As can be easily verified,
 this application reduces in three steps to the free variable {\tt W}.

In the special constructor syntax that can be understood by the interpreter,
this application must be specified as:
%-----------------------------------------------------------------------
\begin{verbatim}

Apply{Apply{Lam{'u`,Lam{'v`,Apply{Var{'u`},Var{'v`}}}} ,
            Lam{'x`,Var{'x`}}},Var{'w`}}                         .

\end{verbatim}
%-----------------------------------------------------------------------
It is important to note here that what are supposed to be concrete 
variable names in this syntax must be specified as constants, i.e.,
as character strings embedded in quotation marks. If you would not
 do this, they
would become variables of \kir but not of the meta-level $\lambda$-terms that are to be interpreted and as such could get
parasitically bound outside the intended context. Moreover, the interpreter 
must compare, as
part of its second pattern match in the inner {\tt case} construct, 
variable occurrences in the $\lambda$-terms of your special
 meta-language against matching entries in the environment structure
E. This can be done only on the basis of {\mys instantiations} with
constant values of the pattern variables in terms of which this
comparison is specified.

In order to interpret the above meta-language expression, it must be 
 put into the position of the third subsequence of the argument sequence
for the function {\tt Secd} (which represents the control structure C), with all other subsequences initially empty:
%------------------------------------------------------------------------
\begin{verbatim}

< <> , <> ,
< Apply{Apply{Lam{'u`,Lam{'v`,Apply{Var{'u`},Var{'v`}}}} ,
              Lam{'x`,Var{'x`}}},Var{'w`}}, <> > ,    <> >

\end{verbatim}
%---------------------------------------------------------------------
Thus, the complete program, ready for execution, is this (with the
body of {\tt Secd} again abbreviated):
%--------------------------------------------------------------------
\begin{verbatim}

ap def 
     Secd [ Env ] =
      ap case 
            ?
            ?
      to [ Env ]
   in Secd
to [ < <> , <> ,
     < Apply{Apply{Lam{'u`,Lam{'v`,Apply{Var{'u`},Var{'v`}}}} ,
                   Lam{'x`,Var{'x`}}},Var{'w`}}, <> > ,    <> > ]

\end{verbatim}
%--------------------------------------------------------------------
Provided you didn't make any mistakes when editing this program,
it returns after 96 reduction steps the correct solution in the form of
the following environment structure:
%-------------------------------------------------------------------
\begin{verbatim}

< < Var{'w`} , <> > , <> , <> , <> , Done >

\end{verbatim}
%--------------------------------------------------------------------
i.e., with the variable construct {\tt Var\{'w`\}} as the sole entry
 of the value stack S, and with all other stacks empty.

You may alternatively try to work your way through a step-by-step execution
of this program, but be warned that it may take quite a while for you to
figure out what is going on in a particular situation and where
the next transformation step is going to take place. Since intermediate
programs cannot be completely visualized on the screen without many
 question marks, it is helpful to study the first few steps very carefully
in order to be able to select as the actual cursor expressions the {\tt case}
applications which are to be reduced next and need to be inspected in detail. 

As an interesting exercise, you could try to supplement this SECD
machine interpreter, within the framework of an applied $\lambda$-
calculus, by the data types {\tt boolean}, {\tt integer number}
and {\tt list}, including a complete set of primitive functions
defined on these types (e.g., the ones you have available in \kir).
This will enable you to  write in this extended meta-language a few
more interesting application programs
and have them executed on your interpreter.
