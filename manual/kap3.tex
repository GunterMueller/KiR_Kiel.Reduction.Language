\section{Editing and Executing Simple Programs on  \pired}

In this section you will learn the first steps necessary to edit and execute
a simple functional program on \pired. A good example to start with is the
program given in the preceding section which computes the
{\mys greatest common divisor}\index{greatest common divisor program} of two integer numbers. 

If properly installed on your UNIX system\footnote{See also appendix 3.}, \pired may be {\mys called}\index{calling \pired} under
the appropriate directory by typing the command
$$
{\tt <unix\_prompter>: reduma}.
$$
The system will respond by producing on your screen (or window) the {\mys user
interface}\index{user interface} of \pired which has the basic format shown in fig.~\ref{f3.1}. It
%--------------------------------------------------------------
\begin{figure}[htb]
\begin{center}
\begin{minipage}{150\unitlength}
\begin{picture}(300,190)(50,0)
%------------------------------------------------------------
\thicklines
%-----------------------------------------------------------
\multiput(0,0)(0,180){2}{\line(1,0){240}}
\multiput(0,0)(240,0){2}{\line(0,1){180}}
\put(0,160){\line(1,0){240}}
\put(10,162){\makebox(30,6)[l]{\tt input :}}
\put(10,172){\makebox(30,6)[l]{\tt message :}}
\put(10,142){\makebox(6,6)[l]{\tt \# }}
\put(90,80){\makebox(40,10){\sc focus\_of\_attention field}}
%------------------------------------------------------------
\end{picture}
\end{minipage}
\end{center}
\caption{Layout of the \pired User Interface}
\label{f3.1}
\end{figure}
%-------------------------------------------------------------
consists of a small field at the top, separated by a horizontal line
from a large field which stretches over
the remainder of the screen at the bottom. The field at the top is divided 
into a line each for {\mys messages}\index{message line}, in which you will initially see
$$
{\tt RED\; 2.3\; Function-Version}\;\;\;\;      ,
$$
 and for {\mys input}\index{input line}. The bottom field will be
initialized with the symbol {\tt \#}, and you will also find the cursor in
the position of the {\tt \#}. Roughly speaking, the contents of this field represent the
\kir expression that is to be actually processed  by \pired.
It is therefore
referred to as the {\mys focus\_of\_attention field} (or {\mys
{\sc fa} field} for short)\index{focus\_of\_attention field}\index{{\sc fa} field}. The {\tt \#}
symbol generally stands for an {\mys empty expression}\index{empty expression} (or an empty {\mys syntactical
position}\index{syntactical position}), and {\tt \#} as
the only item in this field simply indicates that no expression is
as yet loaded.

\pired is now set up and ready for editing, which is a strictly {\mys syntax-directed
process}\index{syntax-directed editing}. This is to say that you always work with complete {\mys syntactical
constructs}\index{syntactical construct}\index{construct} of \kir in which either empty
{\mys syntactical positions} need to be replaced 
with other constructs, or the constructs must themselves be placed into
specific syntactical positions of enclosing constructs. The former corresponds
 to designing programs {\mys top-down}\index{top-down programming}, and the latter corresponds to designing them
{\mys bottom-up}\index{bottom-up programming}. Both directions may be used in any order of convenience. In 
supporting this programming style, the editor makes use of the fact that 
specific {\mys keywords}\index{keyword} of \kir, so-called {\mys constructors}\index{constructor}, form the root nodes
of certain {\mys syntactical constructs}\index{syntactical construct}, and that some of these constructs may
 even require expressions
of a particular type (e.g., variables, patterns etc.)
in some of their components.

To see how this works, you may start typing
on your {\mys keyboard}\index{keyboard} the fragment of a simple \kir expression, say, just the
keyword {\tt if}. The {\tt if} is immediately echoed in the input line, and
the message line fills with {\tt <expr|def>}, indicating that the input
line may legally contain a \kir expression or a (function) definition.
If you now hit the return (or enter) key on your keyboard, both the message and the input
line will be cleared, and in the {\sc fa} field appears a syntactically complete
{\tt if\_then\_else} construct with as yet uninstantiated (or empty)
component expressions,
i.e., with {\tt \#} symbols in the respective syntactical positions. The cursor
has now been moved to the topmost {\tt \#} position, expecting it to be
replaced next by a legal \kir expression, which in this particular case should
be a predicate expression that evaluates to a Boolean constant.
%-----------------------------------------------------------------
\begin{verbatim}

if #
then #
else #

\end{verbatim}
%-----------------------------------------------------------------
You can now use this structure to construct the function body of either of
the two recursive functions that make up the greatest common divisor
program; so let's do the function {\tt Modulo} first. Under the current cursor
position you may therefore type
$$
{\tt ( x \; le \; y }
$$
which again echoes
on the input line, with just {\tt <expr>} on the message line. Note that the
input has an opening but no closing parenthesis, and that both
variables are printed as lower case letters. Upon hitting the return key,
you get inserted into the cursor position a fully parenthesized expression
with the variables now printed as upper case letters.
In doing this, the editor takes the opening parenthesis
as the constructor of an expression which, by convention, has three
components and thus inserts the closing parenthesis immediately following
the third component. Furthermore, since it has neither {\tt x} nor {\tt y}
on the list of keywords of \kir, it takes them as variables and turns them
into upper case letters. The string {\tt le}, however, represents 
a primitive function symbol, i.e., it is a keyword, and therefore left as it is. 
%---------------------------------------------------------------------
\begin{verbatim}

if ( X le Y )
then #
else #

\end{verbatim}
%-----------------------------------------------------------------------
The cursor has now moved to the next empty position, and you may simply
input {\tt x} to get 
%---------------------------------------------------------------------
\begin{verbatim}

if ( X le Y )
then X
else #

\end{verbatim}
%-----------------------------------------------------------------------
Finally, you may fill out the {\tt else} part of the clause,  which can be done
in one step by typing
$$
 {\tt modulo[(x \; - \; y, y}.
$$
 Note that you may forget 
about both the closing parenthesis following the arithmetic expression in the first
argument position of {\tt modulo} and also the closing
{\tt ]} following the second (and last) argument position. Upon hitting the 
return key you will see that both are inserted into the correct positions
by the editor. Again, all character strings that cannot be identified
as keywords of \kir are turned into variables, i.e., their leading letters are
converted to upper case.
%----------------------------------------------------------------------
\begin{verbatim}

if ( X le Y )
then X
else Modulo [ ( X - Y ), Y ]

\end{verbatim}
%-----------------------------------------------------------------------
Looking at this expression,
you should realize that something strange has happened. At this point in
the program 
design, {\tt modulo} is just a name for a function that is not yet defined, so
its arity is not yet known. Also, a function in \kir  may be
applied to any number of arguments irrespective of its own arity. Thus
there is no
way for the editor to figure out how many arguments need actually be supplied.
The closing parenthesis {\tt ]} is therefore inserted on a purely speculative
basis, assuming that the current input line contains a complete argument
list. However, as you will see in just a moment, there are means to add
further arguments to this list or to delete some of them, if so desired.

So far, you have just followed a top-down design path by filling
with concrete expressions  empty
positions of a syntactical construct. Since no
further {\tt \#} symbols are left in the {\sc fa} field expression, you
will have to go a step bottom-up from here to complete the program.

Before doing this, you need to know something about the significance
of the {\mys cursor position}\index{cursor position} and the way the cursor may be {\mys moved}\index{cursor movement} around.
The cursor is always positioned either on a \kir constructor, e.g., a
keyword or a parenthesis {\tt "("} which opens an application in infix notation,
 or on an {\mys atom}\index{atom}, i.e., a {\mys variable}\index{variable}, a {\mys constant}\index{constant}, a {\mys primitive function}\index{primitive function} symbol,
or an empty expression {\tt \#}.
Whatever expression (or fragment thereof) you type on the keyboard first 
shows up in the input line and, upon pressing the return key,
 ends up in the cursor position, i.e.,
it overwrites there either an atom or an entire subexpression held together
by a constructor symbol. So, what really matters in the editing process
is not necessarily the full expression that you see in the 
{\sc fa} field but it is always what we will from now on refer to as the
{\mys cursor expression}\index{cursor expression}. You may check this out at the risk of destroying
your current cursor expression (unless you have already done so by accident)
or, for the moment, just believe it.

In the {\tt if\_then\_else} clause you have developed so far, the cursor
happens to be positioned on the variable {\tt Modulo} for it is the
 leading symbol
of the expression you have inserted last. In order to make the entire clause the subject of further editing operations, the cursor needs to be moved up
to the {\tt if}.

Since the \pired user interface emulates an alphanumerical terminal, the
{\mys cursor movement}\index{cursor movement} may only be controlled by the respective keys on your 
keyboard, of which you usually have available one each pointing {\mys up}, {\mys down},
{\mys left} and {\mys right}\footnote{A mouse, if available, doesn't do anything for you other than selecting
the window you want to work with on a workstation.}\index{up key}\index{down key}\index{left key}\index{right key}.

 As a default option you have at hand an expression-oriented cursor movement.
 With each of the cursor keys you may traverse in  a particular way
the {\mys syntax tree}\index{syntax tree} of a \kir expression.
Here is how it works:

Starting from the current cursor position, the
\begin{itemize}
\item {\tt up} key moves the cursor to the hierarchically higher constructor
(or from a {\mys son node}\index{son node} to its immediate {\mys father node}\index{father node});
\item {\tt down} key moves the cursor from left to right and round robin
through all nodes that are tied to the same immediate father node; 
\item {\tt right} key lets the cursor traverse the syntax tree in
 {\mys pre-order}\index{pre-order traversal} (i.e., recursively top-down and left-right);
\item {\tt left} key lets the cursor traverse the syntax tree in {\mys post-order}\index{post-order traversal}
(i.e., recursively right-left and bottom-up)\footnote{If your keyboard includes a {\mys home key}\index{home key}, it moves the cursor from anywhere within an expression
directly to the topmost constructor.}.
\end{itemize}

Now you can play with the cursor keys a little more systematically and
thereby find out which components of an expression are constructors,
which are atoms, and which are just blind syntactical sugar. For instance,
whichever way you are moving about the current {\sc fa} expression, you will never
get the cursor on the {\tt else} or on any of the parentheses {\tt ),[,]}. They
just let the expression look a little more structured and readable,
 but they are
of no relevance syntactically.  You may also note that, contrary to what
we said earlier about variables being atoms, the identifier {\tt Modulo}
seems to be taken as  a constructor.
This is due to the fact that in this particular high-level representation of an 
application there is no explicit constructor (which would be an applicator); so the function symbol is taken for it instead.

Alternatively, the expressions may be traversed in a line-oriented mode (as every conventional
editor does). This mode may  be enabled with a special control command which will be introduced 
later on.
 
After this
little excursion into cursor movement, you may now
 continue with the construction of the greatest common divisor
program. 

The current {\sc fa} expression must become the right-hand
side of a function definition for {\tt Modulo[X,Y]}, which must
in turn be embedded in a {\tt def} construct.
All you need to do to accomplish this is to maneouver the cursor to
the topmost constructor (which is the {\tt if}) and
 to type as input something like
$$
{\tt Modulo [\;] = \%}
$$
 if, for the time being, you don't want to think about
the formal parameters of the function. The important part of this input is the
percentage sign {\tt \%}. It acts as a {\mys placeholder}\index{placeholder} for which the current
 cursor expression will be substituted  upon entering 
 the input
 as the new cursor expression. 
However, what you actually get when hitting the return key may not be exactly
what you would expect: the editor returns as the new cursor expression a
{\tt def} construct comprising a function definition for {\tt Modulo[ ]}
(i.e., a function which as of now is without parameters) and, surprisingly enough, as a goal
expression another copy of the body of this function. 
%------------------------------------------------------------
\begin{verbatim}

def 
  Modulo [   ] = if ( X le Y )
                 then X
                 else Modulo [ ( X - Y ), Y ]
in if ( X le Y )
   then X
   else Modulo [ ( X - Y ), Y ]

\end{verbatim}
%-------------------------------------------------------------
Being faced with a goal expression that you didn't ask for has to
 do with implementation details of the editor which you don't need
 to worry about. To proceed with your editing, it really doesn't matter
whether or not you have in this place an expression other than the empty
symbol {\tt \#}. In either case, you'll have to make this syntactical
position the cursor expression and type as input (a fragment of) the expression
you want to be there, which in our example would be
$$
 {\tt gcd\; [\; 6, 4  }\;\;\;\;\;                         . 
$$
If you then move the cursor into the position between the parentheses
{\tt [ ]} in the function header, type
$$
 {\tt x,\; y}
$$
 in the input line
(which brings up {\tt <var\_list> } in the message line), and hit the return
key, you get as the new cursor expression:
%------------------------------------------------------------------
\begin{verbatim}

def 
  Modulo [ X, Y ] = if ( X le Y )
                    then X
                    else Modulo [ ( X - Y ), Y ]
in Gcd [ 6, 4 ]

\end{verbatim}
%----------------------------------------------------------------

Note that the list of formal parameters may be {\mys expanded}\index{adding function parameters} by moving the 
cursor to the parameter position prior to or
after which you wish to insert the new parameter, and by entering
 the input line
$$
 {\tt z,\%}\;\;\;\;\; {\rm or}\;\;\;\;\;\; {\tt \%, z}\;\;\;\;                         ,
$$
 respectively
 (where {\tt z} is taken as the identifier
that is to be added). Conversely, a parameter in cursor position may be {\mys removed}\index{removing function parameters}
from the list 
by hitting the {\tt delete} key on your keyboard. You are free to test both
operations on
your current {\sc fa} expression without wrecking it. In the same way you
 may also {\mys expand} or {\mys shrink}\index{adding arguments}\index{removing arguments} the list of
arguments of a function application for it features the same syntactical
structure as a function header. 


Let's now return to the business of completing the greatest common divisor 
program.  What remains to be done is
 to include the second function definition, the one
for {\tt Gcd [X,Y]}, into the {\tt def} construct. 
To do so, you first need to move the cursor to the identifier
{\tt Modulo} which is the leading symbol in the function header of the existing definition.
Whenever then you start typing anything in the input
line, the message line responds with {\tt <var| fdec>}, meaning that you
may replace either just the identifier {\tt Modulo} or the
entire function definition with another expression of the same kind,
neither of which you actually want to do. The third possibility is to
include in the input a {\mys reference}\index{referencing the cursor expression} {\tt \%} to this cursor expression
 and thus embedd it bottom-up in the desired context. 

When using as input
$$
 {\tt \%, gcd [\;]\; =\;\;\;  }\;\;\;\;\;,
$$
the editor will add the fragment
of the definition for {\tt Gcd} after that for {\tt Modulo}\footnote{As of
now, there is now way of doing it the other way around, i.e., to put the definition
of {\tt Gcd} before that of {\tt Modulo}.}.  The resulting
cursor expression looks like this:
%-----------------------------------------------------------------
\begin{verbatim}

def 
  Modulo [ X, Y ] = if ( X le Y )
                    then X
                    else Modulo [ ( X - Y ), Y ],
  Gcd [   ] = #
in Gcd [ 6, 4 ]

\end{verbatim}
%-----------------------------------------------------------------
Now you may go ahead to fill out the missing argument list and the
missing function body of {\tt Gcd} in just about the same way as 
for the function {\tt Modulo}. If everything is done, you should
get the cursor expression
%--------------------------------------------------------------------
\begin{verbatim}

def 
  Modulo [ X, Y ] = if ( X le Y )
                    then X
                    else Modulo [ ( X - Y ), Y ],
  Gcd [ X, Y ] = if ( X eq Y )
                 then X
                 else if ( X gt Y )
                      then Gcd [ Y, Modulo [ X, Y ] ]
                      else Gcd [ X, Modulo [ Y, X ] ]
in Gcd [ 6, 4 ]

\end{verbatim}
%------------------------------------------------------------------
which is ready for execution.

Of course, you may edit this program in any other way you wish. For instance,
you could start with the input {\tt def} which produces in the {\sc fa} field
a {\tt def} construct with an empty position each for a function definition
and a goal expression. From there you can design the program strictly
top-down by {\mys recursive refinement}\index{recursive refinement}. Once you are more familiar with \kir,
you will also know that there is a primitive function {\tt mod} available
which renders the defined function {\tt Modulo} superfluous.

Before executing this program (which irrevocably destroys the original program text),
 you better store it away on a permanent file since you may wish to use it several times.
In order to do so, it is important to get acquainted with
a few {\mys control functions}\index{control function} of \pired that are available
as {\mys logical function keys}\index{function key}\index{logical function key}
ranging from F1 to F16. Some or all of these logical keys may be 
assigned, as part of the \pired initialization procedure,
 to {\mys physical function keys}\index{physical function key} that are available on
your particular workstation or terminal. If your keyboard doesn't
 have that many keys
or some of them already have other system functions assigned to them,
you may either re-define the use of the available keys according to your
preference (you will be told later on in section 4.5 
how this is to be done) or instead type
{\tt :n} (where n is the number of the logical function key).

Most of the logical function keys, in one way or another, relate to the particularities
 of the reduction semantics supported by \pired. As pointed out before,
executing a program is realized as a sequence of meaning-preserving
 program transformations,
of which each produces (the internal representation of) 
a legitimate \kir program. In order to make intermediate programs
of interest visible to the user for inspection,
\pired provides the means to control, by means of a so-called {\mys reduction counter}\index{reduction counter},
the number of transformation (or reduction) steps after which it ought to stop,
 and to display the program expression reached at that point. If the program
is to run to completion in one shot, then this number must simply be chosen
 large enough to allow for a reduction sequence of the expected length. 

Moreover, there are several means to {\mys store programs} or program parts 
temporarily
on various {\mys back-up buffers}\index{buffer}\index{back-up buffer}, or permanently on an {\mys editor file system}\index{editor file system}\index{file system}
where they survive a \pired session. Temporary storage
enables you to return quickly to initial programs or to previous states of
execution. This feature is particularly helpful during the design and validation phase of a program.

The operations that control the number of {\mys reduction steps}\index{controlling the number of reductions}
are assigned to the function keys\footnote{Whenever we refer, 
fron now on, to function keys, we mean the logical keys of \pired.}
 as follows:
\begin{description}
\item[{\tt F1}] performs at most one reduction step on the current cursor expression
and returns the resulting expression in its place;
\footnote{The number of reductions performed under F1 can be re-defined by changing the
respective entry in the initialization file for your \pired version (see also
appendix 2)};
\item[{\tt F9}] performs in one go at most 1000 reduction steps on the cursor
expression and, again, returns the resulting expression in its place;
\item[{\tt F8}] gets you in a general {\mys command mode}\index{command mode} in which you are asked on the
message line to enter a particular command on the input line. In this mode
you may type {\tt r n} (note that there must be a blank between
the {\tt r} (which stands for reductions) and the {\tt n}) 
and then hit the return key
to have at most some n reductions performed on the cursor expression\footnote{The upper limit for n is basically given by the integer format 
 of the machine on which you have \pired installed.}. 
The phrase `..at most..' refers to the fact that the reduction of the
cursor expression may terminate with a
constant expression
before exhausting the number of steps
that you allowed for. The number of reduction steps that has actually been
performed always shows up in the message line upon returning the resulting
expression.
\end{description}

The following function keys are available for {\mys storing} and {\mys retrieving}\index{storing cursor expressions}\index{retrieving cursor expressions}
cursor expressions:
\begin{description}
\item[{\tt F10}] gets you into an {\mys output mode}\index{output mode}
in which you are asked on the message line to enter the name of an
{\mys editor file}\index{editor file}  into which the cursor expression is to be written.
You simply type the desired name (legitimate file names depend on the
 underlying operating systems; usually all character strings starting with a
 letter will do) and hit the return key. If the file name is new, the
system will respond in the message line with
$$
 {\tt wrote\; 1\; expression\; to\; filename.ed}\;\;\;\;\;,
$$
 if a file with this name already exists, it will ask you
to verify that you want it to be overwritten. 
\item[{\tt F2}] gets you into an {\mys input mode}\index{input mode} in which you are asked on the message
line to enter the name of an editor file. If the file exists, the system
 responds on the message line with
$$
 {\tt\; 1\; expression\; read\; from \;filename.ed}
$$
and replaces with this expression the actual cursor expression. Otherwise it
will tell you that a file with the given name can not be opened.
\item[{\tt F14} {\rm or} {\tt F15}] writes the cursor expression in a first or a second
{\mys auxiliary buffer}\index{auxiliary buffer}, respectively;
\item[{\tt F6} {\rm or} {\tt F7}] replaces the cursor expression with an expression
 read from a first or second auxiliary buffer, respectively;
\item[{\tt F13}] writes the cursor expression in another {\mys back-up buffer}\index{back-up buffer},
which is implicitely done whenever you start a reduction sequence;
\item[{\tt F5}] exchanges the cursor expression with the expression held
in the back-up buffer\footnote{Note that each of these buffers can hold at
most one expression: whenever you write in succession two or more
 expressions in a buffer, only the last expression can be retrieved, all others are lost.}
\end{description}

Note that if you accidently get into the wrong mode our your input cannot
be accepted under the current mode for reasons of which  
you will be appropriately informed on the message line, you may get out
of this situation
 by first clearing the input line with the {\tt delete} key and then hitting
the return key, which will also clear the message line.

After you've stored your {\tt Gcd} program away on a permanent file, say {\tt gcd},
 you are all set to execute the program.

 If you are confident about the program's correctness, 
you may press the function key
F9, thus allowing for at most 1000 reduction steps, and you will immediately
get as a new cursor expression the correct value {\tt 2}.  
On the message line you will also
find that it took 27 reduction steps to get there. 

You may now reload the program from the {\tt gcd} file, 
and run it again, e.g.,  now in a stepwise mode.
To do this, you just 
push the function key F1 several times and 
look at what comes up in the {\sc fa} field.

The expression returned after the first step looks like this: 
%-------------------------------------------------------------------
\begin{verbatim}
if ( 6 eq 4 )
then 6
else if ( 6 gt 4 )
     then ap def 
               Modulo [ X , Y ] = if ( X le Y )
                                  then X
                                  else Modulo [ ( X - Y ) , Y ] ,
               Gcd [ X , Y ] = if ( X eq Y )
                               then X
                               else if ( X gt Y )
                                    then Gcd [ Y , Modulo [ X , Y ] ]
                                    else Gcd [ X , Modulo [ Y , X ] ]
             in Gcd
          to [ 4 , ap def 
                        Modulo [ X , Y ] = if ( X le Y )
                                           then X
                                           else Modulo [ ( X - Y ) , Y ]
                      in Modulo
                   to [ 6 , 4 ] ]
     else ap def 
               Modulo [ X , Y ] = if ( X le Y )
                                  then X
                                  else Modulo [ ( X - Y ) , Y ] ,
               Gcd [ X , Y ] = if ( X eq Y )
                               then X
                               else if ( X gt Y )
                                    then Gcd [ Y , Modulo [ X , Y ] ]
                                    else Gcd [ X , Modulo [ Y , X ] ]
             in Gcd
          to [ 6 , ap def 
                        Modulo [ X , Y ] = if ( X le Y )
                                           then X
                                           else Modulo [ ( X - Y ) , Y ]
                      in Modulo
                   to [ 4 , 6 ] ]

\end{verbatim}
%------------------------------------------------------------------

What has happened here is the following:
The application {\tt Gcd[6,4]} in the goal expression of the original program
has been expanded by 
the function body of {\tt Gcd}, i.e.,by the right-hand side of its
definition, which consists of two nested 
{\tt if\_then\_else} clauses. 
Moreover, in this body all occurrences of the formal parameters {\tt X} and
{\tt Y} have been substituted by the actual parameters (or argument values)
{\tt 6} and {\tt 4}, respectively, and -- this is the most interesting part --
 all applications of
\begin{itemize} 
\item the function identifier {\tt Gcd} are embedded in applications of 
copies of the entire original {\tt def} construct,
\item the function identifier {\tt Modulo} are embedded in  applications
of copies of a {\tt def} construct from which
 the definition of {\tt Gcd} is deleted. 
\end{itemize}
Both applications are returned by \pired in the syntactical form
$$
{\tt ap \; def \; ... \; in\; Gcd\; |\; Modulo  \; to [ ..., ...] }
$$
where {\tt ap} is the explicit {\mys applicator} introduced in chapter 2.
The original {\tt def} construct has disappeared.

If you take a close look, you will also note that
the {\tt def} constructs thus distributed over the original 
goal expression have inserted in their own goal term positions
just the identifiers of the functions that need to
be applied to the arguments in the enclosing applications. 
Identifiers in this position merely act as selectors for a
function from the set of definitions. Moreover, this set 
is stripped down to
 just the functions that are actually referenced from within
the goal expressions: if you have {\tt Gcd} there, both {\tt Gcd} and {\tt
Modulo} are required since the latter is referenced from within the former;
if you have {\tt Modulo} there, the definition for {\tt Gcd} is dropped
since {\tt Modulo} recursively just calls itself. 

The next  four reduction steps on the cursor expression are quite simple: they
evaluate the two nested {\tt if\_then\_else} clauses and return the application shown on top of the next page.
%--------------------------------------------------------------------
\begin{figure}
\begin{verbatim}

ap def 
     Modulo [ X , Y ] = if ( X le Y )
                        then X
                        else Modulo [ ( X - Y ) , Y ] ,
     Gcd [ X , Y ] = if ( X eq Y )
                     then X
                     else if ( X gt Y )
                          then Gcd [ Y , Modulo [ X , Y ] ]
                          else Gcd [ X , Modulo [ Y , X ] ]
   in Gcd
to [ 4 , ap def 
              Modulo [ X , Y ] = if ( X le Y )
                                 then X
                                 else Modulo [ ( X - Y ) , Y ]
            in Modulo
         to [ 6 , 4 ] ]

\end{verbatim}
\end{figure}
%--------------------------------------------------------------------
Here you have in fact an application of the function {\tt Gcd} to a first
argument value {\tt 4} and to a second argument value which is to be computed
from an application of {\tt Modulo} to the arguments {\tt 6} and {\tt 4}.
To understand what happens next, you need to know that \pired realizes
a so-called {\mys applicative order} (or {\mys call\_by\_value})\index{applicative 
order reduction regime}\index{call\_by\_value}
reduction regime: before a function is applied to its arguments, the
arguments themselves are being evaluated first. So, in our particular
situation, the reduction process will continue with the application of
{\tt Modulo} in the argument position of {\tt Gcd}.

The expression that returns after one more reduction step therefore has this
application replaced by the instantiated body of {\tt Modulo} which, in
another three steps, reduces to an application of {\tt Modulo} to
the arguments {\tt 2} and {\tt 4}. The value of this is obviously {\tt 2},
which you reach after three more steps. 

Now you have completed a full cycle through the reduction
of an application of the function {\tt Gcd}.
As a result you get, in slightly different notation, the original
program which, however, has its goal expression 
replaced by {\tt Gcd[4,2]}. The difference relates to the representation of this application: the entire {\tt def} construct with {\tt Gcd} as 
a function
selector in its goal expression is now applied, by means of an explicit
applicator, to the arguments {\tt 4} and {\tt 2}.  
%---------------------------------------------------------------------
\begin{verbatim}

ap def 
     Modulo [ X , Y ] = if ( X le Y )
                        then X
                        else Modulo [ ( X - Y ) , Y ] ,
     Gcd [ X , Y ] = if ( X eq Y )
                     then X
                     else if ( X gt Y )
                          then Gcd [ Y , Modulo [ X , Y ] ]
                          else Gcd [ X , Modulo [ Y , X ] ]
   in Gcd
to [ 4 , 2 ]

\end{verbatim}
%-----------------------------------------------------------------------
Knowing that it takes five reduction steps to get from an application of
{\tt Gcd} to arguments, of which the first is greater than the second, to
the point where its nested {\tt if\_then\_else} clauses are evaluated,
you may simply skip all the tedious intermediate steps by first entering the
command mode with the function key F8 and then typing into the input line
{\tt r 5}. Upon hitting the return key, you will get the expression: 
%----------------------------------------------------------------------
\begin{verbatim}

ap def 
     Modulo [ X , Y ] = if ( X le Y )
                        then X
                        else Modulo [ ( X - Y ) , Y ] ,
     Gcd [ X , Y ] = if ( X eq Y )
                     then X
                     else if ( X gt Y )
                          then Gcd [ Y , Modulo [ X , Y ] ]
                          else Gcd [ X , Modulo [ Y , X ] ]
   in Gcd
to [ 2 , ap def 
              Modulo [ X , Y ] = if ( X le Y )
                                 then X
                                 else Modulo [ ( X - Y ) , Y ]
            in Modulo
         to [ 4 , 2 ] ]

\end{verbatim}
%---------------------------------------------------------------------

You  may now wish to have evaluated 
the application of {\tt Modulo} in the second argument
position of the outermost application without bothering about the necessary
number of
reduction steps on the one hand (which depends on the actual number of
recursive calls for {\tt Modulo}), and without
having the reduction either stop short of or proceed beyond that point on the
other hand. This is quite easy to accomplish since functional programs are {\mys referentially 
transparent}\index{referential transparency}, i.e., the result of reducing a subexpression
is invariant against the context in which this is done. So, all you need
to do is to first make the application of {\tt Modulo} the new cursor
expression by moving the cursor down under the applicator and then  hit
the function key F9 to trigger a reduction sequence of at most 1000 steps 
(which is way more than necessary to reach a function value). Without
effecting any changes elsewhere,
\pired will return, after only seven reduction steps, the correct value {\tt 2} in
 this argument position and thus complete the second call of {\tt Gcd},as is shown on the next page.

%-----------------------------------------------------------------------
\begin{figure}
\begin{verbatim}

ap def 
     Modulo [ X , Y ] = if ( X le Y )
                        then X
                        else Modulo [ ( X - Y ) , Y ] ,
     Gcd [ X , Y ] = if ( X eq Y )
                     then X
                     else if ( X gt Y )
                          then Gcd [ Y , Modulo [ X , Y ] ]
                          else Gcd [ X , Modulo [ Y , X ] ]
   in Gcd
to [ 2 , 2 ]

\end{verbatim}
\end{figure}
%--------------------------------------------------------------------------
From here on it is quite easy to figure out how many more steps are
required to terminate the program, given that both arguments of
{\tt Gcd} are now equal. It takes one step to instantiate
 the function {\tt Gcd}  one more time, one step to evaluate to {\tt true}
the predicate of the outermost {\tt if\_then\_else} clause in the body of
{\tt Gcd}, and one last step to select as the final result for the greatest
common divisor the value {\tt 2} in the consequence of that clause.
But: before you initiate these steps, don't forget to move the cursor
back to the topmost applicator of
 the expression displayed in the {\sc fa} field, otherwise nothing will happen.

 A session with \pired may be ended either the brutal way, i.e., by typing
{\tt CTR-c} or {\tt CTR-z}, or by entering the command mode and
 then typing {\tt exit}.

Now you have a pretty good idea of how \pired basically works, but
 there is lots more to learn about both the
language \kir and the handling of the editor.

