From udi Thu Apr 14 09:51:15 1994
From: udi (Ulrich Diedrichsen)
Subject: Install-Anleitung
To: wk (Werner Kluge)
Date: Thu, 14 Apr 1994 09:51:12 +0200 (MET DST)
Mime-Version: 1.0
Content-Type: text/plain; charset=US-ASCII
Content-Transfer-Encoding: 7bit
Content-Length: 2930

Moin,

Ich habe eine neue kurze Install-Anleitung geschrieben.
Voraussetzung fuer eine erfolgreich Installation ist, das das pi-red
System mit dem syntaxgesteuerten Editor installiert und compiliert ist.
In der Installationanleitung gehe ich davon aus, das eine Datei
"udi.tar.Z" existiert mit meinen Sourcen. Die Sourcen befinden sich
im Verzeichnis :  ~base/udi

Da mein Englisch nicht so toll ist, habe ich es in deutsch geschrieben :

--------------------------------------------------------------------------------

Um mein Packet zu compilieren/installieren muss man zuerst das File
udi.tar.Z entcompremieren:

  uncompress udi.tar.Z

Danach muss das Tar-Archive-File udi.tar ausgepackt werden :

  tar xvf udi.tar

Hierdurch wird ein neues Unterverzeichnis mit dem Namen ./udi angelegt.

Um die Sourcen zu compilieren muessen folgende Schritte durchgefuehrt werden :

1.)  cd udi

2.)  Die Datei "README" sollte vor der Installation gelesen werden.

3.)  In der Datei "Install" muessen einige Pfade angepasst werden. Bei
     der Anpassung muessen die Hinweise in der Datei beachtet werden.
     Folgende Pfade muessen an die jeweilige Installation angepasst werden :

     REDLIB    = Pfad der pi-red Simulator Library.
     REDSRC    = Sourcecodepfad des pi-red Simulators.
     
     SFEDSRC   = Sourcecodepfad des syntaxgesteuerten Editors.
     XFEDLIB   = Pfad der X-Window Version der Library des 
                 syntaxgesteuerten Editors.

     TOPPATH   = Der absolute Pfad des Verzeichnisses in dem sich das Install
                 File befindet.

     SFED_PATH = Suchpfad der Hilfsdatei (creduma.h) fuer den syntaxgesteuerten
                 Editor

4.)  Aufruf des Install-Shellskripts via : ./Install

5.)  Setzen zweier Environmentvariablen :

      - Die Variable $PATH muss das Verzeichnis ./bin enthalten.
      - Die Variable $FHELPFILE muss auf die Hilfsdatei fuer den 
        syntaxgesteuerten Editor verweisen.

      Das Install-Skript zeigt am Ende der Ausfuehrung einen Vorschlag
      fuer die Belegungen der beiden ENV-Variablen.

6.)   Compilation der Source-Dateien :

      - cd src
      - make depend
      - make

7.)   Nach erfolgreicher Compilation befinden sich im Verzeichnis :

      * udi/bin   Die Ausfuehrbaren Programme:
                  - Xred
                  - kir
                  - marco
                  - sxred

      * udi/lib   Libraries und Beispielmakefiles :
                  - Example.Makefile.Compiler-Editor
                  - Example.Makefile.CompilerGen
                  - editor.o
                  - libkir.o
                  - libmarco.o
                  - marco-main.o

      Eine kurze Erklaerung dieser Dateien findet man in dem README.

8.)   Um die erzeugten Programme und Objektdateien wieder zu loeschen
      muss man in Verzeichnis udi das Shellskript "./DeInstall" aufrufen.


-- 
Ulrich Diedrichsen      udi@informatik.uni-kiel.d400.de

