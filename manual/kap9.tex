\section*{Appendix 1: The Syntax of \kir}
\addcontentsline{toc}{section}{Appendix 1: The Syntax of \kir}
This appendix gives a reasonably complete formal definition of the
 {\mys syntax}\index{\kir syntax} of \kir in the usual Backus/Naur notation. However,
 some of the production rules which exhaustively break down everything into nuts and bolts details will just be explained verbally.
To avoid confusion with the delimiters {\tt <} and {\tt >} of lists
 (sequences) of the syntax proper,
we use the delimiters {\tt $\langle$:} and {\tt :$\rangle$} 
to enclose non-terminal symbols. Likewise, to enclose material
which is to be repeated zero or several times, we use 
the curly brackets (again with colons) {\tt \{:} and
{\tt :\}}.

The basic syntactic entities of which all \kir programs are
constructed  are {\mys expressions}\index{expression}\index{term} (or {\mys terms}). Thus we have a start symbol
{\tt $\langle$:expr:$\rangle$} from which derive the following 
 non-terminals:
$$
\begin{array}{rclc}
{\tt \langle:expr:\rangle} &\Longrightarrow & {\tt \langle:binding\_expr:\rangle} & |\\
& & {\tt \langle:control\_expr:\rangle} & |\\
& & {\tt \langle:ap\_expr:\rangle} & |\\
& & {\tt \langle:struct\_expr:\rangle} & |\\
& & {\tt \langle:const\_expr:\rangle} &
\end{array}
$$
The {\mys binding expressions}\index{binding expression} that derive from {\tt $\langle$:binding\_expr:$\rangle$} are essentially all those which define functions and the
scopes of their formal parameters, including one which directly substitutes non-recursively defined variables by the values of expressions:
$$
\begin{array}{rclc}
{\tt \langle:binding\_expr:\rangle} & \Longrightarrow & {\tt \langle:recfunc\_def:\rangle} & |\\
& & {\tt \langle:letvar\_def:\rangle} & |\\
& & {\tt\langle:func\_def:\rangle} & |\\
& & {\tt\langle:pattern\_match:\rangle} & 
\end{array}
$$
The {\mys control expressions}\index{control expression} break down into {\tt then\_else}
clauses\footnote{Note that complete {\tt if\_then\_else} clauses are
{\mys applications} of 
 {\tt then\_else} clauses to predicate expressions.} and {\tt case}
constructs:
$$
\begin{array}{rclc}
{\tt \langle:control\_expr:\rangle} & \Longrightarrow & {\tt then\;\langle:expr:\rangle\;else\;\langle:expr:\rangle} & |\\
& & {\tt case\;\langle:pattern\_match:\rangle\;\{:,\;\langle:pattern\_match:\rangle:\}} &\\
& & {\tt \{:otherwise\;\langle:expr:\rangle\;:\}\;end}
\end{array}
$$ 
Note here that the {\tt otherwise} expression, contrary to the general
definition of the brackets {\tt \{:} and {\tt :\}}, may appear at most
once in a {\tt case} expression.

The {\mys applicative expressions} come in four alternative forms which,
 depending on the number of arguments and the class (or type) of the expressions that are
in function positions, are appropriately chosen by the \pired editor
when returning intermediate programs:
$$
\begin{array}{rclc}
{\tt \langle:ap\_expr:\rangle} & \Longrightarrow &
{\tt ap\;\langle:expr:\rangle\;to\;[\langle:expr:\rangle\;\{:,\;\langle:expr:\rangle:\}]} & |\\
& & {\tt \langle:expr:\rangle[\langle:expr:\rangle\;\{:,\;\langle:expr:\rangle:\}]} & |\\
& & {\tt (\;\langle:expr:\rangle\;\langle:expr:\rangle\;\langle:expr:\rangle\;)} & |\\
& & {\tt if\;\langle:expr:\rangle\;then\;\langle:expr:\rangle\;else\;\langle:expr:\rangle\;} & 
\end{array}
$$
The {\mys structured expressions}\index{structured expression} partition thus (the item marked $^\ddagger$
 may not yet be implemented):
$$
\begin{array}{rclc}
{\tt \langle:struct\_expr:\rangle} & \Longrightarrow & 
{\tt \langle:tree:\rangle ^\ddagger
\;| \; \langle:sequence:\rangle \; |
\; \langle:vector:\rangle \; |\; \langle:matrix:\rangle} &
\end{array}
$$
and, finally, we have the following {\mys constant expressions}\index{constant expression}:
$$
\begin{array}{rclc}
{\tt \langle:const\_expr:\rangle} & \Longrightarrow & {\tt \langle:var:\rangle\;|\; \langle:num:\rangle\;|\; \langle:bool:\rangle\;} & |\\
& & {\tt \langle:string:\rangle\;|\; \langle:prim\_func:\rangle\;} &
\end{array}
$$
One level down, we have the following production rules for the various binding expressions:
$$
\begin{array}{rclc}
{\tt \langle:recfunc\_def:\rangle} & \Longrightarrow &
{\tt def\;\langle:func\_declar:\rangle\;in\; \langle:expr:\rangle} &\\
{\tt \langle:letvar\_def:\rangle} & \Longrightarrow &
{\tt let\;\langle:var\_declar;\rangle\;in\; \langle:expr:\rangle} &\\
{\tt \langle:func\_def:\rangle} & \Longrightarrow &
{\tt sub[\langle:var\_list:\rangle]\;in\; \langle:expr:\rangle} &\\
{\tt \langle:pattern\_match:\rangle} & \Longrightarrow &
{\tt when\;\langle:patterns:\rangle\;guard\; \langle:expr:\rangle\;do \langle:expr:\rangle} &
\end{array}
$$
The non-terminals other than {\tt $\langle$:expr:$\rangle$} of these
 rules in turn produce
 the various {\mys function} and {\mys variable declarations}\index{function definition}\index{variable declaration} (or
binding constructs):
$$
\begin{array}{rcl}
{\tt \langle:func\_declar:\rangle} & \Longrightarrow & {\tt \langle:func\_def:\rangle\;
\{:,\;\langle:func\_def:\rangle\;:\}}  \; \\
{\tt \langle:func\_def:\rangle} & \Longrightarrow & {\tt \langle:ident:\rangle[\;\langle:id\_list:\rangle\;]\;=\;\langle:expr:\rangle}  \; \\
{\tt \langle:var\_declar:\rangle} & \Longrightarrow & {\tt \langle:ident:\rangle\;=\;\langle:expr:\rangle\;\{:, \langle:ident:\rangle\;=\;\langle:expr:\rangle\;:\}}  \; \\
{\tt \langle:id\_list:\rangle} & \Longrightarrow & {\tt \langle:ident:\rangle\;\{:,\; \langle:ident:\rangle\;:\}\; | \; \epsilon} \; \\
{\tt \langle:patterns:\rangle} & \Longrightarrow & {\tt \langle:pattern:\rangle \; | \;  
(\; \langle:pattern:\rangle\;\{:\;, \langle:pattern:\rangle\;:\})}  \;\\
{\tt \langle:pattern:\rangle} & \Longrightarrow & {\tt \langle:pat\_el:\rangle\;\{:\;, \langle:pattern:\rangle\;:\}} \; |\\
 & & {\tt \langle:user\_constr:\rangle\;\{\;\langle:pattern:\rangle\;\} \; | \; \epsilon}\; \\
{\tt \langle:pat\_el:\rangle} & \Longrightarrow & {\tt \;.\; \; | \; \;\langle:ident:\rangle\;} \;|\; 
 {\tt \;\ldots\; \; | \;  as\;\ldots\; \langle:ident:\rangle\;} \; |\\
& & {\tt \;.+\;} \; | \; {\tt as\;.+\;\langle:ident:\rangle\;} \; |\;
{\tt \;.*\;} \; | \; {\tt as\;.*\;\langle:ident:\rangle\;} \; |\;
{\tt \langle:const:\rangle} \; \; 
\end{array}
$$
 Note that in the various declarations parts we use the non-terminal
{\tt $\langle:$ident$:\rangle$} to denote defining variable occurrences.
They need to be distinguished syntactically
from applied variable occurrences in the
respective body expressions which may be preceded by one or several
protection keys. Thus, for applied variable occurrences
 we have the following production rule:
$$
{\tt \langle:var:\rangle\;\Longrightarrow\;\langle:ident:\rangle\;|\;\backslash \langle:var:\rangle}
$$
The structured expressions break down as follows (again: the item marked $^\ddagger$
 may not yet be implemented):
$$
\begin{array}{rcll}
{\tt \langle:tree:\rangle ^\ddagger} & \Longrightarrow & {\tt <>\;} \;|\; {\tt \;
<\;\langle:expr:\rangle\;,\;\langle:expr:\rangle\;>} &\\
{\tt \langle:sequence:\rangle} & \Longrightarrow & {\tt <>\;} \;|\; {\tt \;
<\;\langle:expr:\rangle\;\{:,\;\langle:expr:\rangle:\}>} &\\
{\tt \langle:vector:\rangle} & \Longrightarrow & {\tt vect<>\;} \;|\; {\tt \;
vect<\langle:num:\rangle\;\{:,\;\langle:num:\rangle\;:\}>} \;|\;\\
& & {\tt tvect<>\;} \;|\; {\tt \;tvect<\langle:num:\rangle\;\{:,\;\langle:num:\rangle\;:\}>} \;|\;\\
& & {\tt bvect<>\;} \;|\; {\tt \;bvect<\langle:bool:\rangle\;\{:,\;\langle:bool:\rangle\;:\}>} \;|\;\\
& & {\tt btvect<>\;} \;|\; {\tt \;btvect<\langle:bool:\rangle\;\{:,\;\langle:bool:\rangle\;:\}>} \;|\;\\
& & {\tt svect<>\;} \;|\; {\tt \;svect<\langle:string:\rangle\;\{:,\;\langle:string:\rangle\;:\}>} &\\
& & {\tt stvect<>\;} \;|\; {\tt \;stvect<\langle:string:\rangle\;\{:,\;\langle:string:\rangle\;:\}>} &\\
{\tt \langle:matrix:\rangle} & \Longrightarrow & 
{\tt mat<<>>.\;} \;|\; {\tt \; mat<\{:<\;\langle:num:\rangle\;\{:,\;\langle:num:\rangle\;>:\}>.} \;|\;\\
& & {\tt bmat<<>>.\;} \;|\; {\tt \; bmat<\{:<\;\langle:bool:\rangle\;\{:,\;\langle:bool:\rangle\;>:\}>.} \;|\;\\
& & {\tt smat<<>>.\;} \;|\; {\tt \; smat<\{:<\;\langle:string:\rangle\;\{:,\;\langle:string:\rangle\;>:\}>.} \\
\end{array}
$$
Note that the production rules for matrices are context-sensitive
 in the sense that the material inside the outermost curly brackets
 represents row vectors,  all of which must have the {\mys same}
 number of elements (which, of course, is not properly specified by
 the simple context-free rules given above).

The following {\mys atoms}\index{atom} may be derived from the non-terminals for
 constant expressions: 
\begin{description}
\item[${\tt \langle:ident:\rangle}$] is any finite sequence of letters, decimal
digits and special characters
which starts with an upper case letter;
\item[${\tt \langle:num:\rangle}$] is any finite sequence of decimal digits with
or without a decimal point;
\item[${\tt \langle:bool:\rangle}$] is either of the Boolean constants {\tt true}
or {\tt false};
\item[${\tt \langle:string:\rangle}$] is any finite sequence of letters, decimal
digits and special characters which is enclosed in a pair of quotation marks {\tt '} and {\tt `};
\item[${\tt \langle:user\_constr:\rangle}$] is a finite sequence of letters starting with an upper case letter.
\end{description}

The following is a complete list of {\mys primitive functions}\index{primitive function} as they can be derived from the non-terminal {\tt prim\_func}
(the symbols in brackets, if
any, are
those used when denoting applications as
 {\tt func[arg\_1,\ldots,arg\_n]}; a symbol followed by $^\ddagger$
 means that the function is not yet
implemented):
\begin{itemize}
\item {\mys monadic value-transforming functions}:
$$
\begin{array}{lclclclc}
{\tt abs} &|& {\tt neg}&|\\
{\tt exp} &|& {\tt ln} &|\\
{\tt sin} &|& {\tt cos} &|& {\tt tan} &|\\
{\tt floor} &|& {\tt ceil} &|& {\tt frac} &|& {\tt trunc} &|\\
{\tt vc\_+\;(vc\_plus)} &|& {\tt vc\_-\;(vc\_minus)} &|& {\tt vc\_*\;(vc\_mult)} &|& {\tt vc\_/\;(vc\_div)}\; &|\\
\end{array}
$$
\item {\mys monadic query functions}:
$$
\begin{array}{lclclc}
{\tt ldim}\;\;\;\;\;\;\;\; &|& {\tt vdim}\;\;\;\;\;\;\;\; &|& {\tt mdim}\;\;\;\;\;\;\;\; &|\\
 {\tt class} &|& {\tt type} &|& {\tt empty}
\end{array}
$$
\item {\mys monadic structuring functions} (the functions marked $^\ddagger$
 which apply
to trees may not yet be implemented):
$$
\begin{array}{lclclclc}
{\tt transpose} &|& {\tt reverse} &|\\
 {\tt to\_scalar}\;\;\; &|& {\tt to\_vector}\;\;\; &|& {\tt to\_tvector}\;\; &|& {\tt to\_matrix}\;\;\; &| \\
{\tt head ^\ddagger
} &|& {\tt tail ^\ddagger
} &\\
\end{array}
$$
\item {\mys dyadic value-transforming functions}: 
$$
\begin{array}{lclclclc}
{\tt +\; (plus)}\;\;\;\; &|& {\tt -\;(minus)}\;\;\; &|& {\tt *\;(mult)}\;\;\;\; &|& {\tt /\;(div)}\;\;\;\;\; &|\\
{\tt mod}  &|& {\tt ip} &|& {\tt max} &|& {\tt min} &|\\
{\tt and} &|& {\tt or} &|& {\tt xor} &|\\
{\tt eq} &|& {\tt ne} &|& {\tt f\_eq}  &|& {\tt f\_ne} &|\\
{\tt ge} &|& {\tt gt}&|& {\tt le} &|& {\tt lt} &\\
\end{array}
$$
\item {\mys dyadic structuring functions}:
$$
\begin{array}{lclclclc}
{\tt lselect}\;\;\;\;\; &|& {\tt lcut}\;\;\;\;\;\;\; &|& {\tt lrotate}\;\;\;\;\; &|& {\tt ++\;(lunite)}\;\;\;\;\; &|\\
{\tt vselect} &|& {\tt vcut} &|& {\tt vrotate} &|& {\tt vunite} &|\\
\end{array}
$$
\item {\mys triadic value-transforming functions}:
$$
\begin{array}{lclclclc}
{\tt c\_+\;(c\_plus)}\; &|& {\tt c\_-\;(c\_minus)} &|& {\tt c\_*\;(c\_mult)}\; &|& {\tt c\_/\;(c\_div)}\;\; &|\\
\end{array}
$$
\item {\mys triadic structuring functions}:
$$
\begin{array}{lclclclc}
{\tt mselect}\;\;\;\;\; &|& {\tt mcut}\;\;\;\;\;\;\; &|& {\tt mrotate}\;\;\;\;\; &|& {\tt munite}\;\;\;\;\; &|\\
{\tt mreplace\_r} &|& {\tt mreplace\_c} &|\\
{\tt ltransform} &|&\\ 
\end{array}
$$
\item {\mys quaternary structuring functions}:
$$
\begin{array}{lclc}
{\tt mreplace}\;\;\;\;\; &|& {\tt repstr} &
\end{array}
$$
\end{itemize}
With the exception of the production rule for matrices, the syntax
 of \kir is context-free. In
particular, any legitimate \kir expression which derives from the start symbol ${\tt \langle:expr:\rangle}$ may be inserted wherever
 this non-terminal occurs on the right hand side of a production rule.
Moreover, \kir is a dynamically typed language. 
No types need to be (and can be) 
specified for the formal parameters of \kir programs, and
 all substitutions by actual parameters are type-free. This is to say that all syntactically correct \kir expressions are legitimate
arguments of defined function applications. 
There is no static type checking that
would reject programs which are not well typed. Type checks are dynamically
performed at run-time upon potential
 instances of $\delta$-reductions. Applications of primitive
 functions that are not type-compatible are treated as constant
 expressions rather than producing error messages, and reductions
 continue in the remaining parts of the program.
Thus, you have ample opportunity to write programs which are rather
meaningless in the sense that they either don't terminate at all or
produce lots of irreducible applications.

This freedom of program design may be considered a
 disadvantage insofar as a rigid monomorphic or polymorphic
 typing discipline is often claimed
to introduce more confidence in program correctness. This point is 
stressed in most other functional or function-based languages~\cite{turn85a,harp88a,huda88}. Programs that are
well typed at least will never produce type errors at run-time. However, they may nevertheless contain many other errors and may not terminate either.
Thus, it is pretty safe to assume that the correctness argument
 to some extent also serves as a pretext for a simple technical reason:
typing is essential when it comes to compiling  programs to
efficiently executable conventional machine code. Since 
contemporary computing machines include only 
rudimentary run-time type checking facilities at best,  the type consistency of programs must be statically
inferred at compile time in order to guarantee that the code is in this respect fail-safe.

Unfortunately, compilation to type-checked code is not compatible with
a full-fledged
 $\lambda$-calculus. Typing itself outlaws many useful higher-order
functions (including self-applications) since all type systems that are
currently in use cannot deal with recursive types~\cite{miln78,hind86}. Moreover,
compiling to static code to some 
extent separates the world of functions from the world of other 
objects. 
Whereas functions can still be passed as
parameters, there is no way of computing new functions from function
applications. Also, compiled and type-checked code must usually run
 to completion in order to produce meaningful and 
intelligible results. Thus, stepwise reduction and inspection of
intermediate programs in a high-level representation cannot be
supported either. 

In contrast to this, the design of \pired was strictly 
guided by the objective
of supporting the reduction
 semantics of a full-fledged $\lambda$-calculus. 
This meant primarily high-level interpretation of $\lambda$-terms
whenever necessary 
(rather than executing just compiled code), and also the absence of a typing discipline.
However, this does not rule out a static type-checker for
\kir programs which infers polymorphic 
types as far as possible. Nevertheless, neither need program execution
be made dependent on the outcome of the type checking process, nor
need all \kir programs be well typed in order
to reduce them to normal forms that are free of irreducible redices. 
In fact, such a type inference system exists for \orel/2~\cite{ples90,zim91} 
and will
in the future be available for \kir as well.
 
