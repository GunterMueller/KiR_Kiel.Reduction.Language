\section*{Appendix 3: Installing \pired on a Host System}
\addcontentsline{toc}{section}{Appendix 3: Installing \pired on your System}

The following primarily concerns the installation of \pired$^+$, i.e., of the
 compiled graph reducer version of \pired, which is our standard version for
release to the public. The installation of the interpreter
 version \pired$^*$ is very similar.

\pired$^+$ will be made available to you on a floppy disc as a compressed
 {\tt tar} file. In order to get it loaded into your system, you must
execute, under the directory under which you wish to install it, the
UNIX commands 
\begin{verbatim}
           tar  xvf  /dev/rfd0c 
           uncompress  red.tar.Z 
           tar  xvf  red.tar
\end{verbatim}
in this sequence. This will open up a new subdirectory {\tt red/} under
 which you can find all the (executable) files and further subdirectories 
necessary to compile and assemble the complete reduction system. 

If you have under the same directory another version of \pired configured for a
different target machine, first execute
\begin{verbatim}
            make  clean
\end{verbatim}
to delete all files that may be invalid. Also, on a UNIX system V/4 
make sure that {\tt /usr/bin} precedes  {\tt usr/ucb} in the directory path
since the {\tt cc} command in the latter uses libraries which have bugs.

If you wish to install \pired on either a SUNSPARC or an Apollo system,
you simply need to run next the executable file 
\begin{verbatim}
            configure  sun|apollo          .
\end{verbatim}
in order to prepare both the editor and the run-time system 
 for compilation to executable code.
For all other systems, you may have to set `by hand' several flags in two
{\tt makefiles} named {\tt fed/Makefile} and {\tt src/Makefile}. These
compiler flags are as follows:

\vspace{5mm}

\noindent Editor flags contained in {\tt fed/Makefile} (with default setting):
 
\vspace{4mm}

\begin{tabular}{ll}
{\tt -DUNIX=1} & for UN*X OS systems;\\
{\tt -DAPOLLO=0} & for Apollo systems set this to 1;\\
{\tt -DTRICK1\_AP=1} & for internal use only (don't change);\\
{\tt -DKEDIT=0} & for use with a distributed version of \pired (don't change);\\
{\tt -DM\_PATT=1} & for multiple pattern matches;\\
{\tt -DTRACE=0} & for internal use only (don't change);\\
{\tt -DClausR=0} & for extended system version with logical variables (don't 
change);\\
{\tt -DB\_DEBUG=0} & for internal use only (don't change);\\
\end{tabular}
\\
\newpage
\noindent Run-time flags contained in {\tt src/Makefile} (with default setting):

\vspace{5mm}

\begin{tabular}{ll}
{\tt -DUNIX=1} & for UN*X OS systems;\\
{\tt -DODDSEX=0} & for Littleendian systems set this to 1;\\
{\tt -DAPOLLO=0} & for Apollo systems set this to 1;\\
{\tt -DSYS5=0} & for System5 systems set this to 1;\\
{\tt -DPI\_RED\_PLUS=1} & for internal use only (don't change);\\
{\tt -DRED\_TO\_NF=1} & enables reductions to normal forms;\\
{\tt -DDEBUG=0} & disables custom-made debug system;\\
{\tt -DDBUG=0} & disables fred-fish's debug package;\\
\end{tabular}

\vspace{5mm}

\noindent Having executed the {\tt configure*} file or set the
 compiler flags directly, you may now call
\begin{verbatim}
            make
\end{verbatim}
which finally produces the executable file {\tt reduma*}. Calling it brings \pired 
into existence on your host system, i.e., the user interface 
shows up on the screen (window) you are
working with.

Before calling {\tt make}, it is advisable to look into the initialization file {\tt red.init} in order to make sure that the system  is
set up with the parameters as you  need them for most of your
 applications. This concerns primarily the sizes of the various
 run-time structures (heap size, stack sizes, etc.). This is good 
practice insofar as any changes of these parameters 
 by 
the respective system commands do not survive a
session, i.e., whenever you call the system with the command 
{\tt reduma}, it is initialized with the old parameters taken from
{\tt red.init} at compile time. 
