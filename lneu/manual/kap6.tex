\section{Structured Objects and Pattern Matching}
\subsection{Primitive Operations on Lists}
The simplest way of representing structured objects in \kir are {\mys lists}\index{list} or {\mys sequences}\index{sequence}
of expressions which have the general form
$$
{\tt <expr\_1,\;\ldots\;,expr\_n>},
$$
where $<$ is an $n$-ary {\mys list constructor}\index{list constructor}, $>$ is an obligatory {\mys end-of-list symbol}\index{end-of-list symbol},
and $expr\_1,\;\ldots\;,expr\_n$ may be any of the legitimate \kir expressions, i.e.,
lists (sequences) may also be recursively nested without any bounds on their depths.

There are a number of primitive {\mys structuring functions}\index{structuring function} available. They
require as arguments a list parameter {\tt list} and an index parameter {\tt index} 
specifying one or several index positions
which are to be affected by the respective operations. 
These applications are reducible if both arguments substituted for the parameters 
are of the respective type and the
{\tt index} value does not exceed the arity of the list. Otherwise the applications remain as
they are.

In particular, the following functions are available:
\begin{description}
\item{\tt lselect[index,list]} which returns from a list the evaluated component
expression found in index position {\tt index};
\item{\tt lcut[index,list]} which cuts off as many of the leading components of a list as
specified by a positive value of {\tt index}, and cuts off as many of the trailing components
of the list as specified by a negative value of {\tt index};
\item{\tt lrotate[index,list]} which rotates a list by as many positions as
 specified by the {\tt index} value
to the right if this value is positive, and to the left if it is negative;
\end{description}

Other functions whose applications require a slightly different syntax include
\begin{description}
\item {\tt lreplace[index,expr,list]} which replaces the component in position {\tt index}
of {\tt list} by the value of {\tt expr}, thereby conceptually creating a new list;
\item{\tt reverse[list]} which simply reverses the order of components of a list;
\item{\tt lunite[list\_1,list\_2] } which appends the components of {\tt list\_2}
to those of {\tt list\_1}, thus forming a new united list.
\end{description}

A simple program which searches for the occurrence of a particular value in a sequence of
integer numbers  may illustrate the use of two of these functions,
{\tt lselect} and {\tt lcut} (see top of next page):

%----------------------------------------------------------------------------------
\begin{figure}
\begin{verbatim}

def 
  Search [ Sequ , El ] = if empty [ Sequ ]
                         then 'not found`
                         else if ( lselect [ 1 , Sequ ] eq El )
                              then 'found`
                              else Search [ lcut [ 1 , Sequ ] , El ]
in Search [ < 5 , 7 , 9 , 4 , 6 , 3 , 6 , 2 , 1 > , 7 ]

\end{verbatim}
\end{figure}
%-----------------------------------------------------------------------------------
The function {\tt Search} basically inspects what is actually 
the first element of a list substituted
for the parameter {\tt Sequ}, and compares it for  equality with a value substituted for {\tt El} and, if
this is the case, returns the string {\tt 'found`}. Otherwise, it is recursively applied
to the remainder of the list until either the next element matches or the list is exhausted
(or empty), in which case the string {\tt 'not found`} is returned. So, when applied to arguments as
above, the program terminates, after two recursive calls of {\tt Search}, with {\tt 'found`}; if the
 second argument is changed to {\tt 8}, it terminates, after 10 recursive calls of {\tt Search}, with
{\tt 'not found`}.

At this point it should be explained why \kir does not yet support the primitive functions
{\tt head}, {\tt tail} and {\tt cons} which you can find in  many other functional or
function-based languages. The reason for this is that lists (sequences) generally are $n$-ary
 structures whereas these functions are usually defined for binary structures. The latter are
merely  special cases of the former which \pired distinguishes neither in the external nor in
the internal (graph) representations. So, implementing these functions could potentially
lead to ambiguities. However, this feature is available in \orel/2 and will soon be implemented in \kir as well. 

Two more things need to be mentioned: If you write programs that need to concatenate
two lists,
you may do so by using either {\tt lunite} or alternatively the infix notation
$$
{\tt (sequ\_1\; ++\; sequ\_2)}
$$
which is semantically the same. Likewise, applications of the function {\tt select} may be
alternatively specified as
$$
{\tt (index\;|\; list)}
$$
In fact, \pired usually transforms occurrences of
{\tt lunite} or {\tt select} into these infix notations when returning intermediate programs.

\subsection{Strict Pattern Matching}
The structuring functions just introduced
 are complete in the sense that they suffice to specify all
application problems involving structured objects (in a strict sense some of them are even
redundant). However, \kir also provides, in the form of {\mys pattern matches}\index{pattern matching}, elegant means to
manipulate structures in a more general sense. 

{\mys Pattern matching} is available in all modern functional languages ~\cite{turn86,lavi87,huda88} and primarily used
as an elegant way of defining functions as ordered sets of alternative equations. In this setting,
pattern matching may simply be seen as a generalized parameter-passing mechanism. 

However, since pattern matching generally is an operation which extracts (sub-) structures of
particular shapes and elements (or element types) from given structural contexts and substitutes
them for placeholders in other structural contexts, its full potential definitely lies in
complex {\mys rule-based modifications}\index{rule-based computation} of structured objects. Exploiting this potential is 
a particularly difficult and hence challenging problem when it comes to performing 
structuring operations on lists (sequences) of heterogeneously typed components with varying 
nesting levels and arities. Here pattern specifications may have to include so-called
{\mys wild cards}\index{wild card} which extract (sub-) structures of variable lengths from otherwise fitting contexts, e.g., in order to decompose a structure depending on the occurrence of specific patterns
in varying syntactical (structural) positions. Such pattern matching generally requires
backtracking and additional {\mys guards} which test for certain properties of the matching
components.  
 
Pattern matches are in \kir defined in terms of special functions which have the form of
so-called {\sc when}-{\mys clauses}\index{{\sc when}-clause}:
$$
{\tt when\;( Pattern\_1,\;Pattern\_2,\;\ldots\;,Pattern\_n)\;guard\;Guard\_expr\;do\;P\_expr}.
$$
In this construct, {\tt Pattern\_i} denotes either a variable, a constant, a type specicification,
or a recursively nested sequence (list) of these items. The {\tt when} constructor, similar to
a {\tt sub}, binds free occurrences of the pattern variables in the
 optional {\mys guard expression}\index{guard expression}
 {\tt Guard\_expr} 
and in the {\mys body expression}\index{function body} {\tt P\_expr}. The application of such a pattern matching
function to $n$ argument terms is denoted as\footnote{Note that the parentheses {\tt (,)}
should be dropped if there is just on pattern specified following the {\tt when}, otherwise
the editor may reject your program.}:
$$
\begin{array}{ll}
{\tt ap}&{\tt case}\;
\\  & {\tt when\;(Pattern\_1;\;\ldots\;,Pattern\_n)\;guard\;Guard\_expr\;do\;P\_expr}\;
\\ & {\tt end}
\\ {\tt to}& {\tt [\;Arg\_1, \;\ldots\;Arg\_n\;]}
\end{array}
$$
and reduces as follows: If the syntactical structures of the patterns and of the corresponding
(evaluated) argument terms match in that for all pairs {\tt Pattern\_i/Arg\_i}
\begin{itemize}
\item for each (sub-) sequence of the pattern there exists a corresponding (sub-) sequence in
the argument so that
\item each variable of the pattern can be associated with a (sub-) structure of the argument,
\item each constant of the pattern matches literally with a constant in an equivalent
syntactical position of the argument;
\item each type specification of the pattern matches with an object of this type in an
equivalent syntactical position of the argument,
\end{itemize}
then all free occurrences of the pattern variables in {\tt Guard\_expr} and {\tt P\_expr}
are substituted by the respective argument (sub-) structures. If, in addition, the guard 
expression thus instantiated evaluates to {\tt true}, then the entire application
reduces to the {\mys weak normal form} of the instantiated body expression {\tt P\_expr}.
Otherwise, the arguments are considered not within the domain of the pattern matching
function (or not of a compatible type), i.e., the application remains as it is. Following this
interpretation, a pattern in fact defines a {\mys structured data type}\index{structured data type}: it is the set of all
objects which match the pattern and satisfy the guard.

When applying pattern matches, they must by convention be embedded in
so-called {\mys {\sc case}-constructs}\index{{\sc case}-construct}. To do so, you may simply type in the input line a fragment of it, 
{\tt ap case when}, whereupon the editor will return in the {\sc fa} field:
%---------------------------------------------------------------------------------
\begin{verbatim}

ap case 
   when # guard # do #
   #
to [ # ]


\end{verbatim}
%-----------------------------------------------------------------------------------
Now you may proceed to replace the empty symbols as you like. Note that the editor
inserts a {\tt \#} even in the place where you would expect the keyword {\tt end}.
This relates to a generalization of the {\tt case} construct to which we will return
a little later.

You should play with a few simple example programs to find out what pattern matching can do for
you. The following may be quite instructive:
%------------------------------------------------------------------------------------
\begin{verbatim}

ap case 
   when ( X , Y , 2 ) guard ( empty [ lcut [ 1 , Y ] ] eq false ) do X [ Y ]
   end
to [ reverse , < 1 , 2 , 3 , 4 > , 2 ]

\end{verbatim}
%-------------------------------------------------------------------------------------
Here you have under the {\tt when} clause three simple patterns, of which two are 
variables and one is a constant. The guard tests whether the argument substituted for the
second pattern, i.e., for the variable {\tt Y}, is a list of at least two elements. This being the case, the argument substituted
for {\tt X} is applied to the argument substituted for {\tt Y} in the body of the pattern
matching function. Since the list of arguments comprises the 
primitive list function {\tt reverse} in the first position, a list of four numbers in the second
 position, and the constant value {\tt 2} in the third position, the pattern match ought to succeed.
So, let's see what happens step-by-step.

After one reduction step, \pired  will produce on your screen the application of a 
modified {\tt case} construct to the same list of arguments (note that the layout mode is
from now on changed to {\tt vertical} in order to have all expressions fit onto the page):
%---------------------------------------------------------------------------------------
\begin{verbatim}

ap case 
   when ( . , . , . )
    guard ( empty( lcut( 1 , < 1 , 2 , 3 , 4 > ) ) eq false )
    do reverse( < 1 , 2 , 3 , 4 > )
   otherwise ap case 
                when ( X , Y , 2 )
                 guard ( empty( lcut( 1 , Y ) ) eq false )
                 do X [ Y ]
                end
             to [ reverse , < 1 , 2 , 3 , 4 > , 2 ]
   end
to [ reverse , < 1 , 2 , 3 , 4 > , 2 ]

\end{verbatim}
%------------------------------------------------------------------------------------
The patterns in the {\tt when} clause are now replaced by dots, indicating that the patterns
did match the arguments (a variable matches every legitimate \kir expression), as a consequence
of which they have been substituted for all occurrences of the pattern variables in both
the guard expression and the body expression, i.e., the function {\tt reverse} for {\tt X}
and the list {\tt <1,2,3,4>} for {\tt Y}. In addition to this, \pired has extended the {\tt case} 
construct by  an {\mys otherwise expression}\index{otherwise expression} consisting of the complete original application.
This is the {\mys escape hatch} for the case that the guard expression reduces to something
other than the Boolean constant {\tt true}. However, another three reduction steps will reduce the
guard to {\tt true} and throw this alternative away, returning:
%-----------------------------------------------------------------------------------------
\begin{verbatim}

reverse( < 1 , 2 , 3 , 4 > ) ,

\end{verbatim}
%--------------------------------------------------------------------------------------------
which in one more step reduces to
%----------------------------------------------------------------------------------------
\begin{verbatim}

< 4 , 3 , 2 , 1 > .

\end{verbatim}
%------------------------------------------------------------------------------------
Now you may try to to reduce the same program with just one element in the argument
list:
%----------------------------------------------------------------------------------
\begin{verbatim}

ap case 
   when ( X , Y , 2 )
    guard ( empty [ lcut [ 1 , Y ] ] eq false )
    do X [ Y ]
   end
to [ reverse , < 1 > , 2 ]

\end{verbatim}
%-------------------------------------------------------------------------------------
The guard term now reduces to {\tt false} since {\tt lcut[1,<1>]}
yields the empty list {\tt <>}. As a consequence of this, the copy of the original application
held as {\tt otherwise} expression is reproduced as result. \pired suppresses further reductions
on this application in order to avoid unending recursions.

The pattern matches themselves may be made to fail by the following variant of this program:
%------------------------------------------------------------------------------------------
\begin{verbatim}

ap case 
   when ( X , Y , 2 )
    guard ( empty( lcut( 1 , Y ) ) eq false )
    do X [ Y ]
   end
to [ reverse , < 1 , 2 , 3 , 4 > , 3 ]

\end{verbatim}
%-------------------------------------------------------------------------------------------
Here you have in the third argument position the constant value {\tt 3} which does not
match the {\tt 2} in the corresponding pattern position. So, the pattern match is not at all
applicable to the argument list, i.e., \pired reproduces the application without performing
a reduction step (or leaves it as it is). 

You may test for yourself other odd argument sets, e.g., a constant value,
an application or a variable in the second argument position, where 
a list is expected. In all cases, the system reproduces the original application
since neither of these arguments belongs to the domain of the pattern matching function.

Another example program which shows you how a nested sequence can be decomposed
is the following:
%----------------------------------------------------------------------------------
\begin{verbatim}

ap case 
   when  < < X , Y > , Z >  guard true do Z [ X , Y ]
   end
to [ < < 3 , 2 > , + > ]

\end{verbatim}
%--------------------------------------------------------------------------------------
Here you have a pattern which exactly matches the argument. With the guard trivially set to
{\tt true}, \pired returns after one reduction step the instantiated body (in infix notation 
though) {\tt ( 3 + 2 )}, and after one more step the value {\tt 5}.

The match does also succeed if you have substructures rather than atomic components
in your argument sequence since, again, a variable in the pattern matches all legitimate 
\kir expressions. Thus, an application of the same pattern matching function to a list
of lists 
%---------------------------------------------------------------------------------------
\begin{verbatim}

ap case 
   when  < < X , Y > , Z >  guard true do Z [ X , Y ]
   end
to [ < < < 1 , 2 > , < 3 , 4 > > , ++ > ]

\end{verbatim}
%-----------------------------------------------------------------------------------------
reduces in one step to {\tt ( < 1 , 2 > ++ < 3 , 4 > )} and in another step to
{\tt < 1 , 2 , 3 , 4 >}.

There is a mismatch between the pattern and the argument if the latter contains fewer
or more components on a nesting level than are specified in the former, e.g., as in
%----------------------------------------------------------------------------------------
\begin{verbatim}

ap case 
   when < < X , Y > , Z > guard true do Z [ X , Y ]
   end
to [ < < < 1 , 2 > , < 3 , 4 > , < 5 , 6 > > , ++ > ]

\end{verbatim}
%----------------------------------------------------------------------------------------
 or in
%-----------------------------------------------------------------------------------
\begin{verbatim}

ap case 
   when < < X , Y > , Z > guard true do Z [ X , Y ]
   end
to [ < < < 1 , 2 > > , ++ > ] .

\end{verbatim}
%-----------------------------------------------------------------------------------
Neither of these two applications therefore constitutes an instance of reduction:
they are constant expressions which remain unchanged.

As you may have expected, the {\mys {\sc case}-construct}\index{{\sc case}-construct} can be expanded to
include several pattern matches and an optional {\tt otherwise} expression.
Applications of such an {\mys alternative pattern match}\index{alternative pattern match} have the general syntactical form: 

$$
\begin{array}{ll}
{\tt ap}&{\tt case}\;
\\  & \;\;{\tt when\;(Pattern\_11;\;\ldots\;,Pattern\_1n)\;guard\;Guard\_expr\_1\;do\;P\_expr\_1}\;
\\ & \;\;\;\;\;\;\vdots
\\  & \;\;{\tt when\;(Pattern\_m1;\;\ldots\;,Pattern\_mn)\;guard\;Guard\_expr\_m\;do\;P\_expr\_m}\;
\\ & \;\;{\tt otherwise \; Esc\_expr}
\\ & {\tt end}
\\ {\tt to}& {\tt [\;Arg\_1, \;\ldots\;Arg\_n\;]}
\end{array}
$$
 
The {\tt case} construct constitutes an $n$-ary function which, when applied to $n$ argument
terms, tries all pattern matches in the sequence from top to bottom. The first
successful match, including a satisfiable guard, is reduced to its (weak) normal form which,
by definition, is also the normal form of the entire {\tt case} application. If none of the
pattern/guard combinations matches, then the result is the normal form of the
{\tt otherwise} expression as it is either explicitely specified or, if this specification
 is missing, to the {\tt case} application itself which \pired alternatively takes for it (compare
also the first example program in this section).

\subsection{An Example : Quicksorting}
Let us now study a more elaborate program which uses alternative pattern matching in a
rather tricky way: {\mys quicksorting} a sequence of integer values.

The {\mys quicksort algorithm}\index{quicksort algorithm} basically works as follows: If the sequence
is empty or contains just one element, then it is already sorted, and you are done.
If it contains two or more elements, then the second element is taken as a so-called
{\mys pivot number} against which all elements of the sequence are compared: those that are
smaller are put into a first subsequence, those that equal the pivot number are
put into a second subsequence, and those that are greater are sorted into a third 
subsequence. Subsequently, the first and the third subsequence are recursively quicksorted
until they contain only one element or are empty and thus are sorted. The second subsequence is
trivially sorted. Finally, all sorted subsequences are recursively concatenated to a flat
sequence which has all elements of the original sequence  sorted in ascending order. 

A \kir program that realizes this algorithm is shown on the opposite page (note that the vertical layout
mode is on, otherwise the program would not fit onto the page).

This program breaks down into the definition of a function
\begin{description}
\item[{\tt Quicksort}] which in its body uses a {\tt case} construct to check 
whether the argument substituted for the parameter {\tt Sequ} is 
an empty sequence or a sequence of one element, and otherwise applies to it 
another pattern match which constructs the recursive calls of {\tt Quicksort} to the 
subsequences received from an application of the function
\item[{\tt Sort}] which sorts, by means of the parameter {\tt Pivot}, the
sequence substituted for {\tt Sequ} into three subsequences named {\tt U,V,W},
\end{description}
as just described. The goal expression of the {\tt def} construct applies
{\tt Quicksort} to a sequence of integer values.
%-------------------------------------------------------------------------------------
\begin{figure}
\begin{verbatim}

def 
  Quicksort [ Sequ ] =
   ap case when <> guard true do <> ,
      when < X > guard true do < X >
      otherwise ap case 
                   when < S_1 , S_2 , S_3 > guard true
                    do ( Quicksort [ S_1 ] ++ ( S_2 ++ Quicksort [ S_3 ] ) )
                   end
                to [ Sort [ lselect [ 2 , Sequ ] , Sequ , <> , <> , <> ] ]
      end
   to [ Sequ ] ,
  Sort [ Pivot , Sequ , U , V , W ] =
   let First = lselect [ 1 , Sequ ]
   in if empty [ Sequ ]
      then < U , V , W >
      else Sort [ Pivot ,
                     lcut [ 1 , Sequ ] ,
                                   if ( First lt Pivot )
                                   then ( U ++ < First > )
                                   else U , if ( First eq Pivot )
                                            then ( V ++ < First > )
                                            else V , if ( First gt Pivot )
                                                     then ( W ++ < First > )
                                                     else W ]
in Quicksort [ < 6 , 4 , 8 , 1 , 9 , 4 , 3 , 2 > ]
 
\end{verbatim}
\end{figure}
%-----------------------------------------------------------------------------------

Now, let's see how this program reduces.

The first reduction step just expands the goal expression by substituting
 the argument sequence for all occurrences of {\tt Sequ} and the entire
{\tt def} construct for all occurrences of {\tt Quicksort} and {\tt Sort}
in the body of {\tt Quicksort}. The next step reduces the application of the 
outermost {\tt case} construct of this body, returning the {\tt otherwise} expression
since the sequence contains more than one element (see next page).
%-------------------------------------------------------------------------------------
\begin{figure}
\begin{verbatim}
 
ap case 
   when < S_1 , S_2 , S_3 > guard true
    do ( ap def 
              Quicksort [ Sequ ] =
               ap case 
                  when <> guard true do <> ,
                  when < X > guard true do < X >
                  otherwise ap case 
                               when < S_1 , S_2 , S_3 > guard true
                                do ( Quicksort [ S_1 ] ++ ( S_2 ++ Quicksort [ S_3 ] ) )
                               end
                            to [ Sort [ ( 2 | Sequ ) , Sequ , <> , <> , <> ] ]
                  end
               to [ Sequ ] ,
              Sort [ Pivot , Sequ , U , V , W ] =
               let First = ( 1 | Sequ )
               in if empty( Sequ )
                  then < U , V , W >
                  else Sort [ Pivot ,
                                lcut( 1 , Sequ ) ,
                                  if ( First lt Pivot )
                                  then ( U ++ < First > )
                                  else U , if ( First eq Pivot )
                                  then ( V ++ < First > )
                                  else V , if ( First gt Pivot )
                                           then ( W ++ < First > )
                                           else W ]
            in Quicksort
         to [ S_1 ] ++ ( S_2 ++ ap def 
                                     Quicksort [ Sequ ] = 
                                         ap case
                                            when <>
                                             guard true
                                             do <>,
                                            ?
                                         to [ Sequ ] ,
                                        ?
                                   in Quicksort
                                to [ S_3 ] ) )
   end
to [ ? ]

\end{verbatim}
\end{figure}
%-------------------------------------------------------------------------------------------

The important part of this expression is what is hiding under the question mark in the
argument position of the new {\tt case} application, which is the application of {\tt Sort}
to concrete argument terms. This application is supposed to reduce to a sequence of
three subsequences which are sorted against the second element of the original argument
sequence. When moving the cursor under this question mark and pressing the function key
F9 (for 1000 reduction steps), you get in this position, as expected,
%-----------------------------------------------------------------------------------------
\begin{verbatim}

< < 1 , 3 , 2 > , < 4 , 4 > , < 6 , 8 , 9 > >

\end{verbatim}
%------------------------------------------------------------------------------------------
The next reduction step applies the pattern {\tt < S\_1 , S\_2 , S\_3 > } to this argument
and creates recursive calls of {\tt Quicksort} to the sequences associated with
{\tt S\_1} and {\tt S\_3} which are concatenated with the sequence bound to {\tt S\_2}
(in order to avoid unnecessary details, the layout modes {\tt abb\_def} and {\tt dgoal} are
turned on here):
%--------------------------------------------------------------------------------------
\begin{verbatim}

( ap def* 
     in Quicksort
  to [ < 1 , 3 , 2 > ] ++ ( < 4 , 4 > ++ ap def* 
                                            in Quicksort
                                         to [ < 6 , 8 , 9 > ] ) )

\end{verbatim}
%---------------------------------------------------------------------------------------
If you now move the cursor to the innermost application of {\tt Quicksort} and reduce it
locally, you will get
%----------------------------------------------------------------------------------------
\begin{verbatim}

( ap def* 
     in Quicksort
  to [ < 1 , 3 , 2 > ] ++ ( < 4 , 4 > ++ ( ap def* 
                                              in Quicksort
                                           to [ < 6 > ] ++ ( < 8 > ++ ap def* 
                                                                         in Quicksort
                                                                      to [ < 9 > ] ) ) ) )      ,

\end{verbatim}
%--------------------------------------------------------------------------------------------
and when locally reducing the new applications of {\tt Quicksort} on the right
%------------------------------------------------------------------------------------
\begin{verbatim}

( ap def*
     Quicksort
  to [ < 1 , 2 , 3 > ] ++ ( < 4 , 4 > ++ ( < 6 > ++ ( < 8 > ++ < 9 > ) ) ) )          ,

\end{verbatim}
%------------------------------------------------------------------------------------
and so on, until you get as the final result the sorted list
%--------------------------------------------------------------------------------
\begin{verbatim}

< 1 , 2 , 3 , 4 , 4 , 6 , 8 , 9 >     .

\end{verbatim}
%------------------------------------------------------------------------------------
If you have fun playing with this program, you may try a somewhat larger argument sequence, say,
%--------------------------------------------------------------------------------------
\begin{verbatim}

< 54 , 87 , 5 , 2 , 98 , 45 , 23 , 45 , 56 , 12 , 343 , 87542 ,
  98 , 56 , 5 , 3 , 5 , 8 , 1 , 9 , 7 , 56 , 44 , 45 , 76 , 87 ,
  98 , 99 , 12 , 34 , 45 , 56 , 76 , 54 , 43 , 32 , 21 , 98 ,
  99 , 5 , 2 , 3 , 4 , 45 , 56 , 88 , 3 , 5 , 0 , 99 , 88 , 23 ,
  12 , 54 , 87 , 67 , 46 , 86 , 98 , 3 , 4 , 6 , 98 , 0 , 45 ,
  65 , 3 , 1 , 9 , 7 , 65 , 5 , 4 , 98 , 6532 , 34 , 76 , 56 ,
  2 , 3 , 98 , 6 , 5 , 4 , 3 , 8 , 7 , 1 , 19 , 12 , 32 , 19 ,
  5 , 6 , 3 , 9 , 7 , 12 , 14 , 16 , 82 , 1 , 2 , 54 , 5 , 4 ,
  3 , 2 , 99 , 9 , 88 , 8 , 99 , 5 , 3 , 6 , 6 , 7 , 5 , 4 , 3 ,
  99 , 88 , 77 , 66 , 55 , 44 , 88 , 99 , 44 , 66 , 77 , 88 ,
  11 , 12 , 13 , 77 , 55 , 5 , 6 >

\end{verbatim}
%------------------------------------------------------------------------------------
and reduce it in one go by setting the reduction counter to  some generous value, say 
{\tt 9999999}, which by all means suffices to have the computation terminate. Almost instantly,
\pired will respond with the sorted list
%------------------------------------------------------------------------------------
\begin{verbatim}

< 0 , 0 , 1 , 1 , 1 , 1 , 2 , 2 , 2 , 2 , 2 , 3 , 3 , 3 , 3 , 3 ,
  3 , 3 , 3 , 3 , 3 , 3 , 4 , 4 , 4 , 4 , 4 , 4 , 5 , 5 , 5 , 5 ,
  5 , 5 , 5 , 5 , 5 , 5 , 5 , 5 , 6 , 6 , 6 , 6 , 6 , 6 , 7 , 7 ,
  7 , 7 , 7 , 8 , 8 , 8 , 9 , 9 , 9 , 9 , 11 , 12 , 12 , 12 , 12 ,
  12 , 12 , 13 , 14 , 16 , 19 , 19 , 21 , 23 , 23 , 32 , 32 , 34 ,
  34 , 43 , 44 , 44 , 44 , 45 , 45 , 45 , 45 , 45 , 45 , 46 , 54 ,
  54 , 54 , 54 , 55 , 55 , 56 , 56 , 56 , 56 , 56 , 56 , 65 , 65 ,
  66 , 66 , 67 , 76 , 76 , 76 , 77 , 77 , 77 , 82 , 86 , 87 , 87 ,
  87 , 88 , 88 , 88 , 88 , 88 , 88 , 98 , 98 , 98 , 98 , 98 , 98 ,
  98 , 98 , 99 , 99 , 99 , 99 , 99 , 99 , 99 , 343 , 6532 , 87542 > ,

\end{verbatim}
%------------------------------------------------------------------------------------
after having  actually performed {\tt 8782} reduction steps.

\subsection{Pattern Matching with Wild Cards}
Sofar you have only learned about pattern matches that are strict in the sense that the argument
terms must {\mys exactly match} the structures specified in the patterns, which is
to say that the arities
of all matching  nodes (which are the list constructors) must be exactly the same. However,
when dealing with $n$-ary lists (sequences), this requirement is often too restrictive to
specify real-life applications in an elegant and concise way. For instance, you may wish to
search through a sequence  of expressions for the occurrence of a particular substructure but you neither
know nor are interested in its exact position (it may not even occur at all),
 and you also don't care about the other
components of the sequence. Or, you may wish to identify in a sequence of variable length
one or several substructures,
modify or remove them, and do something else with the remaining fragments.

\kir supports these applications by means of patterns that may include so-called
{\mys wild cards}\index{wild card}\index{patterns with wild cards}. Syntactically, these are atomic pattern components (or leaf nodes) just
like variables or constants.
However, semantically they  are to match against subsequences of varying lengths, depending
on the pattern context in which they occur on the one hand, and on the particular shapes
of the arguments on the other hand. Wild cards may or may not be
combined with pattern variables, in which case matching subsequences of the arguments
are, as before, substituted for occurrences of these variables in the respective body expressions.  

The following wild cards are available or scheduled for a later release of \kir (the latter are
marked by a $^\ddagger$ ):
\begin{itemize}
\item  {\tt .  } or {\tt Var} which matches against a subsequence of exactly one item
length;
\item  {\tt ...} or {\tt as ... Var} which covers  subsequences of minimal lengths,
including empty sequences;
\item  {\tt .+}$^\ddagger$  or {\tt as .+ Var}$^\ddagger$ which covers  subsequences of minimal lengths, excluding
empty sequences;
\item {\tt .*}$^\ddagger$ or {\tt .* Var}$^\ddagger$ which covers subsequences of maximal lengths, including empty sequences.
\end{itemize}
As you may have guessed, the construct {\tt as wild\_card Var} is the one that substitutes
 the matching subsequences for occurrences of the variable {\tt Var}. 

Some simple examples may help you to get an idea as to what can be done with these wild cards.

The first one is to demonstrate, in several variants, which parts of a sequence are matched
by the various wild cards.
%--------------------------------------------------------------------------------------
\begin{verbatim}

ap case 
   when < ... , < U , V > , as ... P , < W , Z > , ... >
    guard true do < < U , W , Z , V > , P >
   end
to [ < 1 , 2 , < 4 , 5 > , 3 , < 6 , 6 > , < 8 , 8 , 8 > , < 7 , 9 > , 4 , 3 > ]

\end{verbatim}
%-----------------------------------------------------------------------------------------
Here you can expect a match of the following kind: 
\begin{itemize}
\item the {\tt ... } matches against the first two elements of the sequence which are simply
ignored;
\item the {\tt <U,V>} matches against the first substructure of two elements, which is {\tt <4,5>};
\item the {\tt as ... P} matches against the shortest intervening subsequence that is between the last and the next 
substructure of two elements, which is just the {\tt 3};
\item the {\tt <w,z>} matches  agains the {\tt <6,6>}, and
\item the remaining {\tt ...} matches against the rest of the sequence which, again, is
ignored.
\end{itemize}
Thus, after one reduction step you get:
%--------------------------------------------------------------------------------------
\begin{verbatim}

< < 4 , 6 , 6 , 5 > , < 3 > >

\end{verbatim}
%---------------------------------------------------------------------------------------
If you now change the wild card {\tt as ... P} to {\tt as .* P} (assuming it would be available), it covers a subsequence
of maximal length to the next occurrence of a subsequence with two elements (which is
{\tt <7,9>}):
%-----------------------------------------------------------------------------------
\begin{verbatim}

ap case 
   when < ... , < U , V > , as .* P , < W , Z > , ... >
    guard true do < < U , W , Z , V > , P >
   end
to [ < 1 , 2 , < 4 , 5 > , 3 , < 6 , 6 > , < 8 , 8 , 8 > , < 7 , 9 > , 4 , 3 > ]

\end{verbatim}
%--------------------------------------------------------------------------------------
This would reduce in one step to:
%--------------------------------------------------------------------------------
\begin{verbatim}

< < 4 , 7 , 9 , 5 > , < 3 , < 6 , 6 > , < 8 , 8 , 8 > > >
 
\end{verbatim}
%--------------------------------------------------------------------------------
Another interesting variant is this one:
%---------------------------------------------------------------------------------
\begin{verbatim}

ap case 
   when < .* , < U , V > , as ... P , < W , Z > , ... >
    guard true do < < U , W , Z , V > , P >
   end
to [ < 1 , 2 , < 4 , 5 > , 3 , < 6 , 6 > , < 8 , 8 , 8 > , < 7 , 9 > , 4 , 3 > ]

\end{verbatim}
%--------------------------------------------------------------------------------
Here we have in the first pattern position a wild card requiring a top-level 
subsequence of maximal length to a substructure of two elements which, however,
must be followed by another subsequence after which there must a second
substructure of two elements. This is generally not easy to figure out, requiring
a lot of backtracking and retrying, until the correct solution is found.
However, the pattern matching facilities of \pired  will soon be
 smart enough to handle this and return
as the correct solution the one that matches {\tt <U,V>} against {\tt <6,6>}
and {\tt <W,Z>} against {\tt <7,9>}:
%-----------------------------------------------------------------------------------
\begin{verbatim}

< < 6 , 7 , 9 , 6 > , < < 8 , 8 , 8 > > >

\end{verbatim}
%-----------------------------------------------------------------------------------
You may play with  other variants of this example for yourself, e.g., with more
than just two strict sub-patterns and different wild cards in between, in order to
develop a better understanding of how this form of pattern matching works.

A last example which shows how pattern matching with wild cards can be used in 
the context of recursive functions is a program which tests a sequence of
integer values for the {\mys palindrome}\index{palindrome property} property, meaning that reading the 
sequence from left to right is the same as reading it from right to left. This problem
can be specified as follows:
%-------------------------------------------------------------------------------
\begin{verbatim}

def 
  Pal [ Sequ ] =  ap case 
                     when <> guard true do 'palindrome` ,
                     when < . > guard true do 'palindrome` ,
                     when < U , as ... W , V > guard ( U eq V ) do Pal [ W ]
                     otherwise 'no palindrome`
                     end
                  to [ Sequ ]
in Pal [ < 1 , 2 , 3 , 4 , 4 , 3 , 2 , 1 > ]

\end{verbatim}
%------------------------------------------------------------------------------------
The {\tt case} construct in the body of the function {\tt Pal}
basically does the following: if the argument sequence substituted for {\tt Sequ}
is either empty or contains just one element, then the sequence is a palindrome
and we are done. If it contains more than one element, the third {\tt when}
clause extracts the first and the last element, compares them for equality and, if this
is the case, applies the function {\tt Pal} recursively to the subsequence without the
first and the last element. If none of the patterns matches, then the sequence is no
palindrome.

As you reduce this program step-by-step, you will note that in cycles of four steps
(which it takes to complete recursive calls of {\tt Pal}) \pired returns the initial program
 with the actual first and last elements cut off the argument sequence, until the sequence is
empty and you get the string {\tt 'palindrome`} as result. 
