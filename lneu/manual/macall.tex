%--------- first of 597 lines --------------------------
\newcommand{\mys}{\sf} % sans serif (!!); bold; italics
\newcommand{\titel}[1]{{\em #1}}
\newcommand{\orel}{{\sc Orel}}
\newcommand{\BeReL}{{\sc Orel}}
\newcommand{\pired}{$\pi$--{\sc red }}
\newcommand{\kir}{{\sc KiR }}
%-------------------------------------
%----------- allmac : first of 3654 lines -----------
%-- f62.tex ----------------------------------------------------------
\newcommand{\myluarrow}[2]{%
\begin{picture}(2,#1)
\linethickness{0mm}
\put(0,0){\framebox(2,#1)[c]{%
$\left.\rule{0mm}{#2\unitlength} \right\Uparrow$}}
\end{picture}
}
%%::::::::::::::
%%mac03.tex
%%::::::::::::::
%----------- mac03 : first of 42 lines -----------
%-- sollte in macall bleiben !! ------------------
%  macros : aus kap3
%----------------------------------------
\newcommand{\ksub}[1]{{\bf #1}}
\newcommand{\kop}[1]{{\bf #1}}
\newcommand{\kif}[1]{{\bf #1}}
\newcommand{\kds}[1]{{\bf #1}}
%-------------------------------------
%----------- mac03 : last of 42 lines --------------------------
%%::::::::::::::
%%mac04.tex
%%::::::::::::::
%-- f41.tex, f62.tex, f613.tex, f8.18 (f719.tex) --------------------
\newcommand{\mynewldarrow}[1]{%
\begin{picture}(2,2)(0,-4)
\linethickness{0mm}
\put(0,-3.9){\framebox(2,4)[c]{$\Downarrow$}}
\thinlines
\put(0.4,0){\line(0,1){#1}}
\put(1.4,0){\line(0,1){#1}}
\put(1.4,#1){\line(0,1){0.95}}
\end{picture}
}
%-- f41.tex, f62.tex ----------------------------------------------------------
\newcommand{\mynewluarrow}[1]{%
\begin{picture}(2,2)(0,-0.8)
\linethickness{0mm}
\put(0,#1){\framebox(2,4)[c]{$\Uparrow$}}
\thinlines
\put(0.5,0.1){\line(0,1){#1}}
\put(1.4,0.1){\line(0,1){#1}}
\put(1.4,-0.8){\line(0,1){1.1}}
\end{picture}
}
%-- f41.tex, f62.tex, f8.18 (f719.tex) --------------------
\newcommand{\mynewtlarrow}[1]{%
\begin{picture}(10,2)(-4,0)
\linethickness{0mm}
\put(-3.8,-0.25){\framebox(4,2)[c]{$\Leftarrow$}}
\thinlines
\put(0,1.5){\line(1,0){#1}}
\put(0,0.55){\line(1,0){#1}}
\put(#1,1.5){\line(1,0){0.8}}
\end{picture}
}
%-- f41.tex, f62.tex, f8.18 (f719.tex) --------------------
\newcommand{\mynewblarrow}[1]{%
\begin{picture}(10,2)(-4,0)
\linethickness{0mm}
\put(-3.8,-0.25){\framebox(4,2)[c]{$\Leftarrow$}}
\thinlines
\put(0,1.5){\line(1,0){#1}}
\put(0,0.5){\line(1,0){#1}}
\put(#1,0.5){\line(1,0){0.8}}
\end{picture}
}
%%::::::::::::::
%%mac0407.tex
%%::::::::::::::
%----- mac0407 : first of 17 lines ------------------------------------
%-------------------------------------------------------
%   center left mathmode
%   (array mit einer Spalte)
%-------------------------------------------------------
%
\newcommand{\mcent}[1]{%
\def\ls{\\\noalign{\smallskip}}
\def\lm{\\\noalign{\medskip}}
\begin{displaymath}
\begin{array}{l}
\openup10pt
#1
\end{array}
\end{displaymath}}
%
%----- mac0407 : first of 17 lines ------------------------------------
%%::::::::::::::
%%mac05.tex
%%::::::::::::::
%%::::::::::::::
%%mac0506.tex
%%::::::::::::::
%----------- mac0506 : first of 12 lines -----------
% macros in kap5, kap6
%-- im text  ---------------------
%waagerechter Pfeil mit kleinem Strich
\newcommand{\pfeil}{\mbox{%
\protect\begin{picture}(10,4)
\protect\put(1,1){\vector(1,0){8}}
\protect\put(1,0){\line(0,1){2}}
\protect\end{picture}}}
%----------- mac0506 : first of 12 lines -----------
%%::::::::::::::
%%mac06.tex
%%::::::::::::::
%%::::::::::::::
%%mac0608.tex
%%::::::::::::::
%----------- mac0608 : first of 63 lines -----------
%-- alg boxes im text ------
\newcommand{\EAR}{\begin{picture}(13,4)
\put(6,1.5){\oval(12,4.2)}
\put(1.3,0){$EAR$}
%put(0,-0.5){\framebox(12,4)[c]{$EAR$}}
\end{picture}
}
%---------------------------
%----------- macarrow : first of 63 lines -----------
%-- f62.tex, f613.tex, f8.18 (f719.tex) --------------------
\newcommand{\myldarrow}[2]{%
\begin{picture}(2,#1)
\linethickness{0mm}
\put(0,0){\framebox(2,#1)[c]{$\left.\rule{0mm}{#2\unitlength} \right\Downarrow$}}
\end{picture}
}
%-- f6.2, f8.18 (f719.tex) --------------------
\newcommand{\mylrarrow}[1]{%
\begin{picture}(10,2)(-4,0)
\linethickness{0mm}
\put(-3.5,-0.25){\framebox(4,2)[r]{$\Leftarrow$}}
\put(#1,-0.25){\framebox(4,2)[l]{$\Rightarrow$}}
\thinlines
\put(0.2,1.5){\line(1,0){#1}}
\put(0.2,0.4){\line(1,0){#1}}
\end{picture}
}
%-- f62.tex, f8.18 (f719.tex) --------------------
\newcommand{\mylarrow}[1]{%
\begin{picture}(10,2)(-4,0)
\linethickness{0mm}
\put(-3.7,-0.3){\framebox(4,2)[c]{$\Leftarrow$}}
\thinlines
\put(0,1.4){\line(1,0){#1}}
\put(0,0.4){\line(1,0){#1}}
\end{picture}
}
%-- f62.tex, f8.18 (f719.tex) --------------------
\newcommand{\myrarrow}[1]{%
\begin{picture}(10,2)
\linethickness{0mm}
\put(#1,-0.3){\framebox(4,2)[c]{$\Rightarrow$}}
\thinlines
\put(0.3,1.4){\line(1,0){#1}}
\put(0.3,0.4){\line(1,0){#1}}
\end{picture}
}
%----------- macarrow : last of 63 lines -----------
%----------- mac0608 : last of 63 lines -----------
%%::::::::::::::
%%mac07.tex
%%::::::::::::::
%----- mac07 : first of 74 lines ------------------------------------
%%%-------------------------------------------------------
%%%   center left no-mathmode
%%%   (tabular mit einer Spalte)
%%%-------------------------------------------------------
%%%
%%\newcommand{\mcentn}[1]{%
%%\def\ls{\\\noalign{\smallskip}}
%%\def\lm{\\\noalign{\medskip}}
%%\begin{displaymath}
%%\begin{tabular}{l}
%%%\openup10pt
%%#1
%%\end{tabular}
%%\end{displaymath}}
%%%-- f1 -------------------------------------------------------------
%%\newcommand{\sca}{%
%% \begin{picture}(285,35)(0,0)
%%\put(65,10){\line(3,2){22.5}}
%%\put(125,10){\line(-3,2){22.5}}
%%  \linethickness{0 mm}
%%  \put(75,25){\framebox(40,10){%
%%    {\bf scalar}$^{(2)}$
%%  }}
%%  \put(55,0){\framebox(20,10){%
%%    $type$
%%  }}
%%  \put(115,0){\framebox(20,10){%
%%  $value$
%%  }}
%% \end{picture}
%%  }
%%%-- f1 -------------------------------------------------------------
%%\newcommand{\vect}{%
%% \begin{picture}(285,75)(0,0)
%%%-------------------------------------------------------------------
%%\put(70,40){\line(2,1){30}}
%%\put(170,35){\line(-5,2){50}}
%%%-------------------------------------------------------------------
%%\put(140,10){\line(2,1){30}}
%%\put(210,10){\line(-2,1){30}}
%%%-------------------------------------------------------------------
%%  \linethickness{0 mm}
%%  \put(90,55){\framebox(40,10){%
%%    {\bf vector}$^{(2)}$
%%  }}
%%  \put(60,30){\framebox(20,10){%
%%    $type$
%%  }}
%%  \put(165,25){\framebox(20,10){%
%%    $<^{(d)}$
%%  }}
%%  \put(130,0){\framebox(20,10){%
%%    $value\_1$
%%  }}
%%  \put(170,0){\framebox(10,10){%
%%    $\ldots$
%%  }}
%%  \put(200,0){\framebox(20,10){%
%%    $value\_d$
%%  }}
%% \end{picture}
%%}
%%%-- f1 -------------------------------------------------------------
%%\newcommand{\tre}{%
%% \begin{picture}(285,45)(0,0)
%%%-------------------------------------------------------------------
%%\put(75,10){\line(2,1){30}}
%%\put(145,10){\line(-2,1){30}}
%%%-------------------------------------------------------------------
%%  \linethickness{0 mm}
%%  \put(100,25){\framebox(20,10){%
%%% \put(140,25){%plus 40 !!!
%%    $<^{(n)}$
%%  }}
%%  \put(60,0){\framebox(30,10){%
%%    $expr\_1$
%%  }}
%%  \put(100,0){\framebox(20,10){%
%%    $\cdots$
%%  }}
%%  \put(130,0){\framebox(30,10){%
%%    $expr\_n$
%%  }}
%% \end{picture}
%%} }
%%%-------------------------------------------------------------------
%----- mac07 : last of 74 lines ------------------------------------
%%::::::::::::::
%%mac08.tex
%%::::::::::::::
%%::::::::::::::
%%mac0809.tex
%%::::::::::::::
%----------- mac0809 : first of 26 lines --------------------------
%-- text und f86.tex --------------------
% `mm' auswechseln gegen `ex' o.ae. !!!!
%----------------------------------------
\newcommand{\npi}{%
%%\mbox{\boldmath
%%$/\!\!/\!\!\raisebox{3.1 mm}{\_}\!\raisebox{3.1 mm}{\_}\,$}
\mbox{\boldmath
%$/\!\!/\!\!\raisebox{0.75 em}{\_}\!\raisebox{0.75 em}{\_}\,$}
$/\!\!/\!\!\raisebox{1.65 ex}{\_}\!\raisebox{1.65 ex}{\_}\,$}
}
\newcommand{\nbnpi}{%
%%\mbox{%
%%$/\!\!/\!\!\raisebox{3.1 mm}{\_}\!\raisebox{3.1 mm}{\_}\,$}
\mbox{%
$/\!\!/\!\!\raisebox{1.65 ex}{\_}\!\raisebox{1.65 ex}{\_}\,$}
}
%---------------------------------------------------------------------
%  stack (picture) length 90
%---------------------------------------------------------------------
\newcommand{\stackhe}[2]{%
%% \linethickness{0mm}
%% \framebox(90,10){%
%%   \thinlines
%%   \begin{minipage}{90\unitlength}
\begin{picture}(90,10)
\thicklines
% Stack E
\put(3,1){\line(1,0){80}}
\put(3,9){\line(1,0){76}}
\put(83,1){\line(-1,2){4}}
\put(84,0){\small  #1}
\linethickness{0mm}
\put(5,1){\framebox(74,8)[l]{\small {#2}}}
\thinlines
\end{picture}
%% \end{minipage}
%% }
}
%-------------------------------------------------------
%   array mit zwei Spalten und {%links
%   (no math mode)
%-------------------------------------------------------
%
\newcommand{\mleftlpn}[1]{%
\def\ls{\\\noalign{\smallskip}}
\def\lm{\\\noalign{\medskip}}
\vbox{%
$
\left\{% 
\begin{tabular}{lp{9.5cm}} % wg. 10pt fuer mit !!
%begin{tabular}{lp{12 cm}}
#1
\end{tabular}
\right.
$
}}
%----------- mac0809 : last of 26 lines --------------------------
%%::::::::::::::
%%mac09.tex
%%::::::::::::::
%-- sollte in macall bleiben !! ------------------
%-------------------------------------
\newcommand{\DEL}{{\bf \Delta}}
%-------------------------------------
\newcommand{\etasign}{$\eta$}
%-------------------------------------
\newcommand{\apm}{\mbox{\bf @\hspace*{-1ex}\rule[-.7ex]{0.5pt}{3ex}\hspace*{1.25ex}}\,}
\newcommand{\apmh}[1]{\mbox{\bf @\hspace*{-1ex}\rule[-.7ex]{0.5pt}{3ex}\hspace*{0.75ex}}^{#1}}
%-------------------------------------
\newcommand{\NALS}{{\it alphas}}
\newcommand{\NLAS}{{\it lambdas}}
\newcommand{\NAPS}{{\it apps}}
%----------------------------------------
%%::::::::::::::
%%mac11.tex
%%::::::::::::::
%-- auch kap8 --------------------------------------------------------
\newcommand{\marklt}[1]{%
\begin{picture}(15,4)
\thinlines
\put(15,2){\circle*{1}}
\put(15,2){\line(-1,0){5}}
\linethickness{0 mm}
\put(0,0){\framebox(10,4)[r]{\footnotesize #1~}}
\thinlines
\end{picture}
}
%-- auch kap8, kap4 --------------------------------------------------
\newcommand{\markrt}[1]{%
\begin{picture}(15,4)
\thinlines
\put(0,2){\circle*{1}}
\put(0,2){\line(1,0){5}}
\linethickness{0 mm}
\put(5,0){\framebox(10,4)[l]{\footnotesize ~#1}}
\thinlines
\end{picture}
}
%---------------------------------------------------------------------
%----- mac11 : first of 545 lines -----------
\newcommand{\SEMPTY}{\begin{picture}(4.0,2.0)
\thinlines
\put(0,0){\usebox{\slantzwei}}
\put(2,0){\usebox{\slantbzwei}}
\thinlines
\put(0,0){\line(1,0){1.3}}
\put(4,0){\line(-1,0){1.3}}
\end{picture}
}
%---------------------------------------------------------------------
\newsavebox{\slanttodrei}
%setlength{\unitlength}{1mm}
\savebox{\slanttodrei}(3,3)[c]{ %setlength{\unitlength}{1mm}
\begin{picture}(3,3)
\thinlines
\multiput(3,3)(-0.1,-0.05){30}{\line(1,0){0.1}}
\end{picture}
}
%---------------------------------------------------------------------
\newsavebox{\slanttwotwo}
%setlength{\unitlength}{1mm}
\savebox{\slanttwotwo}(2.2,2.2)[c]{ %setlength{\unitlength}{1mm}
\begin{picture}(2.2,2.2)
\thinlines
\multiput(0,0)(0.1,0.1){22}{\line(1,0){0.1}}
\end{picture}
}
%---------------------------------------------------------------------
\newsavebox{\slantdtwotwo}
%setlength{\unitlength}{1mm}
\savebox{\slantdtwotwo}(2.2,2.2)[c]{ %setlength{\unitlength}{1mm}
\begin{picture}(2.2,2.2)
\thinlines
\multiput(0,2.2)(0.1,-0.1){22}{\line(1,0){0.1}}
\end{picture}
}
%---------------------------------------------------------------------
\newsavebox{\slantdonesix}
%setlength{\unitlength}{1mm}
\savebox{\slantdonesix}(2.2,2.2)[c]{ %setlength{\unitlength}{1mm}
\begin{picture}(2.2,2.2)
\thinlines
\multiput(0,2.2)(0.1,-0.1){16}{\line(1,0){0.1}}
\end{picture}
}
%---------------------------------------------------------------------
\newsavebox{\slantddrei}
%setlength{\unitlength}{1mm}
\savebox{\slantddrei}(3,3)[c]{ %setlength{\unitlength}{1mm}
\begin{picture}(3,3)
\thinlines
\multiput(0,3)(0.1,-0.1){30}{\line(1,0){0.1}}
\end{picture}
}
%---------------------------------------------------------------------
\newsavebox{\slantbeins}
%setlength{\unitlength}{1mm}
\savebox{\slantbeins}(1,1)[c]{ %setlength{\unitlength}{1mm}
\begin{picture}(1,1)
\thinlines
\multiput(1,0)(-0.1,0.1){10}{\line(1,0){0.1}}
%multiput(2,0)(-0.125,0.125){16}{\circle*{0.5}}
\end{picture}
}
%---------------------------------------------------------------------
\newsavebox{\slantbzwei}
%setlength{\unitlength}{1mm}
\savebox{\slantbzwei}(2,2)[c]{ %setlength{\unitlength}{1mm}
\begin{picture}(2,2)
\thinlines
\multiput(2,0)(-0.1,0.1){20}{\line(1,0){0.1}}
%multiput(2,0)(-0.125,0.125){16}{\circle*{0.5}}
\end{picture}
}
%---------------------------------------------------------------------
\newsavebox{\slantbdrei}
%setlength{\unitlength}{1mm}
\savebox{\slantbdrei}(3,3)[c]{ %setlength{\unitlength}{1mm}
\begin{picture}(3,3)
\thinlines
\multiput(3,0)(-0.1,0.1){30}{\line(1,0){0.1}}
\end{picture}
}
%---------------------------------------------------------------------
\newsavebox{\slantbvier}
%setlength{\unitlength}{1mm}
\savebox{\slantbvier}(4,4)[c]{ %setlength{\unitlength}{1mm}
\begin{picture}(4,4)
\thinlines
\multiput(4,0)(-0.1,0.1){40}{\line(1,0){0.1}}
\end{picture}
}
%---------------------------------------------------------------------
\newsavebox{\slantbdreise}
%setlength{\unitlength}{1mm}
\savebox{\slantbdreise}(4,4)[c]{ %setlength{\unitlength}{1mm}
\begin{picture}(4,4)
\thinlines
\multiput(4,0)(-0.1,0.1){36}{\line(1,0){0.1}}
\end{picture}
}
%---------------------------------------------------------------------
\newsavebox{\slanteins}
%setlength{\unitlength}{1mm}
\savebox{\slanteins}(1,1)[c]{ %setlength{\unitlength}{1mm}
\begin{picture}(1,1)
\thinlines
\multiput(0,0)(0.1,0.1){10}{\line(1,0){0.1}}
%multiput(0,0)(0.125,0.125){16}{\circle*{0.5}}
\end{picture}
}
%---------------------------------------------------------------------
\newsavebox{\slantzwei}
%setlength{\unitlength}{1mm}
\savebox{\slantzwei}(2,2)[c]{ %setlength{\unitlength}{1mm}
\begin{picture}(2,2)
\thinlines
\multiput(0,0)(0.1,0.1){20}{\line(1,0){0.1}}
%multiput(0,0)(0.125,0.125){16}{\circle*{0.5}}
\end{picture}
}
%---------------------------------------------------------------------
\newsavebox{\slantdrei}
%setlength{\unitlength}{1mm}
\savebox{\slantdrei}(3,3)[c]{ %setlength{\unitlength}{1mm}
\begin{picture}(3,3)
\thinlines
\multiput(0,0)(0.1,0.1){30}{\line(1,0){0.1}}
\end{picture}
}
%---------------------------------------------------------------------
\newsavebox{\slantvier}
%setlength{\unitlength}{1mm}
\savebox{\slantvier}(4,4)[c]{ %setlength{\unitlength}{1mm}
\begin{picture}(4,4)
\thinlines
\multiput(0,0)(0.1,0.1){40}{\line(1,0){0.1}}
\end{picture}
}
%---------------------------------------------------------------------
\newsavebox{\slantdreise}
%setlength{\unitlength}{1mm}
\savebox{\slantdreise}(4,4)[c]{ %setlength{\unitlength}{1mm}
\begin{picture}(4,4)
\thinlines
\multiput(0,0)(0.1,0.1){36}{\line(1,0){0.1}}
\end{picture}
}
%%::::::::::::::
%%macall.tex
%%::::::::::::::
%-------- macall.tex : first of 128 lines --------------------------
\newcounter{LAS}
\newcounter{APS}
%---------------------------------------------------------------------
\newcommand{\nlet}[1]{\mbox{$\not{\!#1}$}}
%---------------------------------------------------------------------
%math. Zeichen in boldmath
\newcommand{\MATH}[1]{\mbox{{\boldmath $#1$}}}
%---------------------------------------------------------------------
\newcommand{\idots}{\mathinner{\ldotp\ldotp}}
%---------------------------------------------------------------------
\makeatletter
%-------------------------------
\def\ivdots{\vbox{\baselineskip4\p@ \lineskiplimit\z@
\kern6\p@\hbox{.}\hbox{.}}}
%-------------------------------
\makeatother
%---------------------------------------------------------------------
\newdimen\descripdim
\newenvironment{descrip}[1]{%
\def\descrlabelwidth{.2\linewidth}
\def\mlabel##1{##1\hfil}
\list{#1}{%
\let\makelabel\mlabel
\settowidth\descripdim{#1}
\ifdim \descripdim >\labelwidth
\ifdim \descripdim >\descrlabelwidth
\descripdim\descrlabelwidth\fi
\labelwidth\descripdim \leftmargin\descripdim
\advance\leftmargin\labelsep
\else \labelwidth\descripdim\fi
}
}%
{\endlist}
%---------------------------------------------------------------------
% bf \mbox{\bf @} und lambda
%---------------------------------------------------------------------
\newcommand{\lamvee}{\stackrel{\vee}{\lambda}}
\newcommand{\aph}[1]{{\mbox{\bf @}}^{#1}}
\newcommand{\apsh}[1]{\stackrel{\sim}{\mbox{\bf @}}^{#1}}
\newcommand{\lash}[1]{\stackrel{\sim}{\mbox{\boldmath $\lambda$}}^{#1}\!\!\!}
\newcommand{\las}{\stackrel{\sim}{\mbox{\boldmath $\lambda$}}\!}
\newcommand{\Las}{\stackrel{\sim}{\mbox{$\bf \Lambda$}}}
\newcommand{\LA}{\mbox{$\bf \Lambda$}}
\newcommand{\ap}{\mbox{\bf @}}
\newcommand{\kon}[1]{\mbox{\bf #1}}
\newcommand{\apq}{\overline{\mbox{\bf @}}}
\newcommand{\apqh}[1]{\overline{\mbox{\bf @}}^{#1}}
\newcommand{\apqs}{\overline{\mbox{\bf @}^{*}}}
\newcommand{\apns}{\mbox{\bf @}^{*}}
\newcommand{\aps}{\stackrel{\sim}{\mbox{\bf @}}}
%-- nur aus kap6 ?? --------------------------------------------------
\newcommand{\appq}{\overline{\mbox{\bf @}_p}}
\newcommand{\app}{\mbox{\bf @}_p}
\newcommand{\apph}[1]{\mbox{\bf @}_p^{#1}}
%-- kap6, kap8 -----------------------
%\newcommand{\NA}{\bf \nabla}
\newcommand{\NA}{\MATH{\nabla}}
\newcommand{\nbNA}{{\nabla}}
%-------------------------------------------------------
%   Symbole fuer `Menge der natuerlichen Zahlen'
%   \nz : mathmode
%   \nat : roman
%   \nata : small caps
%-------------------------------------------------------
%
\def\nz{\relax\ifmmode {I\hskip -3pt N}
\else{\hbox{$I\hskip -3pt N$}}\fi}
%12 pt : \def\nz{\relax\ifmmode {I\hskip -3.5pt N}
%12 pt : \else{\hbox{$I\hskip -3.5pt N$}}\fi}
%---------------------------------
%def\nat{\relax\ifmmode {\rm I\hskip -1.5pt N}
%else{I\hskip -1.5pt N}\fi}
%---------------------------------
%def\nata{\relax\ifmmode {\hbox{\sc I\hskip -2.0pt N}}
%else{\sc I\hskip -2.0pt N}\fi}
%----------------------------------------------------------------
%input{berel}
%------------------------------------------------------------------
% THE \LaTeX LOGO IS DEFINED HERE.
%------------------------------------------------------------------
%\def\LaTeX{{\rm L\kern-.36em\raise.3ex\hbox{\sc a}\kern-.15em
%    T\kern-.1667em\lower.7ex\hbox{E}\kern-.125emX}}
%------------------------------------------------------------------
% THE \HLDFL LOGO IS DEFINED HERE.
%------------------------------------------------------------------
%newcommand{\HLDFL}{{\sc HlDFl}}
\newcommand{\HLDFL}{{\sc hl}%
\kern-.1em{\sc D}\kern-.1em%
\raise.3ex\hbox{\sc f}\kern-.1em{\sc l}\kern.2em}
%------------------------------------------------------------------
% THE \HLFL LOGO IS DEFINED HERE.
%------------------------------------------------------------------
%newcommand{\HLFL}{{\sc HlFl}}
\newcommand{\HLFL}{{\sc hl}\kern-.1em{\sc F}\kern-.1em{\sc l}\kern.2em}
%------------------------------------------------------------------
% THE \BeReL LOGO IS DEFINED HERE.
%------------------------------------------------------------------
%%\newcommand{\BeReL}{{\protect\rm B\protect\kern-.1em\protect\raise.3ex%
%%\protect\hbox{\protect\sc e}\protect\kern-.1em
%%R\protect\kern-.125em\protect\raise.3ex\protect\hbox{\protect\sc e}%
%%\protect\kern-.1emL\protect\kern.2em}}
%\newcommand{\BeReL}{{\protect\rm B\protect\kern-.1em\protect\raise.3ex%
%\protect\hbox{\protect\sc e}\protect\kern-.17em
%    R\protect\kern-.1667em\protect\raise.3ex%
%\protect\hbox{\protect\sc e}\protect\kern-.125emL\protect\kern.2em}}
%
%\newcommand{\BeReLb}{{\rm B\kern-.1em\raise.3ex\hbox{\sc e}\kern-.17em
%    R\kern-.1667em\lower.7ex\hbox{E}\kern-.125emL\kern.2em}}
%
%\newcommand{\BeReLa}{{\rm B\kern-.1em{\sc e}\kern-.1em
%    R\kern-.1em{\sc e}\kern-.1emL\kern.2em}}
%
%\newcommand{\BeReLc}{{\rm B\kern-.1em\hbox{\sc e}\kern-.1em
%    R\kern-.1667em\lower.7ex\hbox{E}\kern-.125emL\kern.2em}}
%-- kap6, kap11 -------------------------------------
%\newsavebox{\algbox}
%\setlength{\unitlength}{1mm}
%\savebox{\algbox}(15,6)[c]{\setlength{\unitlength}{1mm}
\newcommand{\algbox}[1]{%
\begin{picture}(15,6)
\thinlines
\put(0,6){\line(1,0){15}}
\put(0,0){\line(1,0){15}}
\put(0,3){\oval(6,6)[l]}
\put(15,3){\oval(6,6)[r]}
\linethickness{0 mm}
\put(0,0){\framebox(15,6)[c]{\small #1}}
\end{picture}
}
%----------------------------------------------------
%-------- macall.tex : last of 128 lines ----------------------------
%%::::::::::::::
%%macdia.tex
%%::::::::::::::
%%::::::::::::::
%%macnet.tex : jetzt mac0203
%%::::::::::::::
%%::::::::::::::
%%macsave.tex
%%::::::::::::::
%----------- allmac : first of 3654 lines -----------
%--------- last of 597 lines --------------------------
